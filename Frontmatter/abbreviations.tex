\newacronym[
	shortplural={stades NF},
	longplural={stades Nieukoop-Faber},
	plural={stades NF}]
	{snf}
	{stade NF}
	{stade Nieukoop-Faber}
\newacronym
	{xlaevis}
	{\textit{X. laevis}}
	{\textit{Xenopus laevis}}
\newacronym
	{xtrop}
	{\textit{X. tropicalis}}
	{\textit{Xenopus tropicalis}}
%
\newacronym
	{sns}
	{SNS}
	{système nerveux sympathique}
\newacronym
	{snc}
	{SNC}
	{système nerveux central}
\newacronym
	{atp}
	{ATP}
	{adénosine triphosphate}
\newacronym[
	longplural={acides désoxyribonucléiques},
	shortplural={ADN},
	plural={ADN}]
	{dna}
	{ADN}
	{acide désoxyribonucléique}
\newacronym[
	shortplural={ADNc},
	plural={ADNc},
	longplural={ADN complémentaires}]
	{cdna}
	{ADNc}
	{ADN complémentaire}
\newacronym[
	longplural={acides ribonucléiques},
	shortplural={ARN},
	plural={ARN}]
	{rna}
	{ARN}
	{acide ribonucléique}
\newacronym[
	longplural={ARN messagers},
	shortplural={ARNm},
	plural={ARNm}]
	{mrna}
	{ARNm}
	{ARN messager}
\newacronym[
	longplural={ARN ribosomaux},
	shortplural={ARNr},
	plural={ARNr}]
	{rrna}
	{ARNr}
	{ARN ribosomal}
\newacronym[
	longplural={ARN non-codants},
	shortplural={ARNnc},
	plural={ARNnc}]
	{ncrna}
	{ARNnc}
	{ARN non-codant}
\newacronym[
	longfr={site d'initiation de la transcription},
	longplfr={sites d'initiation de la transcription}]
	{tss}
	{TSS}
	{transcription start site}
\newacronym[
	longfr={site de terminaison de la transcription},
	longplfr={sites de terminaison de la transcription}]
	{tts}
	{TTS}
	{transcription terminaison site}
\newacronym
	{icpg}
	{iCpG}
	{îlot CpG}
\newacronym[
	shortplural={Mpb},
	longplural={méga paires de bases},
	plural={Mpb}]
	{mpb}
	{Mpb}
	{méga paire de bases}
\newacronym[
	shortplural={pb},
	longplural={paires de bases},
	plural={pb}]
	{pb}
	{pb}
	{paire de bases}
\newacronym
	{Na}
	{Na}
	{sodium}
\newacronym
	{K}
	{K}
	{potassium}
\newacronym[
	longfr={syndrome d'adaptation générale}]
	{gas}
	{GAS}
	{general adaptation syndrome}
%
\newacronym
	{gc}
	{GC}
	{glucocorticoïde}
\newacronym
	{mc}
	{MC}
	{minéralocorticoïde}
\newacronym
	{dex}
	{DEX}
	{dexamethasone}
\newacronym
	{cort}
	{CORT}
	{corticostérone}
\newacronym
	{doc}
	{DOC}
	{déoxycorticostérone}
%
\newacronym[
	longplural={hormones thyroïdiennes}]
	{ht}
	{HT}
	{hormone thyroïdienne}
\newacronym
	{t4}
	{T\textsubscript{4}}
	{thyroxine}
\newacronym
	{t3}
	{T\textsubscript{3}}
	{3,5,3'-triiodothyronine}
\newacronym
	{rt3}
	{rT\textsubscript{3}}
	{3,3',5'-triiodothyronine}
\newacronym
	{t2}
	{3,3'-T\textsubscript{2}}
	{3,3'-diiodothyronine}
\newacronym
	{tetrac}
	{TETRAC}
	{acide tetraidothyroacétique}
\newacronym[
	longfr={désiodation du cycle externe}]
	{ord}
	{ORD}
	{outer ring deiodination}
\newacronym[
	longfr={désiodation du cycle interne}]
	{ird}
	{IRD}
	{inner ring deiodination}
%
\newacronym
	{mit}
	{MIT}
	{monoiodotyrosine}
\newacronym
	{dit}
	{DIT}
	{diiodotyrosine}
\newacronym
	{ttr}
	{TTR}
	{transthyrétine}
\newacronym[
	longfr={globine liant la thyroxine},
	longplfr={globines liant la thyroxine}]
	{tbg}
	{TBG}
	{thyroxine binding globin}
%
\newacronym
	{tsh}
	{TSH}
	{thyréostimuline hypophysaire}
\newacronym
	{trh}
	{TRH}
	{thyrolibérine}
\newacronym
	{hpt}
	{HPT}
	{hypothalamo-hypophyso-thyroïdien}
%
\newacronym[
	longfr={globuline liant les corticostéroïdes},
	longplfr={globuline liant les corticostéroïdes}]
	{cbg}
	{CBG}
	{corticosteroid binding globulin}
\newacronym
	{hpa}
	{HPA}
	{hypothalamo-hypophyso-adrénal}
\newacronym
	{hpi}
	{HPI}
	{hypothalamo-hypophyso-interrénal}
\newacronym
	{pomc}
	{POMC}
	{proopiomélanocortine}
\newacronym
	{acth}
	{ACTH}
	{adrénocorticotrophine}
\newacronym[longplural=CRH]
	{crh}
	{CRH}
	{corticolibérine}
%
\newacronym[
	longplural={désiodases de type I}]
	{dio1}
	{DIO1}
	{désiodase de type I}
\newacronym[
	longplural={désiodases de type II}]
	{dio2}
	{DIO2}
	{désiodase de type II}
\newacronym[
	longplural={désiodases de type III}]
	{dio3}
	{DIO3}
	{désiodase de type III}
\newacronym[
	longfr={protéine cytosolique liant les hormones thyroïdiennes},
	longplfr={protéines cytosoliques liant les hormones thyroïdiennes}]
	{cthbp}
	{CTHBP}
	{cytosolic thyroid hormone binding protein}
%
\newacronym[
	longplural={transporteurs hétérodimériques d'acides aminés}]
	{haat}
	{HAAT}
	{transporteur hétérodimérique d'acides aminés}
\newacronym[
	longfr={polypeptide transporteur d'anions organiques},
	longplfr={polypeptides transporteurs d'anions organiques}]
	{oatp}
	{OATP}
	{organic anions transporter polypeptide}
\newacronym[
	longfr={transporteur d'acides aminés de type L},
	longplfr={transporteurs d'acides aminés de type L}]
	{lat}
	{LAT}
	{L-type amino-acid transporter}
\newacronym[
	longfr={transporteur d'ions monocarboxylate},
	longplfr={transporteurs d'ions monocarboxylate},
	longplural={monocarboxylate transporters}]
	{mct}
	{MCT}
	{monocarboxylate transporter}
%
\newacronym[
	longplural={récepteurs nucléaires}]
	{rn}
	{RN}
	{récepteur nucléaire}
\newacronym[
	longplural={récepteurs à l'acide 9-\textit{cis}-rétinoïque}]
	{rxr}
	{RXR}
	{récepteur à l'acide 9-\textit{cis}-rétinoïque}
\newacronym[
	longplural={récepteurs à l'acide all-\textit{trans}-rétinoïque}]
	{rar}
	{RAR}
	{récepteur à l'acide 9-\textit{cis}-rétinoïque}
\newacronym[
	longplural={récepteurs à la vitamine D}]
	{vdr}
	{VDR}
	{récepteur à la vitamine D}
\newacronym[
	longplural={récepteurs aux œstrogènes}]
	{er}
	{ER}
	{récepteur aux œstrogènes}
\newacronym[
	longfr={facteur de transcription du promoteur en amont de l'ovalbumine de poulet},
	longplfr={facteurs de transcription du promoteur en amont de l'ovalbumine de poulet}]
	{couptf}
	{COUP-TF}
	{chicken ovalbumin upstream promoter transcription factor}
\newacronym[
	longfr={facteur nucléaire hépatique 4},
	longplfr={facteurs nucléaires hépatiques 4},
	longplural={hepatic nuclear factors 4}]
	{hnf4}
	{HNF-4}
	{hepatic nuclear factor 4}
\newacronym[
	longfr={récepteur activé par les proliférateurs de peroxisomes},
	longplfr={récepteurs activés par les proliférateurs de peroxisomes}]
	{ppar}
	{PPAR}
	{peroxisome proliferator activated receptor}
\newacronym[
	longplural={récepteurs aux androgènes}]
	{ar}
	{AR}
	{récepteur aux androgènes}
\newacronym[
	longfr={facteur de croissance nerveuse 1 B}]
	{ngf1b}
	{NGF-1B}
	{nerve growth factor 1 B}
\newacronym[
	longfr={protéine reliée aux récepteurs nucléaires 1}]
	{nurr1}
	{NURR-1}
	{nuclear receptor related 1 protein}
\newacronym[
	longfr={récepteur orphelin-1 dérivé du neurone}]
	{nor1}
	{NOR-1}
	{neuron-derived orphan receptor 1}
\newacronym[
	longfr={facteur nucléaire des cellules germinales}]
	{gcnf}
	{GCNF}
	{germ cell nuclear factor}
\newacronym[
	longfr={homologue-1 du récepteur hépatique}]
	{lrh1}
	{LRH-1}
	{liver receptor homologue-1}
\newacronym[
	longfr={facteur stéroïdogénique 1}]
	{sf1}
	{SF-1}
	{steroidogenic factor 1}
\newacronym[
	longfr={protéine $\alpha$ de liaison à l'enhanceur/CCAAT}]
	{cebpa}
	{CEBP$\alpha$}
	{CCAAT/enhancer-binding protein $\alpha$}

%
\newacronym[
	longplural={récepteurs aux \glspl{ht}}]
	{tr}
	{TR}
	{récepteur aux \glspl{ht}}
\newacronym[
	longplural={TRs, types $\alpha$}]
	{tra}
	{TR$\alpha$}{TR, type $\alpha$}
\newacronym[
	longplural={TRs, type $\alpha$\, forme $\Delta$}]
	{trda}
	{TR$\Delta\alpha$}
	{TR, type $\alpha$\, forme $\Delta$}
\newacronym[
	longplural={TRs, type $\beta$}]
	{trb}
	{TR$\beta$}
	{TR, type $\beta$}
\newacronym[
	longplural={TRs, type $\beta$\, forme $\Delta$}]
	{trdb}
	{TR$\Delta\beta$}
	{TR, type $\beta$\, forme $\Delta$}
\newacronym[
	longplural={éléments de réponse aux HTs}]
	{tre}
	{TRE}
	{élément de réponse aux HTs}
\newacronym[
	longplural={éléments négatifs de réponse aux HTs}]
	{ntre}
	{nTRE}
	{élément négatif de réponse aux HTs}
\newacronym[
	longplural=intégrines $\alpha$\textsubscript{V}$\beta$\textsubscript{3}]
	{iab3}
	{$\alpha$\textsubscript{V}$\beta$\textsubscript{3}}
	{intégrine $\alpha$\textsubscript{V}$\beta$\textsubscript{3}}
\newacronym[
	longfr={kinase régulée par des signaux extra-cellulaires},
	longplfr={kinases régulées par des signaux extra-cellulaires}]
	{erk}
	{ERK}
	{extracellular-signal regulated kinase}
\newacronym[
	longfr={kinase des protéines activées par des mitogènes},
	longplfr={kinases des protéines activées par des mitogènes}]
	{mapk}
	{MAPK}
	{mitogen-activated-protein kinase}
\newacronym[
	longfr={coactivateur de PPAR$\gamma$}]
	{pgc1a}
	{PGC-1$\alpha$}
	{peroxisome proliferator-activated receptor gamma co-activator}
%
\newacronym
	{verbA}
	{\textit{v-erbA}}
	{oncogène du virus de l'érythroblastose aviaire}
\newacronym
	{cerbA}
	{\textit{c-erbA}}
	{homologue cellulaire de l'oncogène du virus de l'érythroblastose aviaire}
\newacronym[
	longfr={hormone de croissance},
	longplfr={hormones de croissance}]
	{gh}
	{GH}
	{growth hormone}
\newacronym
	{coa}
	{CoA}
	{coenzyme A}
\newacronym
	{hmg}
	{HMG}
	{3-hydroxy-3-methylglutaryl}
%
\newacronym[
	longfr={palindrome séparé par 0 pb},
	longplfr={palindromes séparés par 0 pb},
	longplural={inverted repeats separated by 0 bp}]
	{ir0}
	{IR0}
	{inverted repeat separated by 0 bp}
\newacronym[
	longfr={palindrome séparé par 1 pb},
	longplfr={palindromes séparés par 1 pb},
	longplural={inverted repeats separated by 1 bp}]
	{ir1}
	{IR1}
	{inverted repeat separated by 1 bp}
\newacronym[
	longfr={palindrome séparé par 2 pb},
	longplfr={palindromes séparés par 2 pb},
	longplural={inverted repeats separated by 2 bp}]
	{ir2}
	{IR2}
	{inverted repeat separated by 2 bp}
\newacronym[
	longfr={palindrome séparé par 3 pb},
	longplfr={palindromes séparés par 3 pb},
	longplural={inverted repeats separated by 3 bp}]
	{ir3}
	{IR3}
	{inverted repeat separated by 3 bp}
\newacronym[
	longfr={palindrome inversé séparé par 3 pb},
	longplfr={palindromes inversés séparés par 3 pb},
	longplural={everted repeats separated by 3 bp}]
	{er3}
	{ER3}
	{everted repeat separated by 3 bp}
\newacronym[
	longfr={palindrome inversé séparé par 6 pb},
	longplfr={palindromes inversés séparés par 6 pb},
	longplural={everted repeats separated by 6 bp}]
	{er6}
	{ER6}
	{everted repeat separated by 6 bp}
\newacronym[
	longfr={répétition en tandem séparée par 0 pb},
	longplfr={répétitions en tandem séparées par 0 pb},
	longplural={direct repeats separated by 0 bp}]
	{dr0}
	{DR0}
	{direct repeat separated by 0 bp}
\newacronym[
	longfr={répétition en tandem séparée par 4 pb},
	longplfr={répétitions en tandem séparées par 4 pb},
	longplural={direct repeats separated by 4 bp}]
	{dr4}
	{DR4}
	{direct repeat separated by 4 bp}
%
\newacronym[
	longplural={récepteurs aux \glspl{gc}}]
	{gr}
	{GR}
	{récepteur aux \glspl{gc}}
\newacronym[
	longplural={GRs, isoforme $\alpha$}]
	{gra}
	{GR$\alpha$}
	{GR, isoforme $\alpha$}
\newacronym[
	longplural={GRs, isoforme $\beta$}]
	{grb}
	{GR$\beta$}
	{GR, isoforme $\beta$}
\newacronym[
	longplural={éléments de réponse aux GCs}]
	{gre}
	{GRE}
	{élément de réponse aux GCs}
\newacronym[
	longplural={éléments de réponse négatifs aux GCs}]
	{ngre}
	{nGRE}
	{élement de réponse négatif aux GCs}
\newacronym[
	longplural={éléments de réponse composites aux GCs}]
	{cgre}
	{cGRE}
	{élément de réponse composite aux GCs}
\newacronym[
	longplural={récepteurs aux MCs}]
	{mr}{MR}{récepteur aux MCs}
%
\newacronym[
	longfr={domaine de liaison à l'ADN},
	longplfr={domaines de liaison à l'ADN}]
	{dbd}
	{DBD}
	{DNA binding domain}
\newacronym[
	longfr={domaine d’activation de la transcription ligand-indépendant},
	longplfr={domaines d’activation de la transcription ligand-indépendant}]
	{af1}
	{AF-1}
	{activation function-1}
\newacronym[
	longfr={domaine d’activation de la transcription ligand-dépendant},
	longplfr={domaines d’activation de la transcription ligand-dépendant}]
	{af2}
	{AF-2}
	{activation function-2}
\newacronym[
	longfr={domaine de liaison du ligand},
	longplfr={domaines de liaison du ligand}]
	{lbd}
	{LBD}
	{ligand binding domain}
\newacronym[
	longfr={domaine N-terminal},
	longplfr={domaines N-terminaux}]
	{ntd}
	{NTD}
	{N-terminal domain}
%
\newacronym
	{dhea}
	{DHEA}
	{dehydroépiandrostérone}
\newacronym
	{17aohpreg}
	{17$\alpha$-OH-pregnénolone}
	{17$\alpha$-hydroxypregnénolone}
\newacronym
	{17aohprog}
	{17$\alpha$-OH-progestérone}
	{17$\alpha$-hydroxyprogestérone}
%
\newacronym
	{cyp17a1}
	{CYP17A1}
	{17$\alpha$-hydroxylase}
\newacronym
	{17aohase}
	{17$\alpha$-OHase}
	{17$\alpha$-hydroxylase}
\newacronym
	{cyp11b1}
	{CYP11B1}
	{11$\beta$-hydroxylase}
\newacronym
	{11bohase}
	{11$\beta$-OHase}
	{11$\beta$-hydroxylase}
\newacronym
	{cyp21a2}
	{CYP21A2}
	{21-hydroxylase}
\newacronym
	{21ohase}
	{21-OHase}
	{21-hydroxylase}
\newacronym
	{cyp11b2}
	{18$\beta$-OHase}
	{18$\beta$-hydroxylase}
\newacronym
	{18bohase}
	{18$\beta$-OHase}
	{18$\beta$-hydroxylase}
\newacronym
	{cyp11a1}
	{CYP11A1}
	{enzyme de clivage de la chaîne latérale du cholestérol}
\newacronym
	{p450scc}
	{P450scc}
	{enzyme de clivage de la chaîne latérale du cholestérol}
\newacronym
	{3bhsd}
	{3-$\beta$-HSD}
	{3-$\beta$-hydroxysteroïde déhydrogenase/$\Delta$-5-4 isomérase}
\newacronym[
	longplural={11-$\beta$-hydroxystéroïde déhydrogenases/$\Delta$-5-4 isomérases, type 1}]
	{11bhsd1}
	{11-$\beta$-HSD1}
	{11-$\beta$-hydroxysteroïde déhydrogenase/$\Delta$-5-4 isomérase, type 1}
\newacronym[
	longplural={11-$\beta$-hydroxysteroïde déhydrogenases/$\Delta$-5-4 isomérases, type 2}]
	{11bhsd2}
	{11-$\beta$-HSD2}
	{11-$\beta$-hydroxysteroïde déhydrogenase/$\Delta$-5-4 isomérase, type 2}
%
\newacronym[
	longfr={lipoprotéine de faible densité},
	longfr={lipoprotéines de faible densité}]
	{ldl}
	{LDL}
	{low-density lipoprotein}
\newacronym[
	longfr={lipoprotéine de très faible densité},
	longfr={lipoprotéines de très faible densité}]
	{vldl}
	{VLDL}
	{very low-density lipoprotein}
\newacronym[
	longfr={lipoprotéine de haute densité},
	longfr={lipoprotéines de haute densité}]
	{hdl}
	{HDL}
	{high-density lipoprotein}
%
\newacronym[
	longfr={facteur de croissance similaires à l'insuline},
	longplfr={facteurs de croissance similaires à l'insuline}]
	{igf}
	{IGF}
	{insulin-like growth factor}
%
\newacronym[
	longplural={ARN polymérases II}]
	{rnapol2}
	{ARN-Pol-II}
	{ARN polymérase II}
\newacronym[
	longfr={protéine de liaison aux boites TATA},
	longplfr={protéines de liaison aux boites TATA}]
	{tbp}
	{TBP}
	{TATA-box binding protein}
\newacronym[
	longfr={protéine de liaison à CREB},
	longplfr={protéines de liaison à CREB}]
	{cbp}
	{CBP}
	{CREB binding protein}
\newacronym[
	longfr={coactivateur des récepteurs aux stéroïdes},
	longplfr={coactivateurs des récepteurs aux stéroïdes}]
	{src}
	{SRC}
	{steroid receptor coactivator}
\newacronym[
	longfr={protéine interagissant avec GR 1},
	longfr={protéines interagissant avec GR 1}]
	{grip1}
	{GRIP1}
	{GR-interacting protein 1}
\newacronym[
	longfr={protéine nucléaire interagissant avec TR 6},
	longplfr={protéines nucléaires interagissant avec TR 6}]
	{ntrip6}
	{NTRIP6}
	{nuclear TR-interacting protein 6}
\newacronym[
	longfr={proto-oncogènes codant pour des tyrosine kinases Src}]
	{Src}
	{Src}
	{Proto-oncogene tyrosine-protein kinase Src}
\newacronym[
	longfr={protéine de choc thermique de 90 kDa},
	longplfr={protéines de choc thermique de 90 kDa}]
	{hsp90}
	{HSP90}
	{heat-shock protein, 90 kDa}
\newacronym[
	longfr={protéine de choc thermique de 70 kDa},
	longplfr={protéines de choc thermique de 70 kDa}]
	{hsp70}
	{HSP70}
	{heat-shock protein, 70 kDa}
\newacronym[
	longfr={protéine de choc thermique de 56 kDa},
	longplfr={protéines de choc thermique de 56 kDa}]
	{hsp56}
	{HSP56}
	{heat-shock protein, 56 kDa}
\newacronym[
	longfr={protéine de choc thermique de 40 kDa},
	longplfr={protéines de choc thermique de 40 kDa}]
	{hsp40}
	{HSP40}
	{heat-shock protein, 40 kDa}
\newacronym[
	longfr={Corépresseur des récepteurs nucléaires},
	longplfr={Corépresseurs des récepteurs nucléaires}]
	{ncor}
	{NCoR}
	{nuclear receptors corepressor}
\newacronym[
	longfr={médiateur d'extinction de RAR et TR},
	longplfr={médiateurs d'extinction de RAR et TR}]
	{smrt}
	{SMRT}
	{silencing mediator for RAR and TR}
\newacronym
	{hdac}
	{HDAC}
	{histone déacétylase}
\newacronym
	{hat}
	{HAT}
	{histone acétyltransférase}
\newacronym[
	longfr={arginine méthyltransferase associée aux coactivateurs},
	longplfr={arginine méthyltransferases associées aux coactivateurs}]
	{carm}
	{CARM}
	{coactivator-associated arginine methyltransferase}
\newacronym[
	longfr={protéine activante 1},
	longplfr={protéines activantes 1}]
	{ap1}
	{AP-1}
	{activating protein 1}
\newacronym
	{nfkb}
	{NF-$\kappa$B}
	{facteur nucléaire $\kappa$B}
\newacronym
	{ikb}
	{I$\kappa$B}
	{inhibiteur de $\kappa$B}
\newacronym
	{il1}
	{IL-1}
	{interleukine 1}
\newacronym
	{il6}
	{IL-6}
	{interleukine 6}
\newacronym
	{il8}
	{IL-8}
	{interleukine 8}
\newacronym
	{il10}
	{IL-10}
	{interleukine 10}
\newacronym[
	longfr={transducteur du signal et activateur de la transcription 5},
	longplfr={transducteurs du signal et activateurs de la transcription 5},
	longplural={signal transducers and activators of transcription 5}]
	{stat5}
	{STAT-5}
	{signal transducer and activator of transcription 5}
\newacronym
	{tslp}
	{TSLP}
	{lymphopoiétine stromale thymique}
\newacronym[
	longfr={facteur de régulation des interferons 3},
	longplfr={facteurs de régulation des interferons 3},
	longplural={interferon regulatory factors 3}]
	{irf3}
	{IRF-3}
	{interferon regulatory factor 3}
\newacronym[
	longfr={facteur de régulation des interferons 7},
	longplfr={facteurs de régulation des interferons 7},
	longplural={interferon regulatory factors 7}]
	{irf7}
	{IRF-7}
	{interferon regulatory factor 7}
\newacronym[
	longfr={facteur positif d'élongation de la transcription b},
	longplfr={facteurs positifs d'élongation de la transcription b},
	longplural={positive transcription elongation factors b}]
	{ptefb}
	{PTEFb}
	{positive transcription elongation factor b}
\newacronym[
	longfr={facteur $\alpha$ de nécrose tumorale},
	longplural={facteurs $\alpha$ de nécrose tumorale}]
	{tnfa}
	{TNF$\alpha$}
	{tumor necrosis factor $\alpha$}
\newacronym
	{infg}
	{INF$\gamma$}
	{interféron $\gamma$}
\newacronym[
	longfr={molécule 1 d'adhésion des cellules vasculaires}]
	{vcam1}
	{VCAM-1}
	{vascular cell adhesion molecule 1}
\newacronym[
	longfr={CD40, membre 5 de la superfamille des récepteurs aux facteurs de nécrose tumorale}]
	{cd40}
	{CD40}
	{CD40, tumor necrosis factor receptor superfamily, member 5}
\newacronym[
	longfr={facteur cytosolique 1 des neutrophiles}]
	{ncf1}
	{NCF-1}
	{neutrophil cytosolic factor 1}
\newacronym
	{mhc}
	{MHC}
	{complexe majeur d'histocompatibilité}
%
\newacronym
	{pepck}
	{PEPCK}
	{phosphoénolpyruvate carboxykinase}
\newacronym[
	longfr={facteur similaire à krüppel 9}]
	{klf9}
	{KLF9}
	{krüppel-like factor 9}
\newacronym[
	longfr={facteur de transcription avec une glissière à leucine basique induit par les HTs}]
	{thbzip}
	{TH/bZIP}
	{thyroid hormone induced basic leucine zipper transcription factor}
%
\newacronym
	{ra}
	{RA}
	{arthrite rhumatoïde}
\newacronym
	{oa}
	{OA}
	{ostéoarthrose}
	
\newacronym[
	longfr={épiderme caudal},
	longplfr={épidermes caudaux}]
	{tf}
	{TF}
	{tailfin}	
\newacronym[
	longfr={bourgeon de membre postérieur},
	longplfr={bourgeons de membres postérieurs}]
	{hlb}
	{HLB}
	{hindlimb bud}
		
\newacronym
	{chx}
	{CHX}
	{cycloheximide}
	
\newacronym[
	longfr={réaction en chaîne de polymérase},
	longplfr={réactions en chaîne de polymérase}]
	{pcr}
	{PCR}
	{polymerase chain reaction}
\newacronym[
	longfr={réaction en chaîne de polymérase quantitative et en temps réel},
	longplfr={réactions en chaîne de polymérase quantitatives et en temps réel}]
	{rtqpcr}
	{RT-qPCR}
	{real time quantitative polymerase chain reaction}
\newacronym[
	longfr={nouvelle technologie de séquençage},
	longplfr={nouvelles technologies de séquençage}]
	{ngs}
	{NGS}
	{next-generation sequencing}
\newacronym[
	longfr={immunoprécipitation de chromatine}]
	{chip}
	{ChIP}
	{chromatin immunoprecipitation}
\newacronym[
	longfr={séquençage massif des produits d'immunoprécipitation de chromatine}]
	{chipseq}
	{ChIP-Seq}
	{deep sequencing of chromatin immunoprecipitation products}
\newacronym[
	longfr={tag appariés},
	longplfr={tags appariés}]
	{pet}
	{PET}
	{paired-end tag}
\newacronym[
	longfr={analyse des interactions de la chromatine par séquençage de tags appariés}]
	{chiapet}
	{ChIA-PET}
	{chromatin interaction analysis by paired-end tag sequencing}
\newacronym[
	longfr={séquençage des ARN}]
	{rnaseq}
	{RNA-Seq}
	{RNA sequencing}
\newacronym[
	longfr={sequençage de tags appariés d'ARN pleine longueur}]
	{rnapet}
	{RNA-PET}
	{paired-end tags sequencing of full length RNA}
\newacronym[
	longfr={séquençage de tags appariés d'ADN génomique}]
	{gpet}
	{gPET}
	{genomic DNA pair-end tag sequencing}
\newacronym[
	longfr={séquençage par ligation et détection d'oligonucléotides}]
	{solid}
	{SOLiD}
	{sequencing by oligonucleotide ligation and detection}
\newacronym[
	longfr={capture d'ADN méthylé à l'aide d'un domaine MBD et séquençage}]
	{methylcapseq}
	{MethylCAP-Seq}
	{methylated DNA capture by MBD domain and high-throughput sequencing}
\newacronym[
	longfr={immunoprécipitation d'ADN méthylé et séquençage}]
	{medipseq}
	{MeDIP-Seq}
	{methylated DNA immunoprecipitation and high-throughput sequencing}
%
\newacronym
	{swisnf}
	{SWI/SNF}
	{"SWItch/Sucrose NonFermentable"}
\newacronym[
	longfr={complexe répresseur polycomb 2},
	longplfr={complexes répresseurs polycomb 2},
	longplural={polycomb repressive complexes 2}]
	{prc2}
	{PRC-2}
	{polycomb repressive complex 2}
\newacronym[
	longfr={jumonji, domaine d'interaction riche en AT 2}]
	{jarid2}
	{JARID2}
	{jumonji, AT rich interactive domain 2}
\newacronym[
	longfr={enhancer de l'homologue de zeste 2}]
	{ezh2}
	{EZH2}
	{enhancer of zeste homolog 2}
\newacronym[
	longfr={homologue D de l'histone déméthylase 1 contenant un domaine jumonji C}]
	{jhdm1d}
	{JHDM1D}
	{jumonji C domain containing histone demethylase 1 homolog D}
%
\newacronym
	{h3k27}
	{H3K27}
	{lysine 27 de l'histone H3}
\newacronym
	{h3k27me3}
	{H3K27me3}
	{tri-méthylation de la lysine 27 de l'histone H3}
\newacronym
	{h3k4}
	{H3K4}
	{lysine 4 de l'histone H3}
\newacronym
	{h3k4me1}
	{H3K4me1}
	{mono-méthylation de la lysine 4 de l'histone H3}
\newacronym
	{h3k4me2}
	{H3K4me2}
	{di-méthylation de la lysine 4 de l'histone H3}
\newacronym
	{h3k4me3}
	{H3K4me3}
	{tri-méthylation de la lysine 4 de l'histone H3}
%
\newacronym[
	longfr={ontologie de gènes}]
	{go}
	{GO}
	{gene ontology}
\newacronym
	{acp}
	{ACP}
	{analyse en composantes principales}
%
\newacronym
	{aadac}
	{AADAC}
	{arylacétamide déacetylase}
\newacronym[
	longfr={phospholipase à domaine patatine}]
	{pnpla2}
	{PNPLA2}
	{patatin-like phospholipase domain}
\newacronym[
	longfr={composant 4 du complexe de maintenance du minichromosome}]
	{mcm4}
	{MCM4}
	{minichromosome maintenance complex component 4}
\newacronym[
	longfr={composant 2 du complexe de maintenance du minichromosome}]
	{mcm2}
	{MCM2}
	{minichromosome maintenance complex component 2}
\newacronym[
	longfr={protéine à doigt de zinc basonucléine-1}]
	{bnc1}
	{BNC1}
	{zinc finger protein basonuclin-1}
\newacronym[
	longfr={protéine à doigt de zinc 750}]
	{znf750}
	{ZNF750}
	{zinc finger protein 750}
\newacronym
	{krt17}
	{KRT17}
	{keratine 17}
\newacronym
	{lamb3}
	{LAMB3}
	{laminine $\beta$3}
\newacronym
	{scel}
	{SCEL}
	{scieline}
\newacronym
	{lig1}
	{LIG1}
	{ligase 1}
\newacronym[
	longfr={supresseur de la signalisation des cytokines, 2}]
	{socs2}
	{SOCS2}
	{suppressor of cytokine signaling 2}
\newacronym
	{tnc}
	{TNC}
	{tenascine C}
\newacronym[
	longfr={ADN méthyltransférase}]
	{dnmt}
	{DNMT}
	{DNA methyltransferase}
\newacronym[
	longfr={ADN méthyltransférase 3A}]
	{dnmt3a}
	{DNMT3A}
	{DNA methyltransferase 3A}
\newacronym[
	longfr={ADN méthyltransférase 1}]
	{dnmt1}
	{DNMT1}
	{DNA methyltransferase 1}
\newacronym[
	longfr={protéines à domaine de liason aux CpG méthylés}]
	{mbd}
	{MBD}
	{methyl-CpG binding domain protein}
\newacronym[
	longfr={protéine 2 de liaison aux CpG méthylés}]
	{mecp2}
	{MECP2}
	{methyl CpG binding protein 2}
\newacronym[
	longfr={protéine 1 semblable à l'ubiquitine avec des domaines PHD et en doigt ``RING''}]
	{uhrf1}
	{UHRF1}
	{ubiquitin-like with PHD and RING finger domains 1}
\newacronym
	{lepr}
	{LEPR}
	{récepteur à la leptine}
\newacronym[
	longfr={protéine 9 contenant un domaine TUDOR}]
	{tdrd9}
	{TDRD9}
	{tudor domain containing 9}
\newacronym[
	longfr={apolipoprotéine B enzyme d'édition des ARNm, similaire au polypeptide catalytique 2}]
	{apobec2}
	{APOBEC2}
	{apolipoprotein B mRNA editing enzyme, catalytic polypeptide-like 2}
\newacronym
	{sam}
	{SAM}
	{S-adénosyl méthionine}