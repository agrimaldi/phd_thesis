% $Log: abstract.tex,v $
% Revision 1.1  93/05/14  14:56:25  starflt
% Initial revision
% 
% Revision 1.1  90/05/04  10:41:01  lwvanels
% Initial revision
% 
%
%% The text of your abstract and nothing else (other than comments) goes here.
%% It will be single-spaced and the rest of the text that is supposed to go on
%% the abstract page will be generated by the abstractpage environment.  This
%% file should be \input (not \include 'd) from cover.tex.
La métamorphose des amphibiens est le processus rapide et irréversible par lequel un têtard aquatique se transforme en une grenouille respirant à la surface.
Cette transition écologique, réminiscente de la période périnatale chez les mammifères, s'accompagne de changements spectaculaires (régime alimentaire, organes locomoteurs, système respiratoire...).
Ces modifications morphologiques et physiologiques nécessitent la réponse concertée à un signal hormonal, les \glspl{ht}, de différents tissus vers des destin parfois opposés :
apoptose (dans la queue), prolifération (dans les pattes), et remodelage (dans les intestins et le système nerveux central).
\par
Toutefois, la synchronisation de la réponse des différents tissus fait appel à d'autres signaux hormonaux, et notamment les \glspl{gc}.
Ces derniers sont également les médiateurs principaux de la réponse au stress.
Les processus endocriniens de la métamorphose et la réponse au stress sont fortement couplés.
Les \glspl{gc} peuvent ainsi jouer le rôle d'interface permettant l'intégration de signaux environnementaux au niveau de réseaux de régulation.
% Dans ce contexte et au cours du développement, les interactions croisées entre les signalisations \gls{ht} et \gls{gc} pourraient servir de base moléculaire afin que le système biologique puisse intégrer des signaux environnementaux (extrinsèques) et développementaux (intrinsèques), donnant lieu à l'émergence de phénotypes altérés et de réponses potentiellement adaptées.
\par
Dans le cadre de mon doctorat, j'ai analysé les transcriptomes des \glspl{hlb} et l'épiderme caudal de têtards de \gls{xtrop} traités ponctuellement avec des \glspl{ht} et / ou des \glspl{gc}.
La comparaison de ces deux tissus a permis de caractériser la diversité des profils d'expression des gènes cibles des \glspl{ht} et des \glspl{gc}.
Il en ressort plusieurs résultats majeurs.
Tout d'abord, la diversité des profils d'interactions entre ces deux voies est limitée, et la majorité des types de profils sont communs aux deux tissus.
Indépendamment du tissu, certains profils sont caractéristiques de fonctions biologiques spécifiques comme le remodelage de la matrice extracellulaire et le système immunitaire.
Les gènes impliqués dans ces fonctions communes aux deux tissus sont cependant différents.
Enfin, plusieurs facteurs impliqués dans la méthylation de l'ADN sont régulés par les deux hormones.