 % -*- root: ../main.tex -*-
\documentclass[../main.tex]{subfiles}
\begin{document}

\title{Interactions croisées entre hormones thyroïdiennes et glucocorticoïdes durant la métamorphose de \textit{Xenopus~tropicalis}}

\titleen{Transcriptional crosstalk between thyroid hormones and glucocorticoids during \textit{Xenopus tropicalis} metamorphosis}

\author{Alexis Grimaldi}
% If you wish to list your previous degrees on the cover page, use the
% previous degrees command:
%       \prevdegrees{A.A., Harvard University (1985)}
% You can use the \\ command to list multiple previous degrees
%       \prevdegrees{B.S., University of California (1978) \\
%                    S.M., Massachusetts Institute of Technology (1981)}
\department{Université Paris-Sud}

% If the thesis is for two degrees simultaneously, list them both
% separated by \and like this:
% \degree{Doctor of Philosophy \and Master of Science}
\degree{Thèse de Doctorat}

\phdprogram{ED419 BioSigne}

\laboratory{Évolutions des Régulations Endocriniennes}

\discipline{Endocrinologie / Génomique Fonctionnelle}

\degreemonth{April}
\degreeyear{2014}
\thesisdate{16 Mai 2014}
\thesisdateen{May 16\textsuperscript{th} 2014}

% If there is more than one supervisor, use the \supervisor command
% once for each.
\fsupervisor{Laurent M. Sachs}{Directeur de Recherche (CNRS)}
\cosupervisor{Nicolas Buisine}{Maître de Conférences (MNHN)}

% Jury composition
\thesischairman{Anne Mantel}{Professeur (PUPH)}
\firstreporter{Vincent Laudet}{Professeur (ENS)}
\secondreporter{François Tronche}{Directeur de Recherche (CNRS)}
\firstexaminator{Sylvie Dufour}{Directeur de Recherche (CNRS)}
\secondexaminator{Anne Mantel}{Professeur (PUPH)}

% Change background color of the cover
%\pagecolor{BrickRed}

% Make the titlepage based on the above information. If you need
% something special and can't use the standard form, you can specify
% the exact text of the titlepage yourself. Put it in a titlepage
% environment and leave blank lines where you want vertical space.
% The spaces will be adjusted to fill the entire page. The dotted
% lines for the signatures are made with the \signature command.
\maketitle

%\afterpage{\nopagecolor}

% The abstractpage environment sets up everything on the page except
% the text itself.  The title and other header material are put at the
% top of the page, and the supervisors are listed at the bottom.  A
% new page is begun both before and after.  Of course, an abstract may
% be more than one page itself.  If you need more control over the
% format of the page, you can use the abstract environment, which puts
% the word "Abstract" at the beginning and single spaces its text.

%% You can either \input (*not* \include) your abstract file, or you can put
%% the text of the abstract directly between the \begin{abstractpage} and
%% \end{abstractpage} commands.

% First copy: start a new page, and save the page number.
% Uncomment the next line if you do NOT want a page number on your
% abstract and acknowledgments pages.
% \cleardoublepage
%\pagestyle{empty}
%\setcounter{savepage}{\thepage}
\begin{abstractpage}
% $Log: abstract.tex,v $
% Revision 1.1  93/05/14  14:56:25  starflt
% Initial revision
% 
% Revision 1.1  90/05/04  10:41:01  lwvanels
% Initial revision
% 
%
%% The text of your abstract and nothing else (other than comments) goes here.
%% It will be single-spaced and the rest of the text that is supposed to go on
%% the abstract page will be generated by the abstractpage environment.  This
%% file should be \input (not \include 'd) from cover.tex.
La métamorphose des amphibiens est le processus rapide et irréversible par lequel un têtard aquatique se transforme en une grenouille respirant à la surface.
Cette transition écologique, réminiscente de la période périnatale chez les mammifères, s'accompagne de changements spectaculaires (régime alimentaire, organes locomoteurs, système respiratoire...).
Ces modifications morphologiques et physiologiques nécessitent la réponse concertée à un signal hormonal, les \glspl{ht}, de différents tissus vers des destin parfois opposés :
apoptose (dans la queue), prolifération (dans les pattes), et remodelage (dans les intestins et le système nerveux central).
\par
Toutefois, la synchronisation de la réponse des différents tissus fait appel à d'autres signaux hormonaux, et notamment les \glspl{gc}.
Ces derniers sont également les médiateurs principaux de la réponse au stress.
Les processus endocriniens de la métamorphose et la réponse au stress sont fortement couplés.
Les \glspl{gc} peuvent ainsi jouer le rôle d'interface permettant l'intégration de signaux environnementaux au niveau de réseaux de régulation.
% Dans ce contexte et au cours du développement, les interactions croisées entre les signalisations \gls{ht} et \gls{gc} pourraient servir de base moléculaire afin que le système biologique puisse intégrer des signaux environnementaux (extrinsèques) et développementaux (intrinsèques), donnant lieu à l'émergence de phénotypes altérés et de réponses potentiellement adaptées.
\par
Dans le cadre de mon doctorat, j'ai analysé les transcriptomes des bourgeons de membres postérieurs et de l'épiderme caudal de têtards de \gls{xtrop} traités ponctuellement avec des \glspl{ht} et / ou des \glspl{gc}.
La comparaison de ces deux tissus a permis de caractériser la diversité des profils d'expression des gènes cibles des \glspl{ht} et des \glspl{gc}.
Il en ressort plusieurs résultats majeurs.
Tout d'abord, la diversité des profils d'interactions entre ces deux voies est limitée, et la majorité des types de profils sont communs aux deux tissus.
Indépendamment du tissu, certains profils sont caractéristiques de fonctions biologiques spécifiques comme le remodelage de la matrice extracellulaire et le système immunitaire.
Les gènes impliqués dans ces fonctions communes aux deux tissus sont cependant différents.
Enfin, plusieurs facteurs impliqués dans la méthylation de l'ADN sont régulés par les deux hormones.
\end{abstractpage}

% Additional copy: start a new page, and reset the page number.  This way,
% the second copy of the abstract is not counted as separate pages.
% Uncomment the next 6 lines if you need two copies of the abstract
% page.
% \setcounter{page}{\thesavepage}
\clearpage
\begin{abstractpageen}
% $Log: abstract.tex,v $
% Revision 1.1  93/05/14  14:56:25  starflt
% Initial revision
% 
% Revision 1.1  90/05/04  10:41:01  lwvanels
% Initial revision
% 
%
%% The text of your abstract and nothing else (other than comments) goes here.
%% It will be single-spaced and the rest of the text that is supposed to go on
%% the abstract page will be generated by the abstractpage environment.  This
%% file should be \input (not \include 'd) from cover.tex.
Amphibian metamorphosis is the rapid and irreversible process during which an aquatic tadpole transforms into an air breathing adult frog.
This ecological transition, reminiscent of the mammalian perinatal period, comes with spectacular changes (diet, locmotor organs, respiratory system...).
These morphological and physiological modifications necessitate the properly timed response to a single hormonal signal, the thyroid hormones (TH), in various tissues to lead them to sometimes opposite fates:
apoptosis (in the tail), cell prolifération and differenciation (in the limbs) and remodeling (in the intestine and the central nervous system).
\par
However, TH do not act alone.
In particular, glucocorticoids (GC) play important roles during this process.
They also are the main mediator of the stress response.
Endocrine processes of the metamorphosis and the stress response are deeply intertwined.
GC can thus act as an interface to integrate environmental inputs into regulatory networks.
\par
During my doctorate, I analyzed the possible transcriptional crosstalks between TH and GC in two larval tissues: the tailfin (TF) and the hindlimb buds (HLB).
Comparing these two tissues allowed me to caracterize the diversity of TH and GC target gene expression profiles.
This resulted in several major results.
First, the diversity of the profiles of crosstalk between these two pathways is limited, and the majority of the types of profiles is common to both tissues.
Next, independently of the tissues, some profiles are caracteristic of spécific biological functions such as extracellular matrix remodeling and the immune system.
Yet, the genes involved in these shared functions are different between the TF and the HLB.
Finally, several factors involved in DNA methylation are subject to a crosstalk between the two hormones. 
\end{abstractpageen}

\clearpage

\section*{Remerciements}

Je voudrais tout d'abord remercier Laurent Sachs, mon directeur de thèse, pour m'avoir donné la chance de continuer à m’émerveiller devant ce phénomène fascinant qu'est la métamorphose. Ses conseils avisés, son extrême patience, sa disponibilité et son soutient ont été plus que bienvenus.
\par
Merci à Nicolas Buisine pour sa pédagogie, son ouverture d'esprit et les discussions que l'on a eu, parfois loin de mon sujet de thèse, mais toujours enrichissantes.
\par
Indéniablement, Laurent et Nicolas ont participé à forger (à coups de marteau parfois douloureux) le scientifique et l'humain que suis et serais.
\par
Je remercie les membres de mon commité de thèse, Loïc Ponger et Nicolas Pollet qui a également participé à nous alimenter en têtards.
\par
Je tiens à remercier Barbara Demeneix pour m'avoir accueilli dans son laboratoire.
\par
J'exprime ma plus grande gratitude à Vincent Laudet et François Tronche pour avoir accepté d'être les rapporteurs de ma thèse et pour le temps qu'ils ont consacré à l'étude de mon manuscrit.
\par
Je remercie très sincèrement Anne Mantel et Sylvie Dufour d'avoir accepté d’examiner ma thèse.
\par
Merci à Dan Buchholz, avec qui j'ai pu avoir des discussion stimulantes tout au long de son séjour à Paris.
\par
Évidemment, merci à tous les membres du laboratoire.
\par
Je voudrais aussi remercier mes parents pour leur soutient perpétuel, pour m'avoir supporté dans mes passions et pour m'avoir fait découvrir il y a une vingtaine d'années la galerie d'anatomie comparée et la grande galerie de l'évolution.
Ces lieux sont désormais plus que spéciaux pour moi.
\par
Enfin, je ne sais pas comment remercier Aurore, qui, malgré la distance, a su me supporter et m'épauler dans des moments difficiles, et qui est pour moi une source d'inspiration constante.

\clearpage
\end{document}