% $Log: abstract.tex,v $
% Revision 1.1  93/05/14  14:56:25  starflt
% Initial revision
% 
% Revision 1.1  90/05/04  10:41:01  lwvanels
% Initial revision
% 
%
%% The text of your abstract and nothing else (other than comments) goes here.
%% It will be single-spaced and the rest of the text that is supposed to go on
%% the abstract page will be generated by the abstractpage environment.  This
%% file should be \input (not \include 'd) from cover.tex.
Amphibian metamorphosis is the rapid and irreversible process during which an aquatic tadpole transforms into an air breathing adult frog.
This ecological transition, reminiscent of the mammalian perinatal period, comes with spectacular changes (diet, locmotor organs, respiratory system...).
These morphological and physiological modifications necessitate the properly timed response to a single hormonal signal, the thyroid hormones (TH), in various tissues to lead them to sometimes opposite fates:
apoptosis (in the tail), cell prolifération and differenciation (in the limbs) and remodeling (in the intestine and the central nervous system).
\par
However, TH do not act alone.
In particular, glucocorticoids (GC) play important roles during this process.
They also are the main mediator of the stress response.
Endocrine processes of the metamorphosis and the stress response are deeply intertwined.
GC can thus act as an interface to integrate environmental inputs into regulatory networks.
\par
During my doctorate, I analyzed the possible transcriptional crosstalks between TH and GC in two larval tissues: the tailfin (TF) and the hindlimb buds (HLB).
Comparing these two tissues allowed me to caracterize the diversity of TH and GC target gene expression profiles.
This resulted in several major results.
First, the diversity of the profiles of crosstalk between these two pathways is limited, and the majority of the types of profiles is common to both tissues.
Next, independently of the tissues, some profiles are caracteristic of spécific biological functions such as extracellular matrix remodeling and the immune system.
Yet, the genes involved in these shared functions are different between the TF and the HLB.
Finally, several factors involved in DNA methylation are subject to a crosstalk between the two hormones. 