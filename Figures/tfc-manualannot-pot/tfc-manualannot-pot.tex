% BOTTOM caption
% ------------------------
\begin{figure}[!htbp]
\centering
\vspace{1\baselineskip}
\includegraphics[width=0.9\textwidth]
% ------------------------
%
% SIDE caption
% ------------------------
%\begin{SCfigure}[\sidecaptionrelwidth][!htbp]
%\centering
%\vspace{1\baselineskip}
%\includegraphics[width=0.5\textwidth]
% ------------------------
%
% Main information
% ===========================================================
{Figures/tfc-manualannot-pot/tfc-manualannot-pot.png}
\caption[Termes enrichis dans les catégories de "potentiation" dans l'épiderme caudal]
{
Termes enrichis dans les catégories de "potentiation" entre les effets des deux hormones dans l'épiderme caudal.
Les valeurs négatives (partie gauche) correspondent à une potentiation de l'effet répresseur des hormones.
Les valeurs positives (partie droite) correspondent à une potentiation de l'effet inducteur des hormones.
La valeur absolue de l'axe des abscisses représente le nombre de gène associé à chaque terme (axe des ordonnées).
Seuls les 50 termes les plus représenté sont illustrés.
Les barres verticales rouges correspondent au nombre théorique de gènes associés à chaque terme dans le cas d'une répartition aléatoire entre répression et induction.
}
\label{fig:tfc-manualannot-pot}
% ===========================================================
%
% BOTTOM caption
% ------------------------
\end{figure}
% ------------------------
%
% SIDE caption
% ------------------------
%\end{SCfigure}
% ------------------------