% BOTTOM caption
% ------------------------
\begin{figure}[!htb]
\centering
\vspace{1\baselineskip}
\includegraphics[width=\textwidth]
% ------------------------
%
% SIDE caption
% ------------------------
%\begin{SCfigure}[\sidecaptionrelwidth][!htbp]
%\centering
%\vspace{1\baselineskip}
%\includegraphics[width=0.5\textwidth]
% ------------------------
%
% Main information
% ===========================================================
{Figures/thyroid-gland/thyroid-gland.pdf}
\caption[Anatomie et histologie de la glande thyroïde]
{
Anatomie et histologie de la glande thyroïde.
A) La glande thyroïde est située ventralement dans le cou, sous le cartilage thyroïde et autour de la trachée et de l'œsophage.
Elle est composée de deux lobes (droit et gauche) reliés par l'isthme.
Sur la face dorsale de la glande se situent les glandes para-thyroïdes.
B) Trois principales structures histologiques sont impliquées dans la production des \glspl{ht}.
Les follicules (F) sont des structures sphériques entourées d'une couche épithéliale de cellules folliculaire (thyréocytes; T).
Le lumen de ces follicules, composé de colloïde (C), sert de lieu de stockage aux précurseurs nécessaires à la synthèse des \glspl{ht} par les thyréocytes, en particulier de thyroglobuline.
Les cellules para-folliculaires (CPF) sont réparties dans les espaces interstitiels entre les follicules, et secrète la calcitonine.
Images tirées des cours en ligne de l'université du Connecticut et du département de biologie de l'université du Massachusetts Amherst.
}
\label{fig:thyroid-gland}
% ===========================================================
%
% BOTTOM caption
% ------------------------
\end{figure}
% ------------------------
%
% SIDE caption
% ------------------------
%\end{SCfigure}
% ------------------------
%
%
%\missingfigure{Make a figure}