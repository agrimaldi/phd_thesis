% BOTTOM caption
% ------------------------
\begin{figure}[!htb]
\centering
\vspace{1\baselineskip}
\includegraphics[width=\textwidth]
% ------------------------
%
% SIDE caption
% ------------------------
%\begin{SCfigure}[\sidecaptionrelwidth][!htbp]
%\centering
%\vspace{1\baselineskip}
%\includegraphics[width=0.5\textwidth]
% ------------------------
%
% Main information
% ===========================================================
{Figures/metamorphosis/metamorphosis.pdf}
\caption[Changements anatomiques durant la métamorphose du Xénope]
{
Changements anatomiques durant la métamorphose du Xénope.
Figure inspirée de \citet{Shi2001}.
Les modifications morphologiques et biochimiques majeures survenant durant la métamorphose sont présentées table \ref{tab:metamorphosis}.
La longueur des queues entre les \glspl{snf} 63-66 et des pattes entre les \glspl{snf} 52-58 sont à l'échelle.
Les schémas de coupes transversales d'intestins mettent en évidence les différences entre le système digestif larvaire et adulte.
Le premier ne possède qu'une involution, et est riche en tissus conjonctif.
Le second possède un épithélium comportant plusieurs replis, intégré dans une structure complexe présentant tissus conjonctifs entre la couche épithéliale et la couche musculaire à la périphérie.
Les ovales gris foncés et blancs représentent respectivement les cellules épithéliales adultes en prolifération et les cellules épithéliales larvaires en cours d'apoptose.
Les schémas des différents stades sont issus de \citet{Nieukoop1956}.
}
\label{fig:metamorphosis}
% ===========================================================
%
% BOTTOM caption
% ------------------------
\end{figure}
% ------------------------
%
% SIDE caption
% ------------------------
%\end{SCfigure}
% ------------------------
%
%
%\missingfigure{Make a figure}