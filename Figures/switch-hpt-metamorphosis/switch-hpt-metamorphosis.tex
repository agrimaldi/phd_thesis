% BOTTOM caption
% ------------------------
\begin{figure}[!htb]
\centering
\vspace{1\baselineskip}
\includegraphics[width=\textwidth]
% ------------------------
%
% SIDE caption
% ------------------------
%\begin{SCfigure}[\sidecaptionrelwidth][!htbp]
%\centering
%\vspace{1\baselineskip}
%\includegraphics[width=0.5\textwidth]
% ------------------------
%
% Main information
% ===========================================================
{Figures/switch-hpt-metamorphosis/switch-hpt-metamorphosis.pdf}
\caption[Régulations de l'axe \gls{hpt} durant la métamorphose]
{
Régulations de l'axe \gls{hpt} durant la métamorphose.
Chez le têtard, c'est la \gls{crh} qui stimule l'hypophyse antérieure à produire la TSH.
Les \glspl{ht} ainsi synthétisées n'exercent quasiment pas de rétrocontrôle négatif, permettant ainsi à leur concentration d'augmenter au cour de la métamorphose.
Après la métamorphose, les \glspl{ht} exercent le rétrocontrole négatif classique sur la synthèse de \gls{trh} et \gls{tsh} (voir \autoref{par:hpt}).
}
\label{fig:switch-hpt-metamorphosis}
% ===========================================================
%
% BOTTOM caption
% ------------------------
\end{figure}
% ------------------------
%
% SIDE caption
% ------------------------
%\end{SCfigure}
% ------------------------
%
%
%\missingfigure{Switch CRH/TRH during metamorphosis}