% BOTTOM caption
% ------------------------
\begin{figure}[b]
\centering
\vspace{1\baselineskip}
\includegraphics[width=\textwidth]
% ------------------------
%
% SIDE caption
% ------------------------
%\begin{SCfigure}[\sidecaptionrelwidth][!htbp]
%\centering
%\vspace{1\baselineskip}
%\includegraphics[width=0.5\textwidth]
% ------------------------
%
% Main information
% ===========================================================
{Figures/comparison-tfc-hlc-de-genes/comparison-tfc-hlc-de-genes.pdf}
\caption[Heatmap des profils d'expression des gènes différentiellement exprimés dans au moins une condition]
{
Heatmap des profils d'expression des gènes différentiellement exprimés dans au moins une condition (pattes et queue confondues, 5392 gènes).
Chaque ligne correspond à un gène.
Chaque colonne correspond à un traitement dans les deux tissus considérés (\gls{tf}: ``.TFC'' \gls{hlb}: ``.HLC'').
L'intensité des couleurs rouges et bleues est proportionnelle au facteur d'induction de l'expression en $\log_2$ d'un gène donné dans un tissus donné et dans un traitement donné par rapport à la condition ``contrôle'' (CTRL.TFC et CTRL.HLC).
Les facteurs d'induction ont été plafonnés à 4 afin que le contraste ne soit pas capturé par les gènes les plus fortement exprimés.
La similitude des profils d'expression des gènes est exprimée en fonction de leur distance euclidienne.
Cette mesure permet leur regroupement au sein de la figure.
Le niveau d'expression des gènes qui ne sont pas exprimés dans un tissu est fixé à zéro et apparaissent en gris à travers les quatre traitements.
}
\label{fig:comparison-tfc-hlc-de-genes}
% ===========================================================
%
% BOTTOM caption
% ------------------------
\end{figure}
% ------------------------
%
% SIDE caption
% ------------------------
%\end{SCfigure}
% ------------------------