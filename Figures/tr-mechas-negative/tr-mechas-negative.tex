% BOTTOM caption
% ------------------------
\begin{figure}[!htbp]
\centering
\vspace{1\baselineskip}
\includegraphics[width=\textwidth]
% ------------------------
%
% SIDE caption
% ------------------------
%\begin{SCfigure}[\sidecaptionrelwidth][!htbp]
%\centering
%\vspace{1\baselineskip}
%\includegraphics[width=0.5\textwidth]
% ------------------------
%
% Main information
% ===========================================================
{Figures/tr-mechas-negative/tr-mechas-negative.pdf}
\caption[Mécanismes de répression par les récepteurs aux hormones thyroïdiennes activés]
{
Mécanismes de répression par les récepteurs aux hormones thyroïdiennes activés.
a) Recrutement de complexes coactivateurs par \gls{rxr}/\gls{tr} liés à un \gls{ntre} en absence de ligand.
En présence de ligand, des complexes corepresseurs sont recrutés et répriment l'expression du gène.
b) En absence de ligand, l'induction du gène est assurée par un autre facteur de transcription se fixant à l'\gls{dna} au niveau d'un \gls{ntre}.
En présence de ligand, \gls{rxr}/\gls{tr} se fixe au \gls{ntre} et réprime l'expression du gène.
c) Même cas de figure que a), mais \gls{rxr}/\gls{tr} ne se fixe pas directement à l'\gls{dna}, un autre facteur de transcription faisant office d'interface.
d) En abscence de ligand, \gls{rxr}/\gls{tr} non-lié à l'\gls{dna} séquestre des complexes corépresseurs tandis qu'un autre facteur de transcription active l'expression du gène cible.
En présence de ligand, \gls{rxr}/\gls{tr} séquestre les complexes coactivateurs, et le facteur de transcription recrute les complexes corépresseurs précédemment non-disponibles, résultant en la répression du gène cible.
}
\label{fig:tr-mechas-negative}
% ===========================================================
%
% BOTTOM caption
% ------------------------
\end{figure}
% ------------------------
%
% SIDE caption
% ------------------------
%\end{SCfigure}
% ------------------------
%
%
%\missingfigure{Make a figure}