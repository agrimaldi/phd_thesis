% BOTTOM caption
% ------------------------
\begin{figure}[!htbp]
\centering
\vspace{1\baselineskip}
\includegraphics[width=\textwidth]
% ------------------------
%
% SIDE caption
% ------------------------
%\begin{SCfigure}[\sidecaptionrelwidth][!htbp]
%\centering
%\vspace{1\baselineskip}
%\includegraphics[width=0.5\textwidth]
% ------------------------
%
% Main information
% ===========================================================
{Figures/tr-mechas/tr-mechas.pdf}
\caption[Activation de la transcription d'un gène cible des hormones thyroïdiennes]
{
Activation de la transcription d'un gène cible des hormones thyroïdiennes.
En absence d'\glspl{ht} et de récepteur, le gène cible est transcrit de façon basale.
En absence de ligand uniquement, \gls{tr} forme un hétérodimère avec \gls{rxr} et se fixe au niveau d'un \gls{tre} au voisinage du gène cible.
\gls{tr}/\gls{rxr} recrute alors des complexes corépresseurs (complexe CoR) qui empêchent le recrutement de la machinerie transcriptionnelle et désacétylent les queues d'histones, rendant la chromatine non-permissive à la transcription.
En présence de \gls{t3}, celle-ci se lie à \gls{tr} et induit un changement de conformation propice au recrutement de complexes coactivateurs (complexe CoA).
Ceux-ci vont pouvoir agir sur la chromatine, notamment en acétylant les queues d'histones et rendant l'environnement chromatinien permissif à la transcription.
Les CoA sont également capables de recruter des complexes médiateurs qui vont à leur tour recruter la machinerie transcriptionnelle au niveau du \gls{tss} et ainsi induire la transcription du gène cible.
Enfin, \gls{tr} lié à la \gls{t3} peut recruter des complexes de remodelage de la chromatine tels que \gls{swisnf} afin d'exposer des éléments \textit{cis}-régulateurs à d'autres facteurs de transcription, et à favoriser le recrutement de la machinerie transcriptionnelle basale (non-illustré ici).
}
\label{fig:tr-mechas}
% ===========================================================
%
% BOTTOM caption
% ------------------------
\end{figure}
% ------------------------
%
% SIDE caption
% ------------------------
%\end{SCfigure}
% ------------------------
%
%
%\missingfigure{Make a figure}