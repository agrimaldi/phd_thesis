% BOTTOM caption
% ------------------------
\begin{figure}[!htbp]
\centering
\vspace{1\baselineskip}
\includegraphics[width=\textwidth]
% ------------------------
%
% SIDE caption
% ------------------------
%\begin{SCfigure}[\sidecaptionrelwidth][!htbp]
%\centering
%\vspace{1\baselineskip}
%\includegraphics[width=0.5\textwidth]
% ------------------------
%
% Main information
% ===========================================================
{Figures/hpa/hpa.pdf}
\caption[Axe hypothalamo-hypophyso-adrénal]
{
L'axe \gls{hpa} implique l'hypothalamus, l'anté-hypophyse et les glandes corticosurrénales.
L'hypothalamus secrète la \gls{crh}, qui à son tour stimule la production d'\gls{acth} par l'hypophyse.
L'\gls{acth} libérée dans la circulation sanguine va pouvoir agir sur la glande corticosurrénale pour initier la synthèse de corticostéroïdes (quasiment aucun stockage n'a lieu).
Ceux-ci vont enfin agir d'une part sur les tissus cibles pour maintenir la physiologie de base, et d'autre part sur l'hypothalamus pour exercer un rétrocontrôle négatif sur la production de \gls{crh}.
Un stress peut également être à l'origine d'une élévation de la concentration en \glspl{gc} \textit{via} la stimulation de la production de \gls{crh}.
Dans ce cas, l'action des \glspl{gc} sur les tissus cibles va induire un maintien de l'homéostasie en contribuant à moduler la réponse au stress.
}
\label{fig:hpa}
% ===========================================================
%
% BOTTOM caption
% ------------------------
\end{figure}
% ------------------------
%
% SIDE caption
% ------------------------
%\end{SCfigure}
% ------------------------
%
%
%\missingfigure{Make a figure}