% BOTTOM caption
% ------------------------
\begin{figure}[!htbp]
\centering
\vspace{1\baselineskip}
\includegraphics[width=\textwidth]
% ------------------------
%
% SIDE caption
% ------------------------
%\begin{SCfigure}[\sidecaptionrelwidth][!htbp]
%\centering
%\vspace{1\baselineskip}
%\includegraphics[width=0.5\textwidth]
% ------------------------
%
% Main information
% ===========================================================
{Figures/tfc-go-antago-cort/tfc-go-antago-cort.pdf}
\caption[Catégories fonctionnelles enrichies parmi les gènes présentant une inhibition de l'effet \gls{cort} par la \gls{t3}]
{
Catégories fonctionnelles enrichies parmi les gènes présentant une inhibition de l'effet \gls{cort} par la \gls{t3}, correspondants aux clusters 5 et 6 de la \autoref{fig:tfc-clusters-antago}~B.
A) Gènes réprimés par la \gls{cort} (cluster 5, \autoref{fig:tfc-clusters-antago}~B).
B) Gènes induits par la \gls{cort} (cluster 6, \autoref{fig:tfc-clusters-antago}~B).
Les catégories fonctionnelles enrichies ont été obtenues à l'aide de GOrilla \citep{Eden2009}.
Les couleurs des boites correspondent à la significativité de l'enrichissement.
Jaune pâle : $10^{-5} \leq p < 10^{-3}$
Orange : $10^{-7} \leq p < 10^{-5}$
}
\label{fig:tfc-go-antago-cort}
% ===========================================================
%
% BOTTOM caption
% ------------------------
\end{figure}
% ------------------------
%
% SIDE caption
% ------------------------
%\end{SCfigure}
% ------------------------
%
%
%\missingfigure{Make a figure}