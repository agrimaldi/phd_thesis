% BOTTOM caption
% ------------------------
\begin{figure}[!htbp]
\centering
\vspace{1\baselineskip}
\includegraphics[width=\textwidth]
% ------------------------
%
% SIDE caption
% ------------------------
%\begin{SCfigure}[\sidecaptionrelwidth][!htbp]
%\centering
%\vspace{1\baselineskip}
%\includegraphics[width=0.5\textwidth]
% ------------------------
%
% Main information
% ===========================================================
{Figures/gr-mechas/gr-mechas.pdf}
\caption[Mécanismes d'action des récepteurs aux glucocorticoïdes]
{
Mécanismes d'action des \glspl{gr}.
Transactivation classique (a), liaison à un \gls{cgre}/\gls{ap1} près d'un homodimère c-Jun (e), potentialisation de l'activation de la transcription par \gls{stat5} (g).
Transrepression par liaison à des \glspl{ngre} : changement de conformation de \gls{gr} (b), liaison directe à l'ADN d'un seul des deux monomères (c), liaison du \gls{ngre} par un trimère de \gls{gr} (d).
\gls{gr} peut également réprimer l'activation de la transcription médiée par un dimère c-Fos/c-Jun: par liaison à un \gls{cgre} (f), par recrutement de \glspl{grip1} (h) ou après être recruté par \glspl{ntrip6} (i).
Enfin, \gls{gr} peut inhiber l'effet transactivateur de \gls{nfkb} par séquestration de p50/p65 (j) ou en empêchant l'action de \gls{irf3} et \gls{ptefb} (tous deux requis pour l'effet transactivateur de \gls{nfkb}). 
1) \Gls{rnapol2}.
2) \Gls{tbp}.
3) Complexe \gls{cbp}/p300/\gls{hat}.
4) Histone.
5) \Gls{gr} lié ou non à une molécule de \gls{gc} (rond rouge). Les traits rouge et bleu représentent respectivement un \gls{ngre} et un \gls{gre} classique.
6) \Gls{irf3}.
7) \Gls{ptefb}.d
8) Hétérodimère de \gls{stat5} liés à un élément de réponse (traits violets) à \gls{stat5}.
9) Hétérodimère de c-Fos (marron) et c-Jun (violet) lié à un élément de réponse à \gls{ap1} (traits verts).
10) Dimère p50/p65 lié à un élément de réponse à \gls{nfkb}.
11) \Gls{grip1}.
12) \gls{ntrip6}.
13) Cofacteur non-identifié.
}
\label{fig:gr-mechas}
% ===========================================================
%
% BOTTOM caption
% ------------------------
\end{figure}
% ------------------------
%
% SIDE caption
% ------------------------
%\end{SCfigure}
% ------------------------
%
%
%\missingfigure{Make a figure}