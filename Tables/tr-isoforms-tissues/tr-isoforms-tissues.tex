\setlength{\extrarowheight}{5px}

\begin{table}[!htbp]
\centering
\vspace{1\baselineskip}
\footnotesize

\def\tabularxcolumn#1{m{#1}}
\newcolumntype{L}{>{\raggedright\arraybackslash}X}
\newcolumntype{M}{>{\setlength\hsize{3\hsize}\raggedright}X}
\newcolumntype{N}{>{\setlength\hsize{1\hsize}\centering}X@{}}
\newcolumntype{O}{>{\setlength\hsize{1\hsize}\centering}X}
\newcolumntype{P}{>{\centering\setlength\hsize{4\hsize}}X}
\newcolumntype{Q}{>{\centering\setlength\hsize{3\hsize}}X}

\begin{tabularx}{\textwidth}{M O O O O O O O}

\toprule

		& \multicolumn{4}{P}{\gls{tra}}
		& \multicolumn{3}{Q}{\gls{trb}} \tabularnewline

		\cmidrule(rl){2-5}		\cmidrule(rl){6-8}

\textbf{Tissu}
	& \gls{tra}1	& \gls{tra}2	& \gls{trda}1	& \gls{trda}2	& \gls{trb}1	& \gls{trb}2	& \gls{trb}4 \tabularnewline

Rein
	& +	& +	& 	& 	& ++	& 	&  \tabularnewline

Foie
	& -	& -	& 	& 	& ++	& 	&  \tabularnewline

Cerveau
	& ++	& ++	& ++	& 	& ++	& ++	& + \tabularnewline

Coeur
	& +	& +	& 	& 	& ++	& +	& + \tabularnewline

Thyroïde
	& 	& 	& 	& 	& ++	& 	&  \tabularnewline

Muscle squelettique
	& +	& +	& 	& 	& +	& 	& + \tabularnewline

Poumons
	& +	& +	& 	& 	& +	& +	&  \tabularnewline

Rate
	& 	& 	& 	& 	& +	& 	& + \tabularnewline

Testicules
	& -	& -	& 	& 	& 	& 	& ++ \tabularnewline

Rétine
	& 	& 	& 	& 	& 	& ++	&  \tabularnewline

Oreille interne
	& 	& 	& 	& 	& 	& +	&  \tabularnewline

Os (ostéoblastes)
	& ++	& ++	& 	& 	& 	& ++	&  \tabularnewline

Os (chondrocytes hypertrophiés)
	& ++	& ++	& 	& 	& ++	& -	&  \tabularnewline

Neurones hypothalamiques
	& +	& +	& 	& 	& 	& ++	&  \tabularnewline

Anté-hypophyse
	& 	& 	& 	& 	& 	& ++	& ++ \tabularnewline

Épithélium intestinal
	& 	& 	& +	& +	& 	& 	&  \tabularnewline

Précocité durant le développement
	& +	& +	& ++	& +	& 	& 	&  \tabularnewline

\bottomrule

\end{tabularx}
\caption[Tissu-spécificité des différentes isoformes des récepteurs aux hormones thyroïdiennes]
{
Tissu-spécificité des différentes isoformes des \glspl{tr}.
Données intégrant des données humaines et murines.
Tiré de \citet{Cheng2010,Jones2007,Abu2000,Williams2000}.
(-) : Trés faiblement exprimé; (+) : Exprimé; (++) : Fortement exprimé.
}
\label{tab:tr-isoforms-tissues}

\def\tabularxcolumn#1{p{#1}}
\end{table}