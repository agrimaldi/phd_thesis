\setlength{\extrarowheight}{5px}

\begin{table}[!htbp]
\footnotesize

\def\tabularxcolumn#1{m{#1}}
\newcolumntype{L}{>{\setlength\hsize{0.4\hsize}\raggedright}X}
\newcolumntype{M}{>{\setlength\hsize{1\hsize}\raggedright}X}
\newcolumntype{N}{>{\setlength\hsize{1\hsize}\raggedright}X@{\hskip 0.1in}}
\newcolumntype{O}{>{\setlength\hsize{1\hsize}\centering}X}
\newcolumntype{P}{>{\centering\setlength\hsize{2\hsize}}X}

\begin{tabularx}{\textwidth}{L N M}

\toprule

\textbf{Tissu}		& \multicolumn{2}{P}{Réponse} \tabularnewline
					  
					  \cmidrule(rl){2-3}

					& Morphologique		& Biochimique \tabularnewline
					
Cerveau
					& Remodelage, croissance des axones, prolifération et mort cellulaire
					& Division cellulaire, apoptose, synthèse protéique \tabularnewline

Foie
					& Remodelage, différentiation fonctionnelle
					& Induction des enzymes du cycle de l'urée et de l'albumine; Passage de l'hémoglobine larvaire à l'hémoglobine adulte \tabularnewline

Œil
					& Repositionnement; nouvelles connexions et neurones rétinaux; remodelage de la lentille
					& Transformation des pigments visuels (porphyropsine-rhodopsine) \tabularnewline

Peau
					& Remodelage; formation des glandes granulaires; kératinisation; apoptose
					& Induction des collagènes, kératines et magainines adultes; induction de collagènases \tabularnewline

Membres
					& Formation \textit{de novo} d'os, de cartilage; de muscle, de nerfs, d'épithélium
					& Prolifération et différentiation; apoptose; chondrogenèse et ossification \tabularnewline

Poumons
					& Formation \textit{de novo} d'épithélium
					& Prolifération et différentiation \tabularnewline

Queue, Branchies
					& Régression complète
					& Apoptose; destruction de la matrice extracellulaire; induction et activation d'enzymes lytiques (collagénases, metalloprotéinases, nucléases); Prolifération des lysosomes; prolifération des macrophages \tabularnewline

Pancréas, Intestins
					& Profond remodelage
					& Reprogrammation du phénotype; induction de protéases, protéines de liaison aux acides gras et stromelysine-3 \tabularnewline

Système immunitaire
					& Redistribution des types populations cellulaires, changement de soi	
					& Altération du système immunitaire et apparition de nouveaux composants immuno-compétents; Prolifération transitoire dans le contexte des tissus apoptotiques	\tabularnewline

Muscle
					& Changement de motilité, croissance et différentiation; apoptose
					& Transition forme larvaire vers adulte, induction de la chaîne lourde de myosine \tabularnewline

\bottomrule

\end{tabularx}
\caption[Effets morphologiques et biochimiques survenant durant la métamorphose]
{
Effets morphologiques et biochimiques survenant durant la métamorphose.
Tiré de \citet{Tata2006}.
}
\label{tab:metamorphosis}

\def\tabularxcolumn#1{p{#1}}
\end{table}