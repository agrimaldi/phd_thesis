\begin{sidewaystable}[!htbp]
{
\footnotesize
\newcolumntype{L}{>{\raggedright\arraybackslash}X}
\begin{tabularx}{\textwidth}{L L L L}

\toprule

\textbf{Paramètre} &
	\textbf{\gls{dio1}} &
	\textbf{\gls{dio2}} &
	\textbf{\gls{dio3}} \\

\midrule

\textbf{Propriétés biochimiques} & ~ & ~ & ~ \\
	Substrats préférentiels (position) &
		\gls{rt3} (5') &
		\gls{t4} \gls{rt3} &
		\gls{t4} \gls{t3} \\
	Demi-vie &
		Plusieurs heures &
		\~ 20 minutes &
		Plusieurs heures \\
	Localisation sub-cellulaire &
		Membrane plasmique &
		Réticulum endoplasmique &
		Membrane plasmique \\
\textbf{Tissus préférentiels} & ~ & ~ & ~ \\
	~ &
		Foie, rein &
		Système nerveux central, hypophyse, tissu adipeux brun, placenta &
		Placenta, système nerveux central \\
\textbf{Réponse à des taux élevés de \gls{t3} et \gls{t4}} & ~ & ~ & ~ \\
	Transcriptionelle &
		$\uparrow \uparrow$ &
		$\downarrow$ &
		$\uparrow \uparrow$ \\
	Post-traductionelle &
		? &
		$\downarrow \downarrow \downarrow$ (Ubiquitination) &
		? \\
\textbf{Régulations physiologiques} & ~ & ~ & ~ \\
	Induction &
		\gls{t3} &
		Froid, suralimentation, catécholamines, acides biliaires, cAMP &
		\gls{t3}, lésions tissulaires, TGF , FGF, EGF, PDGF, activateurs de ERK \\
	Répréssion &
		Jeûne, maladie &
		\gls{t3}, \gls{t4}, Hedgehog &
		\glspl{gc}, hormones de croissance \\
\textbf{Rôles physiologiques} & ~ & ~ & ~ \\
	~ &
		Clairance de la \gls{rt3} &
		Thermogenèse, développement, fournis la \gls{t3} intracellulaire, source majeure de \gls{t3} plasmatique &
		Développement, clairance de \gls{t3} et \gls{t4}, empêche l'accumulation de \gls{t3} intracellulaire \\
\textbf{Rôles dans des contextes pathologiques} & ~ & ~ & ~ \\
	~ &
		Source principale de \gls{t3} circulante chez les patients hyperthyroïdiens &
		? &
		Hypothyroïdisme "consommant", clairance de la \gls{t3} et la \gls{t4} accrue chez les femmes enceintes \\
		
\bottomrule

\end{tabularx}
}
\caption[Principales caractéristiques des désiodases]
{
Principales caractéristiques des désiodases. Adapté de \citet{Bianco2006}
}
\label{tab:desiodases}
\end{sidewaystable}