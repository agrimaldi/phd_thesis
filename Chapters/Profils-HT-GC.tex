 % -*- root: ../main.tex -*-
\documentclass[../main.tex]{subfiles}
\begin{document}

\chapter{Profils d'expression des gènes régulés par les hormones thyroïdiennes et les glucocorticoïdes}


\section{Une diversité de profils d'expression}

\subsection{Tous les profils d'expression ne sont pas dans la nature}
La réponse de l'expression d'un gènes en fonction d'un stimulus est processus finement régulé.
Compte tenu de la multiplicité des actions des \glspl{ht} et des \glspl{gc}, nous aurions pu nous attendre à une combinatoire diversifiée et étendue.
En ne considérant la modulation de l'expression d'un gène que comme un phénomène trinaire (pas d'effet, répression, induction), le traitement combiné avec les des deux hormones pourrait donner lieu à 27 types de profils différent, et jusqu'à 125 si nous prenons en compte qualitativement les interactions possibles entre les deux hormones.
Pourtant, au vu des données, la diversité d'action des \glspl{ht} et des \glspl{gc} est restreinte.
En effet, en combinant le types de réponse dans les deux tissus, nous n'en comptons qu'à peine plus d'une quinzaine.
Par exemple, des profils correspondant à une répression de la transcription par chaque hormones et a une induction de la transcription lors d'un traitement combiné ne sont pas observés (le cas inverse non plus).
De façon générale, il existe peu de profils pour lesquels le co-traitement induit une régulation de l'expression commune avec au moins un des traitements seuls.
\par
Il est intéressant de noter que les deux tissus partagent essentiellement les mêmes types de profils (bien qu'en des proportions différentes).
La seule exception concerne un type de profil propre au \gls{tf} et correspondant à un antagonisme de la réponse à la \gls{cort} par la \gls{t3}.
Il est cependant important de relativiser ces observations car la tissu-spécificité que nous observons n'est pas liée à des profils spécifiques des tissus mais plutôt au nombre et à la composition en gènes régulés dans chacun des profils commun entre les deux tissus.
Au niveau des proportions, nous avons mis en évidence que les gènes "potentiés" représentent plus de la moitié des gènes différentiellement exprimés dans le \gls{tf}, ce qui contraste fortement avec les effectifs trouvés dans les \glspl{hlb}.
Les gènes présentant un profil d'expression d'"antagonisme" concernent environs le même nombre de gènes dans les deux tissus, mais correspondent à des proportion radicalement différente compte tenu des effectifs différents entre de gènes différentiellement exprimés dans au moins une condition.

\subsection{Un programme commun entre les deux tissus}
Cette section vise à discuter des gènes dont l'expression est affectée de façon similaire dans les deux tissus, et qui pourrait représenter le noyau dur de réseaux de régulation communs.
Ces gènes pourraient ainsi faire partie d'une sorte de "caisse à outil" ubiquitaire répondant de façon stéréotypée aux régulations croisées par les deux hormones.
\par
Toutefois, peu de gènes (81 gènes) sont sous l'effet des régulations croisées par les \glspl{ht} et les \glspl{gc} dans les deux tissus, quelque soit la nature des interactions.
Ces gènes correspondent essentiellement au système immunitaire, à la biologie de la matrice extra-cellulaire, et au développement.

\subsubsection{Des régulations croisées communes aux deux tissus}
Parmi les gènes "potentiés" dans le \gls{tf} et les \glspl{hlb}, 61 sont en commun et participent à des processus immunitaires et l'inflammation.
L'intersection des ensemble de gènes antagonisés dans un tissu mais pas dans le second se limitent à 8 gènes impliqués dans le métabolisme général et incluent \gls{dio3} et MMP13L.
\par
Il n'existe ainsi quasiment pas d'interactions croisées qui s'opposent dans les deux tissus ("potentiation" \textit{vs} "antagonisme").
Malgré la forte hétérogénéité dans les différentes catégories (61 \textit{vs} 8 gènes), ces gènes répondent à un même processus biologique, un remodelage accéléré des tissus.
Les régulations croisées potentialise fortement l'expression de ces gènes, ce qui reflète un rôle "d'accélérateur" de la métamorphose.
A l'inverse, les régulations croisées ne semble interférer qu'avec une minorité d ces gènes (seulement 8).
Bien que cela ne présage pas des effets correspondants à plus long terme, cela suggère tout de même que les interactions croisées perturbent moins les processus développementaux qu'il ne les potentialise.
Cela illustre peut être la robustesse des réseaux de régulation, qui sont connus pour jouer un rôle de tampon et/ou d'intégration de différent signaux.

\subsubsection{Des régulations croisées spécifiques de chaque tissu}
Alors que le changement (ou la conservation) du type d'interaction croisées entre les \glspl{hlb} et le \gls{tf} concerne une minorité de gènes, la majeur différence de régulation croisée entre les deux tissus concerne les gènes co-régulés dans un seul des deux tissus.
On trouve ainsi que 795 gènes sont soumis à des régulations croisées dans le \gls{tf} mais pas dans les \glspl{hlb}.
Ils correspondent essentiellement à des programmes développementaux.
Ceci contraste avec les 236 gènes co-régulés dans les \glspl{hlb} mais pas dans le \gls{tf}.
Ainsi, la régulation croisée de gènes par les deux hormones est majoritairement spécifique des tissus.
Les effets transcriptionnels vont donc potentiellement sous-tendre des effets différents dans les deux tissus.
\\
\par
En conclusion, les interactions croisées affectent beaucoup plus les programmes transcriptionnel spécifiques de chaque tissu que les programmes transcriptionnels communs.
Il est donc difficile d'avoir un pouvoir prédictif sur l'impact des régulations croisées à l'échelle de l'organisme, étant donné la forte tissu-spécificité de la réponse aux intéractions croisées.
Cette quasi absence de réponse "stéréotypée" aux régulations croisées par les \glspl{ht} et les \glspl{gc} à travers l'ensemble des tissus renforce l'importance de ne pas extrapoler les données d'un tissu à un autre.
\par
La majorité des effets communs aux deux tissu concernent le système immunitaire et le remodelage de la matrice extracellulaire.
Cependant, leur "implémentation" en terme de réseaux de gènes impliqués, peuvent avoir une forte composante tissu-spécifique.
\\
\par
Cette diversité de profils suggère l'existence d'un nombre limité de mécanismes moléculaires qui en sont à l'origine.
Il est aussi intéressant de voir que certains gènes changent de catégories en fonction du tissu.
Ceux-ci présentent une certaine complexité dans les mécanismes moléculaires impliqués dans ce type de comportement.
Nous discuterons de deux niveaux où les deux signaux hormonaux peuvent interagir et ainsi affecter les niveaux d'expression de leurs gènes cibles.
\begin{itemize}
\item la compétence des tissus à répondre aux ligands
\item le mécanisme d'action de leur récepteur respectif
\end{itemize}


\subsection{Profils d'expression et compétence pour répondre aux hormones thyroïdiennes et aux glucocorticoïdes}
La première étape pour qu'un signal hormonal agisse sur une cellule est l'entrée de ce signal dans la cellule.
Pour les \glspl{gc}, l'entrée suit un mécanisme passif qui ne représente donc pas un processus pouvant agir sur la concentration en \glspl{gc} dans la cellule.
Nous nous concentrerons sur les HT qui elles ont besoin d'un transporteur.

\subsubsection{Transporteurs cellulaires}
Comme vu dans l'introduction, il existe plusieurs transporteurs.
Dans la condition sans hormone, \gls{oatp}1c1 n'est exprimé que dans les \glspl{hlb}.
Il n'y a pas de transporteur spécifiquement exprimé dans le \gls{tf}.
les \glspl{ht} induisent fortement la transcription de \gls{lat}1 spécifiquement dans les \glspl{hlb}.
Au contraire, l'induction de son expression par la \gls{cort} est restreinte au \gls{tf}.
Il ne semble pas y avoir d'interaction croisée pour ce gène.
Concernant \gls{mct}8 et \gls{mct}10, ils présentent tous deux un profil d'expression correspondant à une potentiation des effets d'une hormone par la seconde (additivité ou synergie).
\par
Les transporteurs des hormones thyroïdiennes sont souvent impliquées aussi bien dans l'entrée et l'efflux des \glspl{ht}.
La potentiation de l'expression \gls{ht}-dépendante (parfois tissu-spécifique) de transporteurs par la \gls{cort} suggère ainsi que les \glspl{gc} vont favoriser le taux de renouvellement des \glspl{ht} au sein de la cellule.

\subsubsection{Métabolisme intracellulaire}
Le second niveau de régulation de la disponibilité en ligand se fait au niveau de la métabolisation de précurseurs en facteurs actifs ou la conversion en des métabolites inactifs.
Dans le cas des \glspl{ht}, cette fonction est assurée par les désiodases
En ce qui concerne les \glspl{gc}, les enzymes participant à la voie de biosynthèse des stéroïdes et les \gls{11bhsd1} et 2 vont pouvoir assurer leur inter-conversion en d'autres métabolites stéroïdiens.

\paragraph{Métabolisme intracellulaire des hormones thyroïdiennes}
Comme mentionné dans l'introduction, les désiodase vont assurer l'activation et l'inactivation des \glspl{ht} en jouant sur la désiodation d'atomes d'iodes a des position spécifiques.
La désiodase principalement activante, qui converti la \gls{t4} en \gls{t3}, est la \gls{dio2}.
La transcription de celle-ci est induite par les \glspl{ht} et les \glspl{gc} dans les deux tissus.
Un co-traitement résulte en une expression plus forte qu'avec chaque hormone seule.
\par
L'inactivation des \glspl{ht} est principalement assurée par la \gls{dio3}.
Dans les deux tissus, elle présente un profil d'expression de type antagonisme, ou les \glspl{gc} semble inhiber en partie l'action des \glspl{ht} sur son expression.
\par
Ceci suggère que les \glspl{gc} contribuent à l'augmentation de la forme majoritairement active \gls{t3} dans les cellules.
Il est cependant à noter, que les traitements hormonaux utilisés ici n'impliquent pas la \gls{t4}.
Les effet transcriptionnels sur la \gls{dio2} sont donc, dans ce modèle, probablement le reflet de boucles de rétro-contrôle, mais ne vont pas participer à la potentiation de l'effet des \glspl{ht}.

\paragraph{Métabolisme intracellulaire des glucocorticoïdes}
Concernant les \glspl{gc}, leur quantité au sein des cellules est principalement régulée par les \gls{11bhsd1} et 2.
Compte tenu que les \glspl{gc} peuvent agir \textit{via} les \glspl{mr}, certains tissus comme les reins présentent une forte expression basale de \gls{11bhsd2} afin de prévenir les effets de la corticostérone.
Dans les \gls{hlb}, son expression est réprimée par la \gls{t3} et un co-traitement (la \gls{cort} seule n'a pas d'effet).
Elle n'est pas exprimée dans le \gls{tf}.
La \gls{11bhsd1} est principalement impliquée dans la conversion en corticostérone.
Son expression n'est affectée par aucun traitement dans les \glspl{hlb}, mais elle est fortement induite dans le \gls{tf} par la \gls{cort} et dans une moindre mesure par la \gls{t3} ($log_2(FC)=2.48$ et $log_2(FC)=0.95$ respectivement).
Le co-traitement induit une expression moindre que lors du traitement à la \gls{cort} seule.
\par
Ces résultats suggèrent des dynamiques de régulation de l'activation des \glspl{gc} différentes selon les tissus :
dans les \glspl{hlb}, contrairement au \gls{tf}, l'inactivation potentielle de la \gls{cort} va être empêchée par la \gls{t3}, ce qui pourrait en partie expliquer l'effet très modeste de la \gls{cort} seule sur ce tissu.
\par
Dans le \gls{tf} mais pas dans les \glspl{hlb}, la \gls{t3} favorise l'expression de la \gls{11bhsd1}.
Contrairement à \gls{dio2} dont l'acitivité activatrice est unidirectionnelle, celle de la \gls{11bhsd1} est bidirectionnelle, et cette dernière peut aussi présenter une activité réductase (inactivante).
Il n'est par conséquent pas à exclure que l'augmentation de son expression dans le \gls{tf} sous l'effet de la \gls{t3} contribue à l'inactivation de la \gls{cort} et à une partie de l'"antagonisme" par les \glspl{ht} des effets médiés par les \glspl{gc}.
En effet, dans le \gls{tf}, un total de 63 gènes présentent un profil ou l'effet de la \gls{cort} (inducteur ou répressif) est inhibé par le traitement à la \gls{t3}.

\subsubsection{Récepteurs}
Enfin, dans le cadre des effets génomiques de ces deux hormones, les actions de ces dernières sur la transcription de gènes cibles nécessite la présence de leur récepteurs nucléaires respectifs.

\paragraph{Récepteurs aux hormones thyroïdiennes}
Les deux récepteurs aux \glspl{ht}, \gls{tra} et \gls{trb} sont exprimés dans les deux tissus étudiés.
Leur expression est fortement induite par la \gls{t3} confirmant les résultats obtenus dans littérature.
La \gls{cort} n'a que peu d'effet sur l'expression de ces deux gènes et n'affecte que marginalement l'expression de \gls{tra} dans les \glspl{hlb}, suggérant que les effets des \glspl{gc} sur la signalisation thyroïdienne ne passe par une régulation de la quantité de récepteurs, du moins dans les deux tissus étudiés ici.

\paragraph{Récepteurs aux corticostéroïdes}
\glspl{gr} et \gls{mr} sont exprimés dans les deux tissus.
Son niveau d'expression de \gls{gr} n'est cependant pas régulé, ni par la \gls{t3} ni par la \gls{cort}.
Au contraire, l'expression de \gls{mr} est fortement induite dans les deux tissus par la \gls{t3} ($log_2(FC)=2.32$ dans les \glspl{hlb} et $log_2(FC)=1.97$ dans le \gls{tf}).
La \gls{cort} n'a cependant pas d'effet sur ses niveaux de transcrits.
\par
Le fait que dans les \glspl{hlb} la \gls{11bhsd2} soit réprimée et que l'expression de \gls{mr} soit induite par la \gls{t3} suggère que dans ce tissu, la corticostérone pourrait médier une partie de ses effets \textit{via} \gls{mr}.
Des analyse complémentaire mériteraient d'être menées afin d'éclaircir ce point.
En particulier, l'utilisation d'analogues des \glspl{gc} ou des \glspl{mc} spécifiques de chaque récepteur et des \glspl{chip} dirigées contre \gls{mr} ou \gls{gr} pourraient renseigner sur la spécificité d'action de la corticostérone, de \gls{gr} et de \gls{mr}.
Des anticorps gracieusement prêtés par R. Denver ne se sont pas révélé suffisament spécifiques pour détecter un enrichissement de signal en \gls{chip}-qPCR conventionnelle.
\\
\par
Pour conclure sur ces aspects, l'effet potentiateur des \gls{gc} sur l'induction de la transcription \gls{ht}-dépendante des transporteurs, ainsi que la répression relative de l'expression de \gls{dio3} suggère une accumulation de d'\glspl{ht} active au sein des cellules cible qui pourrait en partie expliquer les profils de type "potentiation".
Dans les deux tissus les récépteurs spécifiques des différents ligands sont présent, renforçant l'idée que les tissus sont compétent pour répondre aux deux signaux hormonaux.
Enfin, concernant spécifiquement les \gls{gc}, il est interessant de noter que dans les \glspl{hlb}, une partie des effets de la \gls{cort} pourrait être véhiiculée par \gls{mr}.
Cet aspet est d'autant plus intéressant qu'il est connu qu'en liant \gls{gr} ou \gls{mr}, la \gls{cort} ne va pas avoir le même répertoire de gènes cibles.
\par
Un autre niveau où les deux signaux hormonaux peuvent interagir et ainsi affecter les niveaux d'expression de leurs gènes cibles est dans la manière dont les \glspl{tr} et les \glspl{gr} vont moduler l'expression d'un gène cible :
leurs mécanismes d'action moléculaires.


\section{Profils d'expression et mécanismes d'action des récepteurs aux hormones thyroïdiennes et aux corticostéroïdes}

Il est très difficile d'établir un lien entre le mécanisme d'action des \glspl{tr} ou \glspl{gr} et les profils d'expression que nous avons mis en évidence.
En effet, nous n'avons pas les informations permettant de distinguer les gènes cibles directs des gènes cibles indirects parmi les gènes des différents profils d'expression que nous avons observé.
C'est à dire distinguer entre les gènes dont la régulation nécessite l'intervention des \glspl{tr} ou des \glspl{gr} directement sur leur niveau de transcription et les gènes dont la transcription est modulée par un autre facteur de transcription qui lui aura été préalablement sous le contrôle des \glspl{tr} ou des \glspl{gr}.
Cependant, dans les différents profils d'expression que nous observons il est fortement probable qu'un certain nombre de gènes soit directement sous le contrôle des \glspl{tr} et des \glspl{gr}.
En effet, même si leur nombre est faible, les gènes connus pour être régulés directement par les \glspl{tr} (\gls{trb}, \gls{klf9} et \gls{thbzip}) et les \glspl{gr} (\gls{klf9}, \gls{pepck}) le sont aussi dans nos travaux.
De plus, les \glspl{tr} et les \glspl{gr} ont des effets versatiles.
Ils sont capables à la fois d'activer ou de réprimer la transcription de gènes cibles en présence de leur ligand.
Il y a donc des gènes cibles activés et des gènes cibles réprimés par le ligand lié à son récepteur.
En général, le nombre de gènes associé à ce catégories reflète une action majoritairement activatrice des \glspl{ht} (77 \%) et répressive des \glspl{gc} (57 \%) \insertref{ce qui colle avec ...}
\par
Il ne nous est pas possible de corréler ces mécanismes d'action et les profils d'expression que nous observons.
Mais, il ressort que le nombre de profils observés suggère plusieurs mécanismes différents pour expliquer qu'une hormone agisse sur l'expression d'un gène et que la seconde puisse affecter ce mécanisme.
On peut subdiviser les types de profils correspondant à une "potentiation" ou à un "antagonisme" (indépendament du sens des variation de niveau d'expression) sur la base du type de réponses aux hormones.
Dans le cas de potentiation, le gène peut répondre à chaque hormone indépendament, le co-traitement révèlant un effet additif ou synergique.
Le gène peut aussi ne répondre qu'a une seule des deux hormones seules, l'aspect potentiateur du co-traitement correspondant à une synergie d'action.
\par
De la même manière, dans le cas d'antagonismes, deux scénarios apparaissent.
\begin{itemize}
\item dans le premier, l'effet d'une des deux hormones sur la transcription du gène cible est nullifié par la seconde sans que cette dernière n'ait d'effet par elle-même.
\item dans le second scénario, la transcription du gène cible va être modulée par chaque hormone indépendament et dans des sens opposés.
En présence des deux hormones, la variation de transcrit sera le reflet d'une variation intermédiaire entre les effet de chaques hormone.
\end{itemize}

Pour les gènes activés par les ligands, il y a peu de mécanismes décrits dans la littérature :
un mécanisme pour les \glspl{tr} et trois pour les \glspl{gr} (voir respectivement \autoref{fig:tr-mechas} et \autoref{fig:gr-mechas} \textit{a}, \textit{e} et \textit{g}).
Il existe aussi quelques variations autour de ces mécanismes permettant un niveau de finesse supérieure dans la régulation (vitesse de réponse, niveau de réponse).
Pour les \glspl{tr}, il a été observé que ces variations fines font intervenir, en fonction du gène cible, des différences de recrutement des récepteurs dépendant de la structure de la chromatine au niveau du site de fixation à l'\gls{dna} \citep{Bilesimo2011} et des recrutements de co-régulateurs spécifiques \citep{Havis2003}.
Un exemple bien connu de gène activé par les deux ligands est le gène \gls{klf9} qui possède un élément régulateur composite permettant aussi bien la fixation de \gls{tr} que celle de \gls{gr} pour induire la transcription.
\par
Quand le récepteur et son ligand sont des répresseurs de la transcription, il y a bien plus de mécanismes d'action proposés (voir respectivement pour \gls{tr} et \gls{gr}, \autoref{fig:tr-mechas-negative} et \autoref{fig:gr-mechas}) mais pour lesquels le détail est moins précis que pour la première catégorie (particulièrement pour les \glspl{tr}).
\par
Dans le cas des profils d'"antagonisation", on peut spéculer que des régulations opposées par chaque hormone s'annulant dans le cas d'un co-traitement pourrait avoir plusieurs origines :
\begin{itemize}
\item La compétition entre \gls{tr} et \gls{gr} pour un même élément de réponse, ou la régulation mutuellement exclusive de l'expression du gène par chaque récépteur activé.
\item bla
\item Enfin, on ne peut pas exclure la possibilité que des populations de cellules différentes pourraient n'être compétentes qu'à une seule de chaque hormones, la nullification des effets étant due au fait que la quantité de transcrits mesurée est une moyenne de l'état transcriptionnel de populations différentes.
\end{itemize}
De même, on peut attribuer aux \glspl{ht} et aux \glspl{gc} des mécanismes variés pour expliquer les effet de "potentiation" de leur action respective :
\begin{itemize}
\item bla
\item bli
\end{itemize}

\end{document}