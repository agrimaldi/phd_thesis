\chapter{Contexte et Objectifs du Projet}


% =====================================================
% ======= BEGIN - Contexte du projet

\section{Contexte du projet}

% -------------------------------
% +++ BEGIN - La métamorphose: un modèle de développement post-embryonnaire

\subsection{La métamorphose: un modèle de développement post-embryonnaire}
Chez la majorité des métazoaires, le développement embryonnaires et le stade adulte sont séparés par une période larvaire.
La métamorphose est la processus par lequel l'organisme acquiert des caractères adulte, délaissant son style de vie larvaire.
Chez les amniotes, une telle transition écologique ne semble au premier abord pas aussi marquée.
Pourtant, la parturition requiert du fétus qu'il acquiert un certain nombre de fonctions absente durant la gestation.
En particulier, chez l'humain, les dernières semaines de grossesse sont caractérisées par la maturation des poumons et du système digestif.
Ces changements anatomiques s'accompagnent, aussi bien chez l'amphibien en cours de métamorphose que chez le nouveau né, d'événements endocriniens restreints dans le temps, et requis dans le remodelage des tissus.
Notamment, le pic d'\glspl{ht} nécessaire au déclenchement de la métamorphose et observé durant le climax entre les \glspl{snf} 62 et 64, a lieu également peu de temps après la naissance.

% +++ END - La métamorphose: un modèle de développement post-embryonnaire
% -------------------------------
% +++ BEGIN - Un stress à la période périnatale peut perturber la signalisation thyroïdienne

\subsection{Un stress à la période périnatale peut perturber la signalisation thyroïdienne}
En plus de la forte conservation de la fenêtre temporelle d'action des \glspl{ht}, les \glspl{gc} jouent également un rôle important durant la métamorphose comme pendant la période périnatale.
Leurs rôle dans le développement des poumons en font une cible thérapeutique de choix notamment pour déclencher l'accouchement ou afin d'induire la maturation du système respiratoire chez des nouveaux nés prématurés.
L'administration pharmaceutique de corticostéroïdes n'est pas la seule source de \glspl{gc}.
Ces derniers étant en effet les médiateurs principaux de la réponse au stress à moyen ou long terme, les conditions environnementales durant la pro-métamorphose/gestation ou à la métamorphose/naissance peuvent influer sur le taux de \glspl{gc} circulants.
Les interactions croisées entre \gls{ht} et \glspl{gc} font que cette altération des niveaux de \glspl{gc} peuvent influer la signalisation thyroïdienne a un moment de la vie ou son rôle dans le développement post-embryonaire peut avoir des conséquences délétères a plus ou moins long terme.
Cette altération des rhéostats physiologiques autour de la métamorphose/période périnatale peut ainsi être à l'origine d'une certaine plasticité phénotypique, permettant à l'organisme de s'adapter de façon précoce à un environnement éprouvant.
Dans le cas d'une réponse mal adaptée, ceci peut se traduire par des pathologies se déclarant potentiellement à un stade avancé de la vie.

% +++ END - Un stress à la période périnatale peut perturber la signalisation thyroïdienne
% -------------------------------
% +++ BEGIN - Xenopus tropicalis, un modèle pertinent de génomique fonctionnelle

\subsection{\gls{xtrop}, un modèle pertinent de génomique fonctionnelle}\label{subsec:xtrop-model}
Le modèle amphibien a depuis longtemps été un modèle de choix dans l'étude du développement chez les vertébrés, en particulier par l'utilisation du xénope \citep{Harland2011}.
Essentiellement deux espèces du genre \textit{Xenopus}, des amphibiens principalement aquatiques de la famille de \textit{Pipidae}, sont utilisées à des fins de recherche.

La première, \gls{xlaevis}, bénéficie d'une utilisation historique dans les cliniques et hôpitaux ou cette espèce était utilisée comme test de grossesse.
Dans son habitât naturel, géographiquement localisé aux régions sub-sahariennes de l'Afrique, \gls{xlaevis} est un animal charognard et se nourrissant de tout débris organique d'origine animale à sa portée.
Ce comportement alimentaire confère au xénope une adaptabilité remarquable concernant son régime, et en fait une espèce aisée à élever en animalerie.
\gls{xlaevis} à été très utilisé durant le 20\textsuperscript{ème} siècle, notamment en biochimie, physiologie.
De plus, les extraits d'œufs sont un matériel de choix dans le cadre d'expériences \textit{in vitro}.
Toutefois, avec l'essor des \gls{ngs}, \gls{xlaevis} est devenu moins pertinent dans un contexte de génomique fonctionnelle.
En effet, bien que son karyotype révèle la présence de 18 paires de chromosomes et suggère une diploïdie, des études ultérieures ont montré qu'un évènement de duplication du génome a eu lieu avant la divergence de \gls{xlaevis}.
Ceci suggère que \gls{xlaevis} soit une espèce pseudo-tétraploïde, la redondance au moins partielle entre deux sets de chromosomes rendant ce modèle incommode pour l'étude de phénomènes génétiques.

La seconde, \gls{xtrop}, présente la plupart des avantages de \gls{xlaevis}, notamment sa facilité d'élevage et la taille importante de ses embryons.
Toutefois, \gls{xtrop} est la seule espèce par les 20 espèces du genre \textit{Xenopus} à être diploïde, ce qui en fait un modèle particulièrement pertinent en génétique et en génomique fonctionnelle.
L'assemblage de son génome, récemment publié \citep{Hellsten2010a}, souffre malheureusement d'un manque de ressources évident comparé à des organismes modèles plus populaires comme la souris ou la drosophile (pour revue, voir \citet{Grimaldi2013}, \autoref{sec:ctdb-review}).
\gls{xtrop} reste malgré tout un modèle particulièrement pertinent en génomique fonctionnelle et en endocrinologie, en particulier pour l'étude de la métamorphose et des mécanismes moléculaires sous-jacent.

% +++ END - Xenopus tropicalis, un modèle pertinent de génomique fonctionnelle
% -------------------------------

% ======= END - Contexte du projet
% =====================================================

% :::::::::::::::::::::::::::::::::::::::::::::::::::::

% =====================================================
% ======= BEGIN - Objectifs

\section{Objectifs}

% -------------------------------
% +++ BEGIN - Caractérisation des intéractions croisées entre HT et GC

\subsection{Caractérisation des interactions croisées entre hormones thyroïdiennes et glucocorticoïdes}
Des travaux précédents de l'équipe ont montré que des gènes impliqués dans le remodelage de la chromatine et la régulation de la transcription par des mécanismes épigénomiques sont régulés par les \glspl{ht}, dont \gls{ezh2} et \gls{jarid2} (\autoref{fig:ht-chromatin-remodelers}).
Ces deux protéines sont respectivement l'enzyme catalysant la \gls{h3k27me3} et une protéine identifiée comme importante dans le ciblage de \gls{prc2} au niveau de gènes cibles.
Une des hypothèse de départ était que les \glspl{gc} pouvaient affecter la régulation \gls{ht}-dépendante de ces gènes, expliquant ainsi certains des effets à moyen terme d'un stress à la période périnatale.
Bien qu'une partie des effets des \glspl{gc} sur la signalisation thyroïdienne puisse passer par une régulation de la disponibilité en ligand, certains gènes sont régulés de façon directe par les deux hormones (voir \autoref{tissue-compet-ht}).
Nous avons donc entreprit de mesurer à grande échelle les effets des \glspl{ht}, des \glspl{gc}, et des deux hormones ensemble sur la quantité de transcrits individuel.
Cette démarche permet de capturer la réponse de l'organisme à l'échelle du transcriptome, et ainsi mettre en évidence des programmes et des réseaux de régulation importants durant une phase du développement ou ces deux hormones jouent un rôle crucial.

Afin d'isoler le noyau dur des réponses aux \gls{ht} et aux \glspl{gc}, nous avons étudié le transcriptome de deux organes au destin cellulaire différents.
D'une part l'épiderme caudal, qui au cours de la métamorphose se résorbe, d'autre part les membres postérieurs, qui en croissant, font intervenir des processus de prolifération et différentiation cellulaire (\autoref{tab:metamorphosis})

% +++ END - Caractérisation des intéractions croisées entre HT et GC
% -------------------------------
% +++ BEGIN - Cartographie à grande échelle des signatures chromatiniennes associées à la régulation de gènes cibles des HT

\subsection{Cartographie à grande échelle des signatures chromatiniennes associées à la régulation de gènes cibles des hormones thyroïdiennes}
Le deuxième objectif de ce travail repose également sur le fait que les \gls{ht} et potentiellement les \glspl{gc} peuvent influer la machinerie impliquée dans le remodelage de la chromatine.

% BOTTOM caption
% ------------------------
\begin{figure}[!htbp]
\centering
\vspace{1\baselineskip}
\includegraphics[width=\textwidth]
% ------------------------
%
% SIDE caption
% ------------------------
%\begin{SCfigure}[\sidecaptionrelwidth][!htbp]
%\centering
%\vspace{1\baselineskip}
%\includegraphics[width=0.5\textwidth]
% ------------------------
%
% Main information
% ===========================================================
{Figures/ht-chromatin-remodelers/ht-chromatin-remodelers.pdf}
\caption[Relations entre hormones thyroïdiennes et remodeleurs de la chromatine]
{
Relations entre hormones thyroïdiennes et remodeleurs de la chromatine.
Les \glspl{ht} peuvent induire la transcription de gènes codant pour des remodeleurs de la chromatine.
Ces derniers agissent ensuite à trois niveaux :
En participant à l'activation de cibles secondaires, en étant recrutés par les hétérodimères \gls{tr}/\gls{rxr}, enfin en déterminant la liaison à l'ADN de \gls{tr}/\gls{rxr}.
}
\label{fig:ht-chromatin-remodelers}
% ===========================================================
%
% BOTTOM caption
% ------------------------
\end{figure}
% ------------------------
%
% SIDE caption
% ------------------------
%\end{SCfigure}
% ------------------------
%
%
%\missingfigure{Make a figure}

Des travaux précédent de l'équipe ont montré pour l'induction \gls{ht}-dépendante de deux gènes, \gls{trb} et \gls{thbzip}, nécessitait une signature chromatininenne particulière, à savoir la \gls{h3k4me3} \citep{Bilesimo2011}.
Nous avons donc entrepris de cartographier à l'échelle du génome les sites de liaison de \gls{tr}, ainsi qu'un nombre de modifications des histones associées à la répression ou l'activation de la transcription :
\begin{itemize}
\item \gls{h3k27me3}, associée à la répression de la transcription, et déposée par \gls{prc2}.
\item La \gls{h3k4me1}, associée à l'activation de la transcription et enrichie en particulier au niveau de régions cis-régulatrices de type enhancer.
\item La \gls{h3k4me2}, également associée à la transcription.
\item \gls{h3k4me3}, enrichie au niveau des promoteurs de gènes transcrits et concomitante avec \gls{h3k27me3} dans le cas gènes bivalents.
\end{itemize}

Cette étude doit être réalisée dans un contexte le plus physiologique possible.
Les tissus utilisés (membres postérieurs et épiderme caudal) ont donc été récoltés sur des têtards traités ou non aux \glspl{ht}.
En fonction des acteurs du remodelage de la chromatine identifiés dans la première partie du projet, il sera judicieux d'étudier ces même marques dans le cadre d'un traitement indépendant ou combiné aux \glspl{ht} et aux \glspl{gc}. 

% +++ END - Cartographie à grande échelle des signatures chromatiniennes associées à la régulation de gènes cibles des HT
% -----------------------------------
% +++ BEGIN - Amélioration de l'assemblage et l'annotation du génome de xtrop

\subsubsection{Amélioration de l'assemblage et l'annotation du génome de \gls{xtrop}}
Enfin, ces travaux s'inscrivent également dans un projet d'amélioration de l'assemblage et de l'annotation du génome de \gls{xtrop}.

En particulier, l'utilisation du \gls{gpet}\insertref{buisine unpublished} permet de réduire la fragmentation de l'assemblage actuel de 60 \%.
Curieusement, près de la moitié des trous d'assemblage font exactement 50 \gls{pb}, et correspondent à des trous de taille inconnue.
Dans ce contexte, il a été nécessaire de réutiliser les données de \gls{gpet} afin d'évaluer la taille réelle de ces trous d'assemblage.

Dans un second temps, les modèles de gènes ont été reconstruits grâce à l'utilisation combinée de la technologie de \gls{rnaseq} et de \gls{rnapet}.
Une façon de valider les nouveaux modèles de gènes est de caractériser le positionnement de l'\gls{rnapol2} au niveau de la partie 5'.
Les profils d'occupation de \gls{rnapol2} montrent en effet un enrichissement 30 à 50 \gls{pb} en aval du \gls{tss}.

% +++ END - Amélioration de l'assemblage et l'annotation du génome de xtrop
% -------------------------------

% ======= END - Objectifs
% =====================================================