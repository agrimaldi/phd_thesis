 % -*- root: ../main.tex -*-
\documentclass[../main.tex]{subfiles}
\begin{document}

\chapter{Fonction moléculaire des gènes régulés par les hormones thyroïdiennes et les glucocorticoïdes}


\section{Comparaison avec des données similaires}

L'étude des effets des \glspl{ht} et des \glspl{gc} sur le transcriptome dans le contexte de la métamorphose présentés ici ne sont pas les premières publiées.
\citet{Kulkarni2012}\comment{and => \&} ont étudié les effets de ces deux hormones sur le transcriptome de queues de têtards traités ou non à la \gls{t3} et / ou à la \gls{cort}.
Il est à noter que les condition expérimentales employées ne sont pas identique à celles utilisées dans le cadre de mon doctorat.
En particulier, ils mesurent les variations de quantités de transcrits dans la queue entière, et pas seulement l'épiderme.
Ils utilisent en outre des concentration de \gls{t3} 5 fois supérieures à celles utilisées ici (50 nM vs 10 nM).
La concentration en \gls{cort} est cependant la même que nous avons utilisé.
Enfin, la technologie employée est différente (puce à \gls{dna} dans leur cas).
L'ensemble de ces différences rend la comparaison directe de nos résultats avec les leurs très délicate.
\par
Dans cet article, les auteurs trouvent $\sim 4500$ gènes régulés par la \gls{t3} et la \gls{cort}, comparé à $\sim 3200$ après traitement à la \gls{t3} seule et $\sim 1900$ après un traitement aux \glspl{gc} seuls.
Cela contraste fortement avec le nombre de gènes obtenus dans nos résultats (5 à 10 fois moins de gènes selon les conditions).
\par
Les différence entre les deux études peuvent s'expliquer notamment par l'utilisation de plateformes et de démarche de traitement des données différentes :
\begin{itemize}
\item Les puces à \gls{dna} sont très bruitées comparé au \gls{rnaseq} (voir démarche expérimentale).
De plus les corrélations des données de puces et de \gls{rnaseq} sont de l'ordre de 0,7 à 0,8\comment{Info de nicolas. Nos données de puces ?}.
Pour comparaison, la corrélation entre les données que j'ai généré et des données de \gls{rnaseq} produites au laboratoire avec une technologie différente – \gls{solid} – est de l'ordre de 0,85 à 0,95.
Ainsi, les biais propres à chaque technologie rend leur comparaison directe difficile\comment{Est-ce mieux avec les corrections introduites dans les lignes précédentes ?}.
\item Dans l'article, les auteurs ne se sont basés que sur la \textit{p}-value de l'analyse différentielle pour classer les gènes en fonction du type de profils d'expression.
Nous avons vu que cette approche est discutable en raison du faible pouvoir statistique des données à haut débit (puces à \gls{dna} et \gls{ngs} confondues).
Les catégories de gènes définies dans l'article sont donc bruitées et correspondent plus à une description 'technique' des résultats qu'à une description des différents profils d'expression des gènes.
\item L'utilisation de tissus différents.
\citep{Kulkarni2011} utilisent la queue entière alors que nous nous somme focalisés sur l'épiderme caudal.
Nous avons en outre effectué les traitements sur des cultures organotypiques de queues ou de \glspl{hlb} et sur des têtards entiers.
\comment{``On a fait les deux, donc on peut comparer'', mais en fait on ne le fait pas ...}
Ceci nous permet d'une part de prendre en compte des effets centraux possibles, d'autre part, detenir en compte d'éventuels effets dus aux conditions de culture.
\end{itemize}
\comment{Sur cette partie j'ai une sensation de redite par rapport aux stratégies expérimentales}




\section{spécificité fonctionelle de certains profils}

La majorité des effets communs aux deux tissu concernent le système immunitaire et le remodelage de la matrice extracellulaire.
Cependant, leur "implémentation" en terme de réseaux de gènes impliqués, peuvent avoir une forte composante tissu-spécifique.

\subsection{Rôle sur le système immun}
Aussi bien dans le \gls{tf} que dans les \glspl{hlb}, il est frappant de constater qu'une composante bien spécifique de la réponse immunitaire est affectée.
Il est bien connu que les \glspl{gc} ont une forte action anti-inflammatoire, et cet effet se retrouve à travers les gènes affectés par ce traitement hormonal.
C'est le cas notamment de \gls{nfkb}, \gls{ncf1}, \gls{cd40}, ou \gls{irf7}.
L'annotation manuelle et l'annotation automatisée des catégories fonctionnelles convergent toutes deux vers une action additive ou synergique des \glspl{ht} et des \glspl{gc} sur la répression de gènes du système immun.
Il est intéressant de noter que durant la métamorphose, le têtard va subir un remodelage du système immunitaire.
Dans le contexte des travaux présentés ici, il semblerait que la perturbation de la signalisation thyroïdienne en réponse à un stress favorise un processus déjà initié par les \gls{ht} et visant à "éteindre" le système immunitaire innée.
Certaine interleukines comme IL8 ou IL1B sont en effet induites par les \glspl{ht} (uniquement dans les \glspl{hlb}), mais cet effet est antagonisé par les \glspl{gc}.
\par
Dans des modèles de tissu nerveux mammaliens, les \glspl{ht} ont également été rapportée comme pro-inflammatoire \citep{Tamura1999,Montesinos2012} (bien que passant par IL12, un gène n'ayant pas d'orthologue chez le xénope).
Cette action pro-inflammatoire des \glspl{ht} peut également passer par l'intéraction avec STAT4, dont l'expression est réprimée par lors d'un co-traitement, suggérant que les \glspl{gc} pourraient agir à ce niveau.
il est connu que les \glspl{gc} sont aussi capable d'interférer avec le rôle immunostimulant des \glspl{ht}, notament via l'augmentation de cytokine pro-résolutives comme IL10.
Cet effet n'est cependant pas observé dans notre modèle, suggérant que ce mécanisme de balance entre \glspl{ht} et \glspl{gc} n'a pas lieu chez le Xénope ou que les acteurs moléculaires qui le sous-tendent n'ont pas été caractérisés.
Toutefois, parmi les cytokines annotées chez le Xénope, aucune ne présente de profil d'expression correspondant à une induction forte par les \glspl{gc} (IL10 n'est quasiment pas exprimé dans les deux tissus étudiés).
\par
D'autres facteurs impliqués dans la réponse immunitaire innée sont affectés.
En particulier, \gls{cd40}, réprimé par les \glspl{ht} et les \glspl{gc} indépendament, l'est fortement lors d'un co-traitement.
La répression forte de ce récepteur lors d'un co-traitement suggère que l'interaction entre les deux voies de signalisation étudiées contribue à une désensibilisation à ce facteur.
Il est à noter que le ligand correspondant (CD40lg) n'est pas exprimé, mais cela peut provenir d'une mauvaise définition des modèles de gènes dans l'annotation utilisée.
\par
Enfin, on notera que \gls{socs2} est régulé de façon très différente selon les tissus.
Dans les \glspl{hlb}, son expression est induite par les \gls{ht}, alors que dans le \gls{tf}, son expression est réprimée par les \glspl{gc}, et de façon potentialisée lors d'un co-traitement.
Le produit de ce gène est impliqué dans le rétrocontrôle négatif qui régule la transduction du signal des cytokines.
Il semblerait en particulier qu'il soit un régulateur négatif de la voie \gls{igf}1.
Ses profils d'expression dans les deux tissus peut laisser penser que dans les \glspl{hlb}, son induction par les \glspl{ht} va contribuer à un contrôle de l'inflammation.
En revanche dans le \gls{tf}, où son expression est fortement réprimée par les \glspl{ht} et les \glspl{gc}, pourrait contribuer à l'emballement de l'inflammation dans un tissu en cours de résorption.
Cet exemple illustre parfaitement la composante tissu-spécifique de l'action des \glspl{gc} sur les processus d'inflammation.


\subsection{Matrice extracellulaire}
La régression de la queueet la poisse des pattes postérieures sont des processus qui font appel à des mecanismes cellulaire commun.
En particulier, une réorganisation profonde de la structure fine de ces organes à lieu, notamment via le remodelage de la matrice extracellulaire, et l'établissement de nouvelles structures keratineuses.
Ce dernier point est appuyé par le fait que la structure de l'épiderme est un autre aspect à être profondément remodelé au cours de la métamorphose.
De nombreux travaux ont déjà pointé le rôle important des \glspl{ht} dans la régulation de gènes codant pour des métalloprotéinases et des collagénases \insertref{ht et remodelage matrice}.
Les données présentées ici confirment le rôle des \glspl{ht} dans l'induction de la transcription de nombreux membres de ces familles, notamment MMP3, MMP13, KRT17 (impliquée dans la modulation de \gls{tnfa} \textit{via} les voies Akt/mTOR ; son expression est potentiée lors d'un co-traitement dans les \glspl{hlb})
,KRT15 (fortement réprimée par les \glspl{ht})
, KRT79 (fortement exprimée dans les \glspl{hlb} et associée à un profil d'expression de type "potentiation", mais pas exprimée dans le \gls{tf})
, et PAPPA2 (impliquée dans le remodelage de la matrice et régulée de manière opposée par les \glspl{ht} et les \glspl{gc}).
Ces quelques cibles spécifiques d'un tissu ou sous l'effet d'une régulation croisée entre les deux hormones permet déjà de mettre en avant un rôle terminal hautement tissu spécifique des \glspl{ht}.
De plus, comme illustré dans le cas de PAPPA2, la régulation de la balance entre \glspl{ht} et \glspl{gc} pourrait avoir des conséquences locales sur la structure de la matrice.
\par
L'importance de l'intégrité de la matrice extra-cellulaire ne se limite cependant pas uniquement à un rôle structural.
En effet, la matrice va au moins en partie déterminer un certain nombre d'interaction entre cellules, orientant ces dernières vers des phases précises du cycle cellulaire.
Un facteur en particulier, \gls{cebpa} semble impliqué dans le lien entre structure de la matrice et cyclage des celules \citep{Rana1994}.
Les autheurs suggerent dans cet article que la contribution de l'intégrité de la matrice sur l'entrée des cellules en G0 et sur les processus d'arret de croissance et de différentiation en général nécessitaient des niveaux élevés de \gls{cebpa}.
De façon intéressante, le gène codant pour ce facteur est fortement induit dans le \gls{tf} et les \glspl{hlb} sous l'effet de la \gls{t3}.
Additionnellement, dans les \glspl{hlb} mais pas dans le \gls{tf}, l'induction \gls{t3}-dépendante est potentiée par la \gls{cort}.
L'expression accrue de ce facteur lorsque les deux hormones sont présentes dans les \glspl{hlb} pourrait donc contribuer à la différentiation précoce des structures et à la longueur finale réduite des membres postérieurs classiquement observée lors d'une métamorphose accélérée ou induite chez le Xénope et d'autres espèces d'amphibiens \citep{Gomez-Mestre2013a}.
Ceci concorderait également avec les résultats obtenus au niveau de l'ossification prématurée du squelette suite à un traitement combiné avec les deux hormones.

rappel phénotype skel, thyrotoxicose, s'en disteingue par antago.

\subsection{Possibles effets à l'echelle du tissu}
Le travail conduit dans le cadre a été réaliser dans le contexte des effets génomiques de \glspl{ht} et des \glspl{gc}.
Les effets non-génomiques, historiquement associés à des effets à court terme, peuvent cependant avoir des conséquence importante dans la mise en place de structure biologiques.
Il est frappant de constater que la fonction biologique de nombreux gènes co-régulés par les \glspl{ht} et les \glspl{gc} partage beaucoup de points communs avec les effets tissulaires des \glspl{ht} passant par un voie non-génomique.
En particulier, l'action sur le cytosquelette, la motilité cellulaire et des processus developpementaux comme l'angiogenèse laisse penser qu'une partie de ces effets pouraient être véhiculés par l'interaction entre des mécanismes d'action génomiques et non-génomiques.


\end{document}