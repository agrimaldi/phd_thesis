 % -*- root: ../main.tex -*-
\documentclass[../main.tex]{subfiles}
\begin{document}

\begin{chapter}{Fonction moléculaire des gènes régulés par les hormones thyroïdiennes et les glucocorticoïdes}


\begin{section}{Comparaison avec des données similaires}

L'étude des effets des \glspl{ht} et des \glspl{gc} sur le transcriptome dans le contexte de la métamorphose présentés ici ne sont pas les premières publiées.
\citet{Kulkarni2012} ont étudié les effets de ces deux hormones sur le transcriptome de queues de têtards traités ou non à la \gls{t3} et / ou à la \gls{cort}.
Il est à noter que les condition expérimentales employées ne sont pas identique à celles utilisées dans le cadre de mon doctorat.
En particulier, ils mesurent les variations de quantités de transcrits dans la queue entière, et pas seulement l'épiderme.
Ils utilisent en outre des concentration de \gls{t3} 5 fois supérieures à celles utilisées ici (50 nM vs 10 nM).
La concentration en \gls{cort} est cependant la même que celle que nous avons utilisé.
Enfin, la technologie employée est différente (puce à \gls{dna} dans leur cas).
L'ensemble de ces différences rend la comparaison directe de nos résultats avec les leurs très délicate.
\par
Dans cet article, les auteurs trouvent $\sim 4500$ gènes régulés par la \gls{t3} et la \gls{cort}, comparé à $\sim 3200$ après traitement à la \gls{t3} seule et $\sim 1900$ après un traitement aux \glspl{gc} seuls.
Cela contraste fortement avec le nombre de gènes obtenus dans nos résultats (5 à 10 fois moins de gènes selon les conditions).
\par
Les différences entre les deux études peuvent s'expliquer notamment par l'utilisation de plate-formes et de démarche de traitement des données différentes :
\begin{itemize}
\item Les puces à \gls{dna} sont très bruitées comparé au \gls{rnaseq} (voir démarche expérimentale).
De plus les corrélations des données de puces (générées au laboratoire) et de \gls{rnaseq} sont de l'ordre de 0,7 à 0,8.
Pour comparaison, la corrélation entre les données que j'ai généré et des données de \gls{rnaseq} produites au laboratoire avec une technologie différente – \gls{solid} – est de l'ordre de 0,85 à 0,95.
Ainsi, les biais propres à chaque technologie rend leur comparaison directe difficile.
\item Dans l'article, les auteurs ne se sont basés que sur la \textit{p}-value de l'analyse différentielle pour classer les gènes en fonction du type de profils d'expression.
Nous avons vu que cette approche est discutable en raison du faible pouvoir statistique des données à haut débit (puces à \gls{dna} et \gls{ngs} confondues).
Les catégories de gènes définies dans l'article sont donc bruitées et correspondent plus à une description 'technique' des résultats qu'à une description des différents profils d'expression des gènes.
\item L'utilisation de tissus différents.
\citep{Kulkarni2011} utilisent la queue entière alors que nous nous somme focalisés sur l'épiderme caudal.
Nous avons en outre effectué les traitements sur des cultures organotypiques de queues ou de \glspl{hlb} et sur des têtards entiers.
Ceci nous permet d'une part de prendre en compte des effets centraux possibles, d'autre part, détenir en compte d'éventuels effets dus aux conditions de culture.
\end{itemize}

\end{section}


\begin{section}{spécificité fonctionnelle de certains profils}

La majorité des effets communs aux deux tissu concernent le système immunitaire et le remodelage de la matrice extracellulaire.
Cependant, leur "implémentation" en terme de réseaux de gènes impliqués, peuvent avoir une forte composante tissu-spécifique.

\begin{subsection}{Rôle sur le système immun}
Aussi bien dans le \gls{tf} que dans les \glspl{hlb}, il est frappant de constater qu'une composante bien spécifique de la réponse immunitaire est affectée.
Il est bien connu que les \glspl{gc} ont une forte action anti-inflammatoire, et cet effet se retrouve à travers les gènes affectés par ce traitement hormonal.
C'est le cas notamment de \gls{nfkb}, \gls{ncf1}, \gls{cd40}, ou \gls{irf7}.
L'annotation manuelle et l'annotation automatisée des catégories fonctionnelles convergent toutes deux vers une action additive ou synergique des \glspl{ht} et des \glspl{gc} sur la répression de gènes du système immun.
Il est intéressant de noter que durant la métamorphose, le têtard va subir un remodelage du système immunitaire \citep{DuPasquier1989,Flajnik1987}, et que les \glspl{gc} y jouent un rôle important \citep{Rollins-Smith1997}.
Dans le contexte des travaux présentés ici, il semblerait que la perturbation de la signalisation thyroïdienne en réponse à un stress ait deux effets complémentaires sur le système immun :
non seulement cette interaction favorise un processus déjà initié par les \gls{ht} visant à "éteindre" le système immunitaire inné, mais elle peut aller à l'encontre de l'effet des \gls{ht} en inhibant leur action pro-inflammatoire.
Les profils d'expression de deux interleukine impliquées dans l'inflammation – IL8, médiateur important de l'inflammation et IL1B, précurseur de CASP1 – illustrent ce dernier point.
Elles sont en effet induites par les \glspl{ht} (uniquement dans les \glspl{hlb}), mais cet effet est antagonisé par les \glspl{gc}, qui réprime leur expression aussi bien lors d'un traitement à la \gls{cort} seule (effet déjà décrit, \citealp{Nissen2000}) que combiné à la \gls{t3}.
\par
Dans des modèles de tissu nerveux mammaliens, les \glspl{ht} ont également été rapportée comme pro-inflammatoire \citep{Tamura1999,Montesinos2012} (bien que passant par IL12, un gène n'ayant pas d'orthologue chez le Xénope).
Cette action pro-inflammatoire des \glspl{ht} peut également passer par l’interaction avec \gls{stat5} \citep{Favre-Young2000}, dont l'expression est réprimée lors d'un co-traitement, suggérant que les \glspl{gc} pourraient agir à ce niveau.
Il est connu que les \glspl{gc} sont aussi capable d'interférer avec le rôle immunostimulant des \glspl{ht}, notamment via l'augmentation de cytokine pro-résolutives comme IL10 \citep{Montesinos2012}.
Cet effet \textit{via} IL10 n'est cependant pas observé dans notre modèle (IL10 n'étant pas exprimé, quelle que soient les conditions), suggérant que ce mécanisme n'a pas lieu chez le Xénope ou que les acteurs moléculaires qui le sous-tendent n'ont pas été caractérisés.
En revanche l'expression de son récepteur IL10RA est fortement réprimée par les \glspl{gc} et d'autant plus dans le cas d'un co-traitement avec les \glspl{ht}, suggérant que si d'autres interleukines ont la capacité de se fixer sur IL10RA, leurs effets biologiques seront au moins partiellement inhibés.
Il est de plus intéressant de remarquer que IL17 (une cytokine pro-inflammatoire impliqué dans de nombreuses maladies autoimmunes dont l'arthrose rhumatoïde, \citealp{Onishi2010}) et ses recepteurs (IL17RA et IL17RE) sont réprimées par les \glspl{gc} et d'autant plus lors d'un co-traitement avec des \glspl{ht}.
Les \glspl{ht} seules ne répriment, dans une moindre mesure, l'expression de IL17RE que dans les \glspl{hlb}.
Ceci suggère que les \glspl{gc} sont capable d'amplifier un effet anti-inflammatoire tissu-spécifique des \glspl{ht}.
\par
D'autres facteurs impliqués dans la réponse immunitaire innée sont affectés.
En particulier, \gls{cd40}, un facteur ayant un rôle de ``pont'' fonctionnel entre la réponse immune innée et acquise \citep{Elgueta2009,Fujii2004} réprimé par les \glspl{ht} et les \glspl{gc} indépendamment, l'est fortement lors d'un co-traitement.
La répression forte de ce récepteur lors d'un co-traitement suggère que l'interaction entre les deux voies de signalisation étudiées contribue à une désensibilisation à ce facteur.
Il est à noter que la façon dont ont été traitées les données de \gls{rnaseq} ne permettent pas de mesurer les niveaux d'expression du ligand de \gls{cd40} (CD40lg) car le gène chevauche entièrement ARHGEF6.
\par
Enfin, on notera que \gls{socs2} est régulé de façon très différente selon les tissus.
Dans les \glspl{hlb}, son expression est induite par les \gls{ht}, alors que dans le \gls{tf}, son expression est réprimée par les \glspl{gc}, et de façon potentialisée lors d'un co-traitement.
Le produit de ce gène est impliqué dans le rétrocontrôle négatif qui régule la transduction du signal des cytokines.
Il semblerait en particulier qu'il soit un régulateur négatif de la voie \gls{igf}1.
Ses profils d'expression dans les deux tissus peut laisser penser que dans les \glspl{hlb}, son induction par les \glspl{ht} va contribuer à un contrôle de l'inflammation.
En revanche dans le \gls{tf}, où son expression est fortement réprimée par les \glspl{ht} et les \glspl{gc}, pourrait contribuer à l'emballement de l'inflammation dans un tissu en cours de résorption.
Cet exemple illustre parfaitement la composante tissu-spécifique de l'action des \glspl{gc} sur les processus d'inflammation.
\end{subsection}


\begin{subsection}{Matrice extracellulaire}
La régression de la queue et la pousse des pattes postérieures sont des processus qui font appel à des mécanismes cellulaire commun.
En particulier, une réorganisation profonde de la structure fine de ces organes à lieu, notamment via le remodelage de la matrice extracellulaire, et l'établissement de nouvelles structures kératineuses \citep{Schreiber2003}.
Ce dernier point est appuyé par le fait que la structure de l'épiderme est un autre aspect à être profondément remodelé au cours de la métamorphose.
De nombreux travaux ont déjà pointé le rôle important des \glspl{ht} dans la régulation de gènes codant pour des métalloprotéinases et des collagénases \citep{Jung2004,Fu2007}.
Les données présentées ici confirment le rôle des \glspl{ht} dans l'induction de la transcription de nombreux membres de ces familles, notamment MMP3, MMP13, KRT17 (impliquée dans la modulation de \gls{tnfa} \textit{via} les voies Akt/mTOR ; son expression est potentiée lors d'un co-traitement dans les \glspl{hlb}), KRT15 (fortement réprimée par les \glspl{ht}), KRT79 (fortement exprimée dans les \glspl{hlb} et associée à un profil d'expression de type "potentiation", mais pas exprimée dans le \gls{tf}) et PAPPA2 (impliquée dans le remodelage de la matrice et régulée de manière opposée par les \glspl{ht} et les \glspl{gc}).
Ces quelques cibles spécifiques d'un tissu ou sous l'effet d'une régulation croisée entre les deux hormones permet déjà de mettre en avant un rôle terminal hautement tissu spécifique des \glspl{ht}.
De plus, comme illustré dans le cas de PAPPA2, la régulation de la balance entre \glspl{ht} et \glspl{gc} pourrait avoir des conséquences locales sur la structure de la matrice.
\par
L'importance de l'intégrité de la matrice extra-cellulaire ne se limite cependant pas uniquement à un rôle structural.
En effet, la matrice va au moins en partie déterminer un certain nombre d'interaction entre cellules, orientant ces dernières vers des phases précises du cycle cellulaire.
Un facteur en particulier, \gls{cebpa} semble impliqué dans le lien entre structure de la matrice et cyclage des cellules \citep{Rana1994}.
Les auteurs suggèrent dans cet article que la contribution de l'intégrité de la matrice sur l'entrée des cellules en G0 et sur les processus d’arrêt de croissance et de différentiation en général nécessitaient des niveaux élevés de \gls{cebpa}.
De façon intéressante, le gène codant pour ce facteur est fortement induit dans le \gls{tf} et les \glspl{hlb} sous l'effet de la \gls{t3}.
Additionnellement, dans les \glspl{hlb} mais pas dans le \gls{tf}, l'induction \gls{t3}-dépendante est potentiée par la \gls{cort}.
Enfin, il a été montré que l'action anti-proliférative de \gls{cebpa} se faisait en partie par l'inhibition directe de CDK2 \citep{Wang2001} dont l'expression est réprimée de façon ``potentiée'' dans le \gls{tf}.
L'expression accrue de \gls{cebpa} lorsque les deux hormones sont présentes dans les \glspl{hlb} pourrait donc contribuer à la différentiation précoce des structures et à la longueur finale réduite des membres postérieurs classiquement observée lors d'une métamorphose accélérée ou induite chez le Xénope \citep{Gomez-Mestre2013a} et d'autres anoures.
Ceci concorderait également avec les résultats obtenus au niveau de l'ossification prématurée du squelette suite à un traitement combiné avec les deux hormones.
Dans le \gls{tf}, ce profil d'expression de \gls{cebpa} associé à la répression de facteurs pro-proliférateurs tels que CDK2 pourrait être un mécanisme général d’arrêt de la prolifération dans un tissu en cours de résorption
% \par
% Ce facteur n'est cependant pas le seul à intervenir à ce niveau.
% En effet, la curation manuelle des fonctions des gènes a permis de faire ressortir plusieurs facteurs impliqués dans la communication de cellule à cellule et les interactions cellule-matrice.
% En particulier
\end{subsection}


\begin{subsection}{Possibles effets à l’échelle du tissu}
Par leurs actions croisées sur la matrice extra-cellulaire et le système immunitaire, les \glspl{ht} et les \glspl{gc} peuvent moduler des phénomènes biologiques plus généraux.
Par exemple, le remodelage de la matrice et les communications de cellule à cellules vont être importantes dans des processus développementaux tels que l'angiogenèse.
L'influence de ces deux hormones sur le système immunitaire va quant à elle potentiellement affecter des tissus particulièrement sensibles à des pathologies auto-immunes comme les intestins ou les articulations (arthrose rhumatoïde ou ostéoarthrite).
Ces deux processus ne sont cependant pas disjoints, puisque de nombreuses maladies auto-immunes vont se traduire par une inflammation locale accrue et une dégradation de la matrice extra-cellulaire.
Ces effets peuvent s'expliquer en partie par les mécanismes d'action génomiques de ces deux hormones.
De plus, la métamorphose a jusqu'à présent été décrite comme un phénomène impliquant les voies génomiques des \glspl{ht} sur la base de l'observation que \gls{tr} était indispensable \citep{Das2010}.
Alors que les effet génomiques sont indéniablement nécessaires aux effets majeurs des \glspl{ht}, il est envisageable que leurs effets par des voies non-génomiques soient des facettes importante de la régulation fine des phénomènes biologiques qu'elles régulent. 
Les effets non-génomiques, historiquement associés à des effets à court terme et indépendants des \glspl{tr}, ont été montrés comme associés à des processus développementaux longs, impliquant potentiellement certaines isoformes de \gls{tr} ou certaines modifications post-traductionnelles de \gls{tr} \citep{Davis2005}, et se traduisant – bien qu'indirectement – par une réponse au niveau du transcriptome \citep{Davis2011}.
\par
Dans ce contexte, il est important de mentionner que deux gènes, MFGE8 et \gls{tnc}, dont l'effet des \glspl{ht} est respectivement antagonisé et potentié par les \glspl{gc}, sont impliqués dans la signalisation du récepteur membranaire \gls{iab3} \citep{Sriramarao1993,Aziz2009}.
Il se pourrait donc que des interactions croisées entre voies génomiques et non-génomiques complexifient le paysage des fonctions biologiques régulées par les \glspl{ht} et les régulations croisées entre ces dernières et les \glspl{gc}.
\end{subsection}

\end{section}

\end{chapter}

\end{document}