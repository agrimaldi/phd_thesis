 % -*- root: ../main.tex -*-
\documentclass[../main.tex]{subfiles}
\begin{document}

\begin{chapter}{Conclusions et Perspectives}


\begin{section}{Conclusions}

L'ensemble de ce travail a permis la caractérisation détaillée des effets transcriptomiques des régulations croisées entre les \glspl{ht} et les \glspl{gc}.
Il en ressort trois points majeurs.
\par
Tout d'abord, il apparaît que les interactions entre les deux hormones résultent en une diversité limitée des profils d'expression des gènes régulés.
Deux catégories principales émergent, indépendamment du sens de la régulation et du tissu étudié :
Certains gènes voient leur expression potentiée lors d'un co-traitement, tandis que d'autres sont associés à un effet antagoniste d'une hormone sur l'autre.
\par
Ensuite, ces types d'interactions sont caractérisés par des fonctions biologiques communes au \gls{tf} et aux \glspl{hlb}, incluant une modulation de la biologie des tissus conjonctifs et du système immunitaire.
En revanche, de façon surprenante, en particulier concernant le système immunitaire, les réponses biologiques présentent, face à ces deux signaux hormonaux, une grande spécificité tissulaire.
\par
Enfin, les \gls{gc} semblent interférer avec des processus de méthylation de l'\gls{dna}, potentiellement en affectant son maintient et sa lecture, et en provoquant une redistribution de cette marque.
Les conséquences pourraient être une variabilité accrue de la réponse, contribuant ainsi pour l'individu à la flexibilité phénotypique et à l'échelle d'une population, à la diversité de phénotypes.

\end{section}


\begin{section}{Perspectives}

L'utilisation du \gls{rnaseq} s'est révélée dans le cadre de ce projet, particulièrement enrichissante pour aborder l'analyse de programmes de régulation croisées entre les \glspl{ht} et les \glspl{gc}.
Toutefois, il est difficile de tirer des conclusions définitives sur la biologie fine qui a lieu.
Les données sur le \gls{tf} restent pour le moment très descriptives.
Pour aller plus loin, il serait nécessaire de compléter ces résultats par des analyses fonctionnelles.
De plus, dans le cadre d'une collaboration avec D. Buchholz (Université de Cincinnati, USA), nous avons réalisé la mesure de les variations de quantité de transcrits par \gls{rnaseq} dans le foie de têtards pré-métamorphiques traités aux \glspl{ht} et aux \glspl{gc}.
Il sera notamment intéressant de comparer la signature transcriptionnelle des intéractions entre \glspl{ht} et \glspl{gc} et s'il existe aussi une réponse immune spécifique de ce tissu.
Nous pouvons nous attendre à ce que le métabolisme soit affecté par ces traitements ; ce travail permettrait donc d'en caractériser les bases moléculaires.
\par
En outre, bien que mon travail mette en valeur les modalités d'interactions entre ces deux hormones et leurs effets sur le transcriptome, tout reste à faire en ce qui concerne l'élucidation des détails mécanistiques qui sous-tendent ces interactions croisées.
La tâche est d'autant plus difficile qu'à la vue de nos résultats, il semblerait qu'une grande diversité de mécanismes soient impliqués, de la compétition pour des co-régulateurs à une action synergique en \textit{cis} en passant par des voies non-génomiques.
\par
Une voie d'approche possible pour caractériser ces mécanismes est de décrire la dynamique des intermédiaires moléculaires impliqués dans la régulation de la transcription, comme par exemple les modifications post-traductionnelles des histones.
Les marques \gls{h3k4me1} et \gls{h3k4me3} sont associées à l'activation d'éléments \textit{cis}-régulateurs au niveau de ``enhanceurs'' et de promoteurs, dont l'état de méthylation peut influer sur la transcription des gènes sous leur contrôle.
J'ai donc initié leur cartographie par \gls{chipseq}.
Sont présentés \autoref{fig:chipseq-histones} des données préliminaires incluant les profils de densité de la mono- et tri-méthylation d'\gls{h3k4}, de H3, et de \gls{tr} dans les \glspl{hlb}.
Ces données apporteront également des informations supplémentaires sur la concomitance de la méthylation d'\gls{h3k4} et la liaison de \gls{tr}, ce qui avait déjà été observé au niveau de deux gènes \citep{Bilesimo2011}.

% BOTTOM caption
% ------------------------
\begin{figure}[tbp]
\centering
\vspace{1\baselineskip}
\includegraphics[width=\textwidth]
% ------------------------
%
% SIDE caption
% ------------------------
%\begin{SCfigure}[\sidecaptionrelwidth][!htbp]
%\centering
%\vspace{1\baselineskip}
%\includegraphics[width=0.5\textwidth]
% ------------------------
%
% Main information
% ===========================================================
{Figures/chipseq-histones/chipseq-histones.pdf}
\caption[Profils de densité de ChIP-Seq pour la méthylation de H3K4 et  TR]
{
Profils de densité de ChIP-Seq pour la mono- et la tri-méthylation d'\gls{h3k4} au niveau du locus de \gls{trb}.
Deux sites de liaison de \gls{tr} concordent spatialement avec un enrichissement de \gls{h3k4me1} et \gls{h3k4me3} (flèches vertes).
À noter que trois pics correspondent à du bruit de fond et sont retrouvés également dans l'``input'' (flèches rouges).
}
\label{fig:chipseq-histones}
% ===========================================================
%
% BOTTOM caption
% ------------------------
\end{figure}
% ------------------------
%
% SIDE caption
% ------------------------
%\end{SCfigure}
% ------------------------

Parmi la multitude de résultats que j'ai obtenus, il apparaît une composante forte associée à la modulation de la méthylation de l'\gls{dna}.
Il reste à ce sujet à déterminer dans quelle mesures d'éventuelles variations des niveaux de méthylation vont affecter certains gènes en particulier ou le génome dans son ensemble, et quelles en sont les conséquences fonctionnelles.
Alors que des résultats préliminaires suggèrent que le niveau de méthylation global ne varie pas au cours de la métamorphose, il est prévu de cartographier à l'échelle du génome par gls{methylcapseq} et \gls{medipseq} les loci méthylés dans les mêmes conditions expérimentales que celles utilisées dans le cadre de ma thèse.
Ceci nous permettra de mettre en relation directe les variations d'expression des facteurs impliqués dans la méthylation de l'\gls{dna} et le dépôt effectif de cette marque épigénétique suite aux différents traitements hormonaux.
La raison pour laquelle ces travaux n'ont pas été initiés plus tôt se trouve dans la difficulté de distinguer entre méthylation et hydroxyméthylation.
La technologie \gls{medipseq} permet l'utilisation d'anticorps spécifiques de l'hydroxyméthylation.


\end{section}

\end{chapter}

\end{document}