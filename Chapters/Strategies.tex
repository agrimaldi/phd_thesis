 % -*- root: ../main.tex -*-
\documentclass[../main.tex]{subfiles}
\begin{document}


\chapter{Stratégies et Démarches Expérimentales}

Les tissus ont été traités dans différentes conditions et les \glspl{rna} extraits (\autoref{fig:thesis-technical-workflow}~vert).
La mesure du niveau d'expression des transcrits a ensuite été fortement parallélisée, à l'aide des techniques modernes de séquençage massif (\autoref{fig:thesis-technical-workflow}~cyan).
Les séquences brutes ont été nettoyées et ont subi tout un ensemble de contrôles qualité, avant d'être associées à leur gène cible (\autoref{fig:thesis-technical-workflow}~orange).
Tout au long de cette trame, des contrôles qualités sont menés pour valider la qualité des échantillons biologiques et de la mesure effectuée par séquençage (\autoref{fig:thesis-technical-workflow}~violet).
Les mesures digitales d'expression des gènes ainsi obtenues ont permis de déterminer les gènes différentiellement exprimés, de dériver des profils d'expression ("clustering") et d'isoler les gènes soumis à une régulation spécifique des co-traitement \gls{ht} - \gls{gc}.
La fouille ultérieure des données (étape communément appelée 'intégration des données') a permis de traduire les listes de gènes différentiellement exprimés en processus biologiques affectés par les interactions croisées en \glspl{ht} et \glspl{gc}. 

%Les échantillons biologiques (\gls{rna}) sont préparé à partir de cultures organotypiques de queues ou de pattes, ou de têtards entiers traité aux \glspl{ht} et / ou aux \glspl{gc} (\autoref{fig:thesis-technical-workflow}~vert).
%Pour chaque tissu et traitement, des banques sont préparées permettant le séquençage des \glspl{rna} (\autoref{fig:thesis-technical-workflow}~cyan).
%Avant de pouvoir analyser les données, des étapes de traitement bioinformatique sont nécessaire (\autoref{fig:thesis-technical-workflow}~orange).
%Tout au long de cette trame, des contrôles qualités sont menés pour valider la qualité des echantillons biologiques et de la mesure effectuée par séquençage (\autoref{fig:thesis-technical-workflow}~violet).
%Le détail est présenté dans les sections suivantes.

La trame générale des principales étapes méritant une attention particulière est présentée \autoref{fig:thesis-technical-workflow}.

% BOTTOM caption
% ------------------------
\begin{figure}[!htbp]
\centering
\vspace{1\baselineskip}
\includegraphics[width=\textwidth]
% ------------------------
%
% SIDE caption
% ------------------------
%\begin{SCfigure}[\sidecaptionrelwidth][!htbp]
%\centering
%\vspace{1\baselineskip}
%\includegraphics[width=0.5\textwidth]
% ------------------------
%
% Main information
% ===========================================================
{Figures/thesis-technical-workflow/thesis-technical-workflow.pdf}
\caption[Aspects techniques de la démarche expérimentale]
{
Diagramme de la démarche experimentale.
Vert: Préparation et collection des échantillons biologiques.
Cyan: Préparation des banques et séquençage.
Orange: Traitement des données.
Violets: Contrôles et vérifications.
}
\label{fig:thesis-technical-workflow}
% ===========================================================
%
% BOTTOM caption
% ------------------------
\end{figure}
% ------------------------
%
% SIDE caption
% ------------------------
%\end{SCfigure}
% ------------------------


% ======================================================
% ======= BEGIN - Collection des échantillons biologiques

\section{Collection des échantillons biologiques}\label{sec:col-bio-samples}

\subsection{Choix des tissus}

Nous avons choisi deux tissus au devenir cellulaire différent: le \gls{tf}, qui est résorbé à la fin de la métamorphose, et les \glspl{hlb} qui se développent sous l'action des \glspl{ht}.
Ces tissus sont d'autant plus pertinents qu'ils répondent aux \glspl{ht} \textit{ex vivo}, et que, dans le cas du \gls{tf}, le co-traitement avec \glspl{ht} et \glspl{gc} accélère sa régression.
Techniquement, j'ai réalisé des cultures organotypiques, qui sont un modèle d'étude classique de la métamorphose car elles récapitulent la régression du \gls{tf}, indépendamment de toute régulation centrale \citep{Kikuyama1982,Kikuyama1983,Galton1990}.
Le développement \textit{ex vivo} des pattes sous l'effet de la \gls{t3} a déjà été décrit \citep{Tata1991}, mais ces résultats n'ont pas été reproduits au laboratoire.
En outre, l'analyse à grande échelle des effets sur l'expression de gènes cibles des \glspl{ht} dans les bourgeons de queue en développement n'ont pour le moment été abordés que chez \gls{xlaevis}, et par puce à \gls{dna} \citep{Buckbinder1992}.
\par
Actuellement, les seules analyses à grande échelle des effets des \glspl{ht} durant la métamorphose portent sur la régression de la queue, utilisée dans son entier \citep{Wang1993,Helbing2003,Kulkarni2012}.
Nous avons choisi de nous restreindre au \gls{tf} car la diversité des types cellulaires s'en retrouve plus limitée, et qu'il s'agit d'un des premiers tissus à répondre au signal hormonal.
\par
Les expériences menées sur les cultures organotypiques ont été complétées par des expériences similaires conduites sur des têtards entiers et ayant subi les mêmes traitements hormonaux (voir section suivante).
Ceci a pour objectif de découpler les effets des hormones propres au tissu d'intérêt des effets centraux et des artefacts liés aux conditions de cultures.
%
%Les cultures organotypiques sont un modèle classique pour étudier la métamorphose car elle récapitule ce processus indépendamment de toute régulation centrale \insertref{vieux papier culture de queues}.
%Nous avons donc choisi d'utiliser cette approche, en choisissant deux tissus au devenir celulaire différent: le \gls{tf}, qui est résorbé à la fin de la métamorphose \insertref{resorb queue}, et les \glspl{hlb} qui croissent sous l'action des \glspl{ht} \insertref{pousse pattes}.
%Ces tissus sont d'autant plus pertinents qu'ils sont connus pour répondre au signal thyroïdien \textit{ex vivo}, et que, dans le cas du \gls{tf}, le traitement concomitant avec des \glspl{gc} accélère la régression de la queue dépendante des \glspl{ht}.
%Les pattes ont également déjà été décrites comme capable de se développer \textit{ex vivo} sous l'éffet de la \gls{t3} \insertref{tata limb growth}, mais ces résultats n'ont pas été reproduits au laboratoire.
%\par
%Dans la littérature, de nombreux travaux s'intéressant à la régression de la queue utilisent l'organe dans son entier.
%Nous avons choisi de nous restreindre à l'épiderme car la diversité des types cellulaires s'en retrouve plus homogène, et qu'il s'agit d'un des premiers tissus à répondre au signal hormonal \insertref{reg queue epiderme}.
%\par
%Les expériences menées sur les cultures organotypiques ont également été reproduite sur les mêmes tissus provenant de têtards entiers ayant subit un traitement hormonal identique (voir section suivante).
%Ceci a permit de découpler les effets des hormones propres au tissu d'intérêt, les effets centraux et impliquant d'autres organes, et les artefacts liés aux conditions de cultures.

\subsection{Traitements hormonaux}

\subsubsection{Hormones thyroïdiennes}
Les \glspl{ht} utilisées dans le cadre de ces travaux sont la \gls{t3}, car c'est principalement elle qui est directement active sur les tissus cibles.
Une concentration finale (aussi bien en culture organotypique que chez les têtards entiers) de 10 nM de \gls{t3} a été utilisée car elle représente une quantité d'hormone proche de la concentration physiologique au moment du climax rapportée chez \gls{xlaevis} \citep{Leloup1977}.

\subsubsection{Glucocorticoïdes}
Chez les amphibiens, la \gls{cort} est le \gls{gc} naturellement produit lors d'un stress.
Nous avons donc utilisé cette hormone à une concentration finale de 100 nM, reproduisant ainsi la quantité physiologique détectée en cas de stress naturel \citep{JolivetJaudet1984,Krain2004}.

\subsubsection{Temps de traitement}
Les traitements hormonaux sur les cultures organotypiques et les têtards entiers ont été effectués pendant 24 h avant la récupération des tissus.
Cette fenêtre de temps est classiquement utilisée dans la littérature, facilitant la comparaison des résultats obtenus, et permet d'observer les premières réponses moléculaires aux traitements hormonaux.
Pour tous les traitements hormonaux, soit un explant de queue, soit un têtard ont été laissés dans le bain de traitement pendant plusieurs jours afin de s'assurer des effets macroscopiques des traitements réalisés.


\subsection{Extraction des ARN}

Les \glspl{rna} totaux ont été extraits tels que décrit dans \citet{Bilesimo2011}.
Ceux-ci ont systématiquement été quantifiés par spectrophotométrie afin de s'assurer que suffisamment de matériel a été récupéré pour permettre la synthèse de banques et le séquençage.
L'absence de dégradation a également été vérifiée par électrophorèse capillaire en utilisant le ``Bioanalyzer'' (Agilent).
En outre, une reverse transcription et une \gls{rtqpcr} sur le gène \gls{klf9} ont été systématiquement réalisées afin de s'assurer de l'effet moléculaire des deux traitements hormonaux.
Une fois collectés, les \gls{rna} ont été envoyés pour séquençage massif à la plateforme de génomique de l'ENS de Paris.

% ======= END - Collection des échantillons biologiques
% =====================================================

% :::::::::::::::::::::::::::::::::::::::::::::::::::::

% ======================================================
% ======= BEGIN - Séquençage à haut débit des ARN

\section{Séquençage à haut débit des transcrits}

\subsection{Principe et justification}
La partie principale du projet porte sur l'évaluation à l'échelle du transcriptome des interactions croisées entre \glspl{ht} et \glspl{gc}.
Les premières techniques permettant la mesure de la quantité de transcrits à grande échelle sont les puces à \gls{dna}.
Leur essor a été déterminant au cours de la dernière décennie et a révolutionné notre perception de l'activité cellulaire.
Elles souffrent néanmoins d'un certain nombre de défauts, en particulier le fait qu'elles n'autorisent la mesure que de gènes déjà caractérisés \textit{a minima}, que la qualité de la mesure dépend grandement de la qualité de la puce en elle même, et que l'étendue dynamique est plutôt réduite.
Bien que cette technologie ait été centrale dans l'étude de réseaux de régulation chez des espèces modèles telles que l'humain, la souris ou la drosophile, les puces spécifiques de \gls{xtrop} restent incomplètes.
Nous avons par conséquent choisi d'utiliser une technologie de séquençage massif des \glspl{rna}, le \gls{rnaseq}.
Cette technique présente l'avantage d'être naïve dans le sens où la mesure des niveaux de transcrits est indépendante d'une connaissance \textit{a priori} de leur structure.
En effet, comme présenté précédemment, le \gls{rnaseq} peut bénéficier de l'évolution et de l'amélioration constante de l'annotation et de l'assemblage des génomes, alors que la détection de nouveaux transcrits par puce à \gls{dna} nécessiterait le dessein d'une nouvelle puce.
De plus, l'étendue dynamique des mesures est beaucoup plus grande que pour les puces à \gls{dna} (\citealp{Grimaldi2013}; \autoref{sec:ctdb-review}).
Les technologies basées sur le séquençage massif ont un autre avantage, plus subtil, vis à vis des technologies basées sur l'hybridation sur puces à \gls{dna} d'acides nucléiques marqués par des fluorophores couplés à une mesure de fluorescence.
L'hybridation moléculaire de fragments d'acides nucléiques est difficile à standardiser et est très sensible aux (faibles) variations expérimentales.
Elle dépend aussi d'un grand nombre de facteurs physico-chimiques liés à la composition des séquences ou à la qualité de la puce, qui affectent directement la qualité de la mesure.
Il en résulte que la mesure de fluorescence, c'est à dire notre mesure indirecte de la quantité du transcrit correspondant, est fondamentalement bruitée.
Les outils de traitement des données de puces à \gls{dna} essaient de modéliser ces biais afin de les corriger dans une certaine limite.
A l'inverse, pour le \gls{rnaseq}, et c'est l'une de ses forces, "l'hybridation" a lieu \textit{in silico} (étape de "positionnement sur le génome", voir ci-dessous) et offre un bien meilleur contrôle: elle peut par exemple être réalisée à des niveaux de stringence différents afin de caractériser une ou plusieurs composantes de bruit et mieux contrôler la résolution des mesures.

% BOTTOM caption
% ------------------------
\begin{figure}[!htb]
\centering
\vspace{1\baselineskip}
\includegraphics[width=\textwidth]
% ------------------------
%
% SIDE caption
% ------------------------
%\begin{SCfigure}[\sidecaptionrelwidth][!htbp]
%\centering
%\vspace{1\baselineskip}
%\includegraphics[width=0.5\textwidth]
% ------------------------
%
% Main information
% ===========================================================
{Figures/rnaseq-principle/rnaseq-principle.pdf}
\caption[Principe du RNA-Seq]
{
Principe du RNA-Seq.
1) Les ARN totaux sont extraits des tissus (ici queue et bourgeon de membres postérieurs) et sélectionnés sur colonne poly-T.
2) Des \glspl{cdna} sont synthétisés à partir des \glspl{rna}, 3) et fragmentés.
Après séquençage 4) (dans le cadre de ce projet en chimie TruSeq), les "reads" sont repositionnés sur le génome, et comptés dans les modèles de gènes connus 5) afin d'en inférer l'expression différentielle.
}
\label{fig:rnaseq-principle}
% ===========================================================
%
% BOTTOM caption
% ------------------------
\end{figure}
% ------------------------
%
% SIDE caption
% ------------------------
%\end{SCfigure}
% ------------------------
%
%
%\missingfigure{Make a figure}

Le principe de base du \gls{rnaseq} reste fondamentalement simple, et consiste à séquencer des fragments de \gls{cdna} synthétisés après extraction des \gls{rna} et sélection (dans le cas présent) par la queue poly-A.
La composition en acides nucléiques des séquences ("reads"), permet de les repositionner sur le génome.
La quantité de "reads" étant proportionnelle à la quantité de transcrits contenant ces séquences, le profil de densité de "read" le long d'un gène renseigne directement sur son niveau d'expression dans la cellule.
En utilisant cette technologie pour chaque condition expérimentale, il devient possible d'identifier les gènes différentiellement exprimés dans une condition par rapport à une autre.
La \autoref{fig:rnaseq-principle} décrit le principe du \gls{rnaseq} dans le contexte de ce projet.

\subsection{Implémentation: Construction des banques}
Il existe plusieurs implémentations du séquençage des \glspl{rna} en fonction de la plateforme utilisée et la question posée.
Dans le cadre de mes travaux, la construction des librairies et le séquençage en lui même ont été externalisés à l'ENS de Paris.
La chimie utilisée est celle proposée par \textit{Illumina} à travers les kits TruSeq.
Dans ce protocole, les \glspl{mrna} sont purifiés par sélection sur la base des queues poly-A, puis fragmentés et convertis en \glspl{cdna} double brins.
La construction des banques à proprement parler débute par la réparation des extrémités des \glspl{cdna} obtenus à l'étape précédente par phosphorylation et génération de bouts francs en combinant des réactions de réparation de l'\gls{dna} (``fill-in'') et d'activité d'exonucléases.
Une base ``A'' est ajoutée à chaque brins en 3', permettant la ligation ultérieure des adaptateurs possédant un "T" sortant en 3'.
Ces adaptateurs sont des \glspl{dna} double-brin contenant les séquences à la réaction de séquençage (voir plus bas) et les "code-barres" utilisés pour le multiplexage (séquençage de plusieurs banques sur une même ligne et discrimination des ``reads'' d'origines différentes au moment du traitement des données brutes).
Ce protocole très standard, qui permet de séquencer 50 \gls{pb} d'une des extrémités de chaque fragment d'\gls{cdna} sans information de brin (``non-strand specific single end reads''), correspond à l'approche la plus simple à mettre en œuvre pour déterminer de façon robuste le niveau d'expression de gènes.
Après dénaturation et amplification des constructions, le produit final peut être séquencé sur plateforme \textit{HiSeq}.
\par
La technologie Illumina repose sur le principe de "séquençage par synthèse", c'est à dire l'addition séquentielle de nucléotides couplés à des fluorophores, et la mesure de la fluorescence incorporée.
Elle nécessite l'hybridation des produits obtenus à la fin de la construction de la banque par complémentarité avec des amorces spécifiques, complémentaires des adaptateurs, et physiquement attachées à une lame de verre (souvent appelée ``flow cell'').
Des clones de \gls{cdna} physiquement liés à la cellule sont produits par amplification par \gls{pcr}.
Les clones font environs 1 $\mu$m de diamètre ce qui permet à la fluorescence d'être détectée par un dispositif optique.
La densité des clones est un paramètre important qui influe à la fois sur la quantité de "reads" obtenus et sur la qualité de la lecture de chaque base.
À chaque cycle, l'incorporation d'un seul nucléotide par \gls{cdna} fixé à la cellule est détectée par fluorescence.
La nécessité de cliver le fluorophore afin de pouvoir incorporer le nucléotide su cycle suivant rend cette chimie robuste dans le cas de séquences homopolymériques.
\par
Le pré-traitement des données brutes consiste à convertir l'intensité de fluorescence de chaque nucléotide en la base correspondante.
Les modèles mathématiques qui sous-tendent cette étape prennent en compte le chevauchement des spectres d'émission de façon empirique sur la base des premiers cycles.
Par conséquent, il est important que le biais de GC soit le plus petit possible.
Le taux d'erreur de lecture augmente progressivement avec le nombre de cycles.

\subsection{Contrôles qualités}
La qualité des séquences récupérées a été vérifiée à l'aide de la suite logicielle ``FastQC'' (\url{http://www.bioinformatics.babraham.ac.uk/projects/fastqc/}).
Cet outil fournit un ensemble de métriques (longueur, score-qualité, nombre de 'N' des séquences, proportion de séquences dupliquées... ) reflétant la qualité générale de l'ensemble des "reads" et permet de visualiser directement les biais éventuels et les erreurs systématiques, et permet par la suite d'adapter le traitement des données à la qualité des séquences produites.

% ======= END - Séquençage à haut débit des ARN
% =====================================================

% :::::::::::::::::::::::::::::::::::::::::::::::::::::

%======================================================
% ======= BEGIN - Traitement des données

\section{Traitement des données}
%Avant que la composante informative contenue dans les séquences obtenues ne soit exploitable, il est nécessaire de traiter les données.
%Ces manipulation ne concernent qu'un aspect purement technique pré-requis l'exploitation des données et ne correspondent pas à l'analyse de ces dernières.
Les séquences brutes ne sont pas directement exploitables car elles ne reflètent pas (encore) le niveau d'expression des gènes.
Pour cela, il est nécessaire, dans un premier temps, d'effectuer un ensemble d'opérations purement techniques sur ces données (filtrage, rabotage, placement sur le génome...), mais qui se révèlent déterminantes pour leur exploitation biologique ultérieure.
Ces différentes étapes de traitement du signal ne correspondent pas à l'analyse des séquences à proprement parler. 

\subsection{Élimination de la redondance}
%Bien que la quantité d'\gls{rna} fournie ait été largement en excès, des étapes de \gls{pcr} inhérentes au protocole de synthèse des banques ont malgré tout été effectuées.
%Il convient donc d'éliminer toute redondance de séquence au sein des population de "reads".
%Pour cela, si au moins deux "reads" sont identiques, seul celui de meilleur qualité est conservé.
Le séquençage massif est une technologie dérivée de la \gls{pcr}, dont les biais sont susceptibles d'affecter la représentativité des "reads" vis à vis des \gls{cdna} séquencés.
L'artefact le plus fréquent est la sur-amplification préférentielle de certains "reads", dont le nombre ne reflète plus l'abondance du transcrit dont ils dérivent.
Par conséquent, seuls les "reads" uniques sont conservés.
Dans le cas ou un "read" est présent en plusieurs copies identiques, seul celui avec le meilleur score est conservé.

\subsection{Rabotage des séquences ("trimming")}
%Compte tenu des informations obtenues par FastQC, il a été nécessaire d'éliminer les 10 premières bases en 5' car enrichies en résidus de séquences d'adapteurs, et une quantité variable de bases en 3' selon la qualité de la lecture.
%Il est a noter que cette étape a été faite après l'élimination de la redondance.
%Ce choix reflète la très bonne qualité du séquençage.
%En effet, dans le cas contraire, les erreurs de séquences auraient pu empêcher l'élimination correcte de la redondance.
Compte tenu des informations obtenues lors des contrôles qualité, il a été nécessaire d'éliminer 10 bases en 5’ des "reads" car elles correspondent souvent aux extrémités des adapteurs de séquençage.
Il est aussi nécessaire d'éliminer une quantité variable de bases à leur extrémité 3', dés que la qualité de lecture de chaque base devient inférieure à un seuil de 30 sur l'échelle PHRED (correspondant à une probabilité d'erreur de 0.1 \%).
En effet, la qualité de chaque mesure tend à diminuer progressivement à mesure des cycles de séquençage.
Ces deux étapes de rabotage sont très courantes et font partie du processus normal de traitement des données brutes.
Il est à noter que cette étape a pu être réalisée après élimination de la redondance, ce qui reflète la très bonne qualité des séquences obtenues (dans le cas contraire, les erreurs de séquences auraient empêché l'élimination correcte de la redondance).

\subsection{Positionnement des séquences sur le génome}
%L'utilisation des séquences générées pour inférer les niveaux d'expression des gènes nécessite de lier les "reads" à des modèles de gènes.
%Pour cela, chaque "read" est replacé sur le génome de référence (étape de mapping) à l'aide de Bowtie v0.12.7 \citep{Langmead2010a}.
%À chaque modèle de gène est enfin associé le nombre de "reads" chevauchant sa position génomique.
%Cette valeur chiffrée correspond à la mesure brute du signal.
L'utilisation des séquences brutes pour inférer le niveaux d'expression des gènes nécessite d'associer les ``reads'' à des modèles de gènes prédéfinis.
Pour cela, chaque ``read'' est replacé sur le génome de référence (étape de "mapping") à l'aide du logiciel Bowtie v0.12.7 \citep{Langmead2009}. Les paramètres utilisés, qui dépendent fortement de la qualité des ``reads'' et peuvent donc être très variables d'une expérience à l'autre, sont les suivants : -m=1 (Ne garder que les ``reads'' positionnés à une seule position) -5=10 ("trimming" de 10 bases en 5', voir sous-section précédente) -l=26 (longueur de graine de 26 bases) -n=1 (autoriser une seule discordance entre la graine et la séquence de référence).
Les résultats produits par Bowtie sont ensuite transformés en un format plus générique (format BED) afin de faciliter les étapes ultérieures.
À noter, il n'a pas été possible d'utiliser un autre format très standard, le format BAM, qui présente l'énorme avantage d'être compressé (et donc permet de réduire les contraintes de stockage) et indexé, ce qui améliore considérablement les temps de calcul et simplifie beaucoup les étapes suivantes.
En effet, l'assemblage du génome de \gls{xtrop} est composé d'environs 20 000 scaffolds alors que le format BAM ne peut en accommoder qu'une centaine.
Ce type de complication est malheureusement très courant dans les approches génomiques des modèles non conventionnels, ce qui entrave fortement le traitement et l'analyse des données.
Ces outils bioinformatiques sont, en général, très performants pour des modèles courants (souris, humain, drosophile...).
Le nombre de ``reads'' localisé entre les bornes de chaque gène correspond alors à la mesure brute de son niveau d'expression. 
Il faut noter que les ``reads'' peuvent être orientés sur chaque brin, quelle que soit l'orientation du gène.
C'est une conséquence normale du protocole de \gls{rnaseq} utilisé (voir ci-dessus).
De plus, les régions où deux gènes se chevauchent sont exclues de l'analyse car il devient alors impossible de déterminer de quel gène les ``reads'' sont issus.

\subsection{Débruitage et normalisation des données}
En raison de contraintes logistiques, les \gls{rna} des différentes répliques biologiques et les différents tissus n'ont pas été séquencés au même moment.
Entre temps, la plateforme de l'ENS a changé de kit de séquençage (TruSeq v1 à TruSeq v2), ce qui a introduit des biais importants dans la mesure du signal, rendant la comparaison des résultats et l'utilisation des répliques biologiques plus difficiles.
Les sources de variabilité associées à chaque set de données ont été caractérisées par \gls{acp} et la composante principale strictement associée à des effets de lot a été éliminée.
Cette méthode de réduction du bruit technologique est utilisée dans le cadre de la comparaison de résultats de puces à \gls{dna} issus de plateformes différentes.
\par
Les mesures brutes du niveau d'expression des gènes dépendent de plusieurs paramètres.
Outre le niveau réel d'expression des gènes, ces valeurs dépendent aussi de plusieurs facteurs, dont les deux principaux sont :
\begin{itemize}
\item la profondeur de séquençage, c'est à dire le nombre total de ``reads'' uniques replacés sur le génome, pour chaque réplique biologique.
Plus il y a de ``reads'', plus cette valeur est élevée, à niveau d'expression équivalent. Corriger les valeurs brutes d'expression des gènes en les multipliant par un facteur correctif pour les rapporter à un nombre total de ``reads'' équivalent serait a priori une approche satisfaisante.
Il semble cependant que ce ne soit pas le cas et qu'un sous ensemble de quelques gènes différentiellement et fortement exprimés puisse l'affecter énormément.
La profondeur de séquence est donc normalisée par une approche légèrement différente, moins sensible aux extrêmes, et basée sur le calcul de la médiane des rapports d'expression brutes des gène (voir détails dans \citealp{Anders2010}).
\item la taille du transcrit. Plus ce dernier est grand, plus il capturera de ``reads'' comparé à un transcrit plus petit, à niveau d'expression équivalent.
Ce biais ne s'applique pas dans le cadre de mon travail, qui ne se base pas sur des mesures absolues du niveau d'expression des gènes.
Leurs niveaux d'expression après chaque traitement est exprimé relativement à celui de la condition contrôle.
\end{itemize}

\subsection{Mesure de l'expression différentielle des gènes}\label{subsec:diff-expr-call}
Le travail de traitement des données du \gls{rnaseq} se conclut sur la mesure de l'expression différentielle des gènes.
Avant d'utiliser un outil faisant appel à un modèle statistique pour déterminer les gènes différentiellement exprimés, les gènes très faiblement exprimés dans l'ensemble des traitements et n'ayant aucune chance de passer les seuils statistiques ultérieurs sont retirés du jeu de donnée.
Cette étape de ``filtre indépendant'' (de tout test statistique) permet de gagner en puissance statistique et est classiquement appliquée au cours de l'analyse de données de puces à \gls{dna} ou de \gls{rnaseq}.
Pour cela, j'ai utilisé le paquet R ``DESeq'' \citep{Anders2010} qui base sa méthode de détection des gènes différentiellement exprimés sur la modélisation des nombres de ``reads'' associés à chaque modèle de gènes par une distribution binomiale inversée.
Cette dernière présente l'avantage de mieux modéliser la distribution du nombre de ``reads'' par rapport à une loi de Poisson, en particulier pour les gènes associés à un nombre moyen à faible de ``reads''.
L’utilisation de cette méthode fait partie des standards actuels de l’analyse de données de \gls{ngs} \citep{Anders2013}.
Les contrastes utilisés correspondent à ceux définis entre la condition ``contrôle'' et ceux issus des trois traitements hormonaux.


\section{Analyse des données}

\subsection{Annotation fonctionnelle automatisée}
L'annotation fonctionnelle de sets de gènes a été dans un premier temps effectuée de façon automatisée en se basant sur les liens établis entre gènes et fonctions biologiques dans le cadre de la \gls{go}.
Il est à noter que malgré l'avantage certain qu'offre ce type de méthode concernant la standardisation de l'analyse, l'annotation \gls{go} est particulièrement biaisée vers des processus très étudiés.
\par
L'outil en ligne utilisé pour effectuer le test statistique d'enrichissement et la représentation des résultats sous forme de graphe orienté acyclique est GOrilla \citep{Eden2009}.
Cet outil implémente la détermination de termes enrichis en se basant sur une distribution hypergéométrique décrite par trois paramètres:
le nombre de gènes total ($N$), le nombre de gènes associé à un terme \gls{go} particulier ($B_{terme GO}$) et le nombre de gènes testés ($n$).
La queue de la distribution permet de déduire la probabilité qu'un certain nombre de gènes parmi la liste testée soient associés au terme \gls{go} considéré.
\par
Il est important de préciser que lors d'une analyse d'enrichissement de termes fonctionnels, il est crucial de bien choisir l'ensemble de gènes servant de base et contre lequel l'enrichissement sera estimé ($N$) (généralement appelé ``background'').
Ainsi, l'utilisation d'un ``background'' artificiellement appauvri en gènes participant à une fonction biologique particulière résultera en un enrichissement apparent de cette fonction biologique quelque soit le set de gènes utilisé.
Au contraire, en ``background'' déjà enrichi en gènes liés à une fonction biologique particulière ne permettra pas sa mise en évidence.
\par
Dans le cadre de l'analyse des résultats présentés dans ce manuscrit, deux usages ont été faits :
Le premier utilise comme ``background'' l'ensemble des gènes exprimés.
Ceci permet de mettre en évidence les fonctions biologiques globalement affectées par les traitements hormonaux.
Le second utilise comme ``background'' la liste des gènes différentiellement exprimés dans au moins un traitement.
Dans ce cas, le contraste mis en avant est celui de la fonction d'un sous-groupe de gènes différentiellement exprimés par rapport à l'effet global des deux hormones.

\subsection{Curation manuelle de l'annotation fonctionnelle}\label{subsec:manual-annot}
Pour aller plus loin dans la caractérisation des rôles biologiques des différentes catégories de profils, j'ai réalisé une curation manuelle des fonctions connues de chaque gènes sur la base des informations disponibles sur les pages ``gene'' du NCBI (\url{http://www.ncbi.nlm.nih.gov/gene}), GENECARDS (\url{http://www.genecards.org/}) et la littérature scientifique.
Bien que cette annotation ne prenne pas en compte l'aspect hiérarchique des fonctions biologiques et moléculaires (et se retrouve en conséquence polluée par une certaine redondance), elle a le mérite d'intégrer des informations moins biaisées et plus récentes d'une part; d'intégrer des liens entre fonctions moléculaires et fonctions biologiques (à plus haut niveau), parfois absents, d'autre part.
Ce type d'analyse complémentaire, bien que moins orthodoxe, a déjà été utilisée dans le cadre de travaux publiés \citep{Hendrix2008} et dans le cas de la dissection de phénomènes biologiques fins, justifie l'exploitabilité des données.

\end{document}