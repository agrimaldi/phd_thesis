 % -*- root: ../main.tex -*-
\documentclass[../main.tex]{subfiles}
\begin{document}

\chapter{Contexte et Objectifs du Projet}

%\epigraph{The saddest aspect of life right now is that science gathers knowledge faster than society gathers wisdom.}{Isaac Asimov}


% =====================================================
% ======= BEGIN - Contexte du projet

\section{Contexte: La période périnatale}

% -------------------------------
% +++ BEGIN - Perturbation de la période périnatale

	\subsection{Une perturbation de la physiologie à la période périnatale peut avoir des effets déletères pour l'organisme}
		Il est supposé que l'exposition à des conditions adverses au moment de la période périnatale peut susciter des adaptations développementales servant à maximiser les chances de survie durant cette période et à préparer l'organisme à une variété d'environnements potentiellement hostiles \citep{Angelier2013,Crespi2013,Patterson2014}.
		Ces adaptations apparentes peuvent varier en fonction des individus \citep{Cockrem2013}.
		Ces processus, bien qu'adaptatifs, peuvent malheureusement aussi se révéler être inadéquats plus tard durant la vie et peuvent altérer les processus normaux de vieillissement.
		Il est maintenant clairement établi qu'une exposition précoce à divers stress contribue au développement de plusieurs pathologies ou syndromes se déclarant plus tard durant la vie, incluant des maladies mentales et cardiovasculaires, une prédisposition à l'obésité et au cancer \citep{Braun2013,Harris2011,Plagemann2006,Challis2000,Davis2013,Barnes2011}.
		Le fait qu'une exposition ponctuelle à un stimulus/stress s'accompagne d'une réponse biologique beaucoup plus tardive suggère fortement qu'il existe une composante épigénétique dans la genèse de ces pathologies ultérieures (pour exemple \citealp{Weaver2004,Begum2013,Crudo2012,Drake2012}).
		La modulation de ces processus dans un contexte physiologique est en partie régulée par la réponse au stress.
		Ainsi, un stress prolongé ou particulier dans son origine ou sa manifestation peut se traduire par la favorisation de pathologies dans un contexte multifactoriel.
		Chez une vaste majorité des vertébrés, la période périnatale est marquée par des voies de signalisation endocriniennes qui contrôlent de nombreux processus développementaux, physiologiques et morphologiques.
		Deux de ces signaux endocriniens sont les \glspl{gc} et les \glspl{ht}.

	% +++ END - Perturbation de la période périnatale
	% -------------------------------
	% +++ BEGIN - Roles des GC

	\subsection{Rôles des glucocorticoïdes}
		Les \glspl{gc} jouent un rôle important durant la période périnatale.
		Leurs rôles dans le développement des poumons chez l'homme en font une cible thérapeutique de choix afin d'induire la maturation du système respiratoire chez les nouveaux nés prématurés \citep{NGC2010}.
		Les \glspl{gc} sont notamment utilisés en routine pour le traitement de l'asthme \citep{Clifton2005,Osei-Kumah2011} et l'hyperplasie congénitale de la glande surrénale \citep{Forest2004}.
		Ils peuvent à eux seul déclencher la parturition chez les ovins et les humains \citep{Mati1973,Whittle2001}.
		L’administration pharmaceutique de corticostéroïdes n'est toutefois pas la seule source de \glspl{gc}.
		Ces derniers étant en effet les médiateurs principaux de la réponse au stress à moyen ou long terme (voir \autoref{subsec:gc-role-stress}), les conditions environnementales durant la gestation ou à la naissance peuvent influer sur leur taux circulants, contribuant ainsi à l'établissement de phénotypes altérés et potentiellement adaptés à un environnement particulier.

	% +++ END - Roles des GC
	% -------------------------------
	% +++ BEGIN - Roles des HT

	\subsection{Rôles des hormones thyroïdiennes}
		Au cours de la période périnatale, les \glspl{ht} jouent un rôle crucial dans la maturation de nombreux tissus et organes, et les pathologies associées à la perturbation de leur signalisation sont parmi les plus répandues dans le monde (voir introduction).
		D'un côté, une déficience ponctuelle en \glspl{ht} à la naissance induit le ``crétinisme'', un syndrome à large spectre, allant de défauts de croissance et de développement cognitif modéré jusqu'à des dommages cérébraux et neurologiques graves \citep{Delange1994,Chen2010}.
		De l'autre, l'excès d'\gls{ht} se traduit par la thyrotoxicose, et potentiellement le décès.
		Aussi bien l'hypothyroïdisme que l'hyperthyroïdisme ont été associés à des risques accrus de maladies cardiovasculaires, de pathologies dégénératives du squelette et de cancers.
		De façon intéressante, la période périnatale est marquée chez la quasi-totalité des vertébrés par une augmentation transitoire et marquée de la concentration circulante en \glspl{ht}.
		La reproduction artificielle de ce pic d'\glspl{ht} à la naissance suffit à prévenir des effets délétères dus à un hypothyroïdisme congénital.

% +++ END - Roles des HT
% -------------------------------

% ======= END - Contexte du projet
% =====================================================

% :::::::::::::::::::::::::::::::::::::::::::::::::::::

% =====================================================
% ======= BEGIN - Objectifs

\section{Objectifs}

% -------------------------------
% +++ BEGIN - Caractérisation des intéractions croisées entre HT et GC

	\subsection{Caractérisation des interactions croisées entre hormones thyroïdiennes et glucocorticoïdes}
		Plusieurs lignes d'évidence suggèrent une interaction croisée entre les \glspl{ht} et les \glspl{gc}, en particulier \textit{via} la régulation de l'expression des enzymes responsables de l'activation ou l'inactivation des \glspl{ht} (\gls{dio2} et \gls{dio3} respectivement), contribuant ainsi à réguler le métabolisme périphérique des \glspl{ht} et leur disponibilité locale.
		Il est intéressant de noter que les \glspl{gc} exercent une action stimulatrice sur le métabolisme des \glspl{ht} durant le développement embryonnaire.
		Chez les fétus ovins par exemple, l'administration de cortisol résulte en une conversion accrue de la \gls{t4} en \gls{t3} dans le foie et les reins \citep{Wu1978}, et contribue à l'augmentation de \gls{t3} circulante \citep{Darras1996}.
		Au contraire, dans les périodes post-embryonnaires de la vie, les \glspl{gc} ont tendance à inhiber la fonction thyroïdienne \citep{Chopra1975,Decuypere1983,Bianco1987}.
		\par
		L'exposition aux \glspl{gc} provoque chez le rat – et à long terme – une baisse de l'expression de la \gls{trh} (qui induit la synthèse de \gls{tsh} qui elle même va favoriser la synthèse d'\glspl{ht}, voir \autoref{subsec:th-prod-control}).
		Ce changement n'a pas d'effet sur les niveaux d'\glspl{ht} sauf à la période périnatale (au moment où il doivent être élevés) où ils sont inférieurs à la normale \citep{Carbone2012}.
		Il a aussi été observé que les \glspl{ht} étaient à l'origine de certains effets pro-inflammatoires attribués aux \glspl{gc} \citep{Montesinos2012}.
		En effet, malgré leur action anti-inflammatoire, les \glspl{gc} ne sont pas toujours capables de contrer l'immunostimulation induites pas les \glspl{ht}.
		Enfin, il a été montré dans un des modèles les plus classique d'exposition aux \glspl{gc} – soit par administration, soit résultant d'un stress maternel – que l'augmentation du risque de développer des anomalies comportementales, métaboliques ou endocriniennes suite à des modifications épigénétiques \cite{Weaver2004}, impliquait la signalisation thyroïdienne \citep{Hellstrom2012}.
		Les interactions croisées entre \glspl{ht} et \glspl{gc} font que l'altération des niveaux de \glspl{gc} peut influer la signalisation thyroïdienne a un moment de la vie ou son rôle dans le développement post-embryonnaire peut avoir des conséquences délétères à plus ou moins long terme.
		Cette altération des rhéostats physiologiques à la période périnatale peut ainsi être à l'origine d'une certaine plasticité phénotypique, permettant à l'organisme de s'adapter de façon précoce à un environnement adverse.
		\par
		Malgré leur contribution importante dans le cadre des interactions individus-environnement, et leur impact sociétal important chez l'humain, le détail moléculaire des interactions entre ces deux voies de signalisation est mal caractérisé.
		\par
		Bien qu'une partie des effets des \glspl{gc} sur la signalisation thyroïdienne puisse passer par une régulation de la disponibilité en ligand, certains gènes comme \gls{klf9} sont régulés de façon directe par les deux hormones \citep{Denver2009b,Bagamasbad2012}.
		Ceci suggère que les interactions \glspl{ht} – \glspl{gc} font intervenir un ensemble de mécanismes moléculaires.
		Nous avons donc entreprit d'apporter des éléments de réponse aux questions suivantes:
		\begin{itemize}
			\item
				Quelle est la nature des interactions entre les voies de signalisation des \glspl{ht} et des \glspl{gc} ?
			\item
				Quelle est l'ampleur de ces interactions; se traduisent-elles par une reprogrammation majeure de l'expression des gènes au niveau des tissus cibles ?
			\item
				Les interactions entre \glspl{gc} et \glspl{ht} sont-elles dominées par les conséquences fonctionnelles de la modulation de la disponibilité en ligand ?
			\item
				Quels sont les processus biologiques affectés ?
		\end{itemize}
		\par
		Pour ce faire, nous avons entrepris de mesurer à grande échelle les effets d'un traitement aux \glspl{ht}, aux \glspl{gc}, et d'un co-traitement avec les deux hormones sur l'abondance relative des transcrits.
		Cette démarche permet de capturer la réponse de l'organisme à l'échelle du transcriptome, et ainsi mettre en évidence des programmes et des réseaux de régulation importants durant une phase du développement où ces deux hormones jouent un rôle crucial.
		\par
		Nous avons choisi d'aborder ces questions dans le cadre de la métamorphose des amphibiens Anoures.
		Comme décrit dans les sections suivantes, ce modèle s'avère particulièrement adapté.
		%De façon intéressante, un processus moléculairement proche de la période périnatale des amniotes est la métamorphose des amphibiens.

		%Afin d'isoler le noyau dur des réponses aux \gls{ht} et aux  \glspl{gc}, nous avons étudié le transcriptome de deux organes au destin cellulaire différents.
		%D'une part l'épiderme caudal, qui au cours de la métamorphose se résorbe, d'autre part les membres postérieurs, qui en croissant, font intervenir des processus de prolifération et différentiation cellulaire (\autoref{tab:metamorphosis})

% +++ END - Caractérisation des interactions croisées entre HT et GC
% -----------------------------------

% ======= END - Objectifs
% =====================================================

\end{document}