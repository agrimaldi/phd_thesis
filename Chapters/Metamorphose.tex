 % -*- root: ../main.tex -*-
\documentclass[../main.tex]{subfiles}
\begin{document}

\chapter{Le Modèle de la Métamorphose des Amphibiens}\label{chap:metamorphosis}

%\epigraph{“Pleasure to me is wonder—the unexplored, the unexpected, the thing that is hidden and the changeless thing that lurks behind superficial mutability. To trace the remote in the immediate; the eternal in the ephemeral; the past in the present; the infinite in the finite; these are to me the springs of delight and beauty.”}{H.P. Lovecraft}


% =====================================================
% ======= BEGIN - Le développement du xénope

\section{Le développement du Xénope: un modèle de développement des amphibiens Anoures}

	Le développement des amphibiens est caractérisé par une ontogénie incluant la transition d'un état larvaire à un état adulte.
	Après l'éclosion et une période de croissance, le têtard va ainsi devoir subir une transition physiologique, anatomique et environnementale rapide et irréversible marquant la transition de l'état larvaire à l'état d'adulte, la métamorphose.
	Le développement du Xénope peut se résumer à 4 phases principales décrites et formalisées en 66 stades chez \gls{xlaevis} par \citet{Nieukoop1956}, les \glspl{snf}.
	Ces stades sont caractérisés par des étapes précises du développement morphologique et anatomique.

	\subsection{Le développement embryonnaire et l'organogenèse}
		Cette phase du développement correspond à la mise en place précoce des structures et fonctions qui donneront lieu à la formation d'un têtard autonome.
		Elle englobe les premiers stades de développement, à partir de la fécondation de l'ovocyte jusqu'à l'acquisition de la capacité à se nourrir de façon autonome.
		Après la fécondation, le zygote va se développer par mitoses successives rapides et synchrones.
		La dynamique de ces premières étapes est basée sur la synthèse d'\glspl{rna} maternels accumulés dans l'ovocyte, et se fait donc en absence de transcription.
		Les gènes zygotiques ne s'expriment qu'à partir de la transition mid-blastuléenne (\gls{snf} 8).
		Durant la gastrulation (\glspl{snf} 9-13), les feuillets embryonnaires se mettent en place.
		La neurulation, qui a lieu aux \glspl{snf} 13-20, est suivie par l'organogenèse (\glspl{snf} 20-45).
		Le têtard commence à se nourrir à partir des \glspl{snf} 44-45, tandis que son intestin et les branchies se développent.
		Ceci marque la fin de l'organogenèse et le début de la pré-métamorphose.

	\subsection{La pré-métamorphose}
		À partir du \gls{snf} 45, le têtard va essentiellement croitre, passant d'une larve d'environs 5 mm à une larve de plus de 2 cm au \gls{snf} 50.
		Les derniers stades de la pré-métamorphose sont marqués par la maturation fonctionnelle de la glande thyroïde et le début de sécrétion d'\glspl{ht} endogènes.

		\subsection{La pro-métamorphose}
		Durant cette phase, les niveaux d'\glspl{ht} augmentent progressivement mais restent modestes.
		La pro-métamorphose est caractérisée par le développement des pattes postérieures à partir du \gls{snf} 51 (\autoref{fig:metamorphosis}).
		Le têtard continue à grandir et se prépare à la transition drastique que représente le climax de la métamorphose.

		% BOTTOM caption
% ------------------------
\begin{figure}[!htb]
\centering
\vspace{1\baselineskip}
\includegraphics[width=\textwidth]
% ------------------------
%
% SIDE caption
% ------------------------
%\begin{SCfigure}[\sidecaptionrelwidth][!htbp]
%\centering
%\vspace{1\baselineskip}
%\includegraphics[width=0.5\textwidth]
% ------------------------
%
% Main information
% ===========================================================
{Figures/metamorphosis/metamorphosis.pdf}
\caption[Changements anatomiques durant la métamorphose du Xénope]
{
Changements anatomiques durant la métamorphose du Xénope.
Figure inspirée de \citet{Shi2001}.
Les modifications morphologiques et biochimiques majeures survenant durant la métamorphose sont présentées table \ref{tab:metamorphosis}.
La longueur des queues entre les \glspl{snf} 63-66 et des pattes entre les \glspl{snf} 52-58 sont à l'échelle.
Les schémas de coupes transversales d'intestins mettent en évidence les différences entre le système digestif larvaire et adulte.
Le premier ne possède qu'une involution, et est riche en tissus conjonctif.
Le second possède un épithélium comportant plusieurs replis, intégré dans une structure complexe présentant tissus conjonctifs entre la couche épithéliale et la couche musculaire à la périphérie.
Les ovales gris foncés et blancs représentent respectivement les cellules épithéliales adultes en prolifération et les cellules épithéliales larvaires en cours d'apoptose.
Les schémas des différents stades sont issus de \citet{Nieukoop1956}.
}
\label{fig:metamorphosis}
% ===========================================================
%
% BOTTOM caption
% ------------------------
\end{figure}
% ------------------------
%
% SIDE caption
% ------------------------
%\end{SCfigure}
% ------------------------
%
%
%\missingfigure{Make a figure}

	\subsection{Le climax de la métamorphose}
		La fin de la pro-métamorphose est morphologiquement marquée par l'apparition des membres antérieurs vers le \gls{snf} 59.
		Ceux-ci ont commencé à se développer presque en même temps que les membres inférieurs, mais sont restés jusqu'alors sous l'épiderme.
		Durant cette phase, le têtard cesse de s'alimenter, une conséquence du profond remodelage de son épithélium intestinal.
		La structure squelettique du têtard, et en particulier de la tête, est profondément remodelée pour laisser place au développement du tronc.
		La fin de la métamorphose est marquée par la résorption rapide de la queue (\autoref{fig:metamorphosis}).
		De façon générale, la totalité des tissus larvaires sont remodelés ou résorbés. Le \autoref{tab:metamorphosis} présente les modifications anatomiques et biochimiques majeures survenant durant la métamorphose.

		\setlength{\extrarowheight}{5px}

\begin{table}[!htbp]
\footnotesize

\def\tabularxcolumn#1{m{#1}}
\newcolumntype{L}{>{\setlength\hsize{0.4\hsize}\raggedright}X}
\newcolumntype{M}{>{\setlength\hsize{1\hsize}\raggedright}X}
\newcolumntype{N}{>{\setlength\hsize{1\hsize}\raggedright}X@{\hskip 0.1in}}
\newcolumntype{O}{>{\setlength\hsize{1\hsize}\centering}X}
\newcolumntype{P}{>{\centering\setlength\hsize{2\hsize}}X}

\begin{tabularx}{\textwidth}{L N M}

\toprule

\textbf{Tissu}		& \multicolumn{2}{P}{Réponse} \tabularnewline
					  
					  \cmidrule(rl){2-3}

					& Morphologique		& Biochimique \tabularnewline
					
Cerveau
					& Remodelage, croissance des axones, prolifération et mort cellulaire
					& Division cellulaire, apoptose, synthèse protéique \tabularnewline

Foie
					& Remodelage, différentiation fonctionnelle
					& Induction des enzymes du cycle de l'urée et de l'albumine; Passage de l'hémoglobine larvaire à l'hémoglobine adulte \tabularnewline

Œil
					& Repositionnement; nouvelles connexions et neurones rétinaux; remodelage de la lentille
					& Transformation des pigments visuels (porphyropsine-rhodopsine) \tabularnewline

Peau
					& Remodelage; formation des glandes granulaires; kératinisation; apoptose
					& Induction des collagènes, kératines et magainines adultes; induction de collagènases \tabularnewline

Membres
					& Formation \textit{de novo} d'os, de cartilage; de muscle, de nerfs, d'épithélium
					& Prolifération et différentiation; apoptose; chondrogenèse et ossification \tabularnewline

Poumons
					& Formation \textit{de novo} d'épithélium
					& Prolifération et différentiation \tabularnewline

Queue, Branchies
					& Régression complète
					& Apoptose; destruction de la matrice extracellulaire; induction et activation d'enzymes lytiques (collagénases, metalloprotéinases, nucléases); Prolifération des lysosomes; prolifération des macrophages \tabularnewline

Pancréas, Intestins
					& Profond remodelage
					& Reprogrammation du phénotype; induction de protéases, protéines de liaison aux acides gras et stromelysine-3 \tabularnewline

Système immunitaire
					& Redistribution des types populations cellulaires, changement de soi	
					& Altération du système immunitaire et apparition de nouveaux composants immuno-compétents; Prolifération transitoire dans le contexte des tissus apoptotiques	\tabularnewline

Muscle
					& Changement de motilité, croissance et différentiation; apoptose
					& Transition forme larvaire vers adulte, induction de la chaîne lourde de myosine \tabularnewline

\bottomrule

\end{tabularx}
\caption[Effets morphologiques et biochimiques survenant durant la métamorphose]
{
Effets morphologiques et biochimiques survenant durant la métamorphose.
Tiré de \citet{Tata2006}.
}
\label{tab:metamorphosis}

\def\tabularxcolumn#1{p{#1}}
\end{table}

		Le climax de la métamorphose marque la période durant laquelle les modifications les plus spectaculaires ont lieu.
		Il en résulte des modifications qui vont séparer écologiquement les grenouilles adultes de leur progéniture.
		En effet, la métamorphose s'accompagne d'un changement de mode vie (alimentation, locomotion ...).
		Cette transition écologique est rapide (quelques jours à quelques semaines selon l'espèce), et nécessite une coordination temporelle et spatiale de la réponse au signal déclenchant la métamorphose.
		Ce signal a été montré par Gudernatch en 1912 comme étant les \glspl{ht}.
		Il observait alors que l'administration d'extraits de glande thyroïde à l'eau d'aquariums contenant des têtards induisait la métamorphose prématurée de ces derniers.
		Chez tous les vertébrés, les \gls{ht} jouent un rôle important pendant une fenêtre de temps courte du développement post-embryonnaire, et vont être indispensables à la maturation de nombreux tissus \citep{Laudet2011b}.
		La conservation au cours de l'évolution de cette fenêtre d'action et des mécanismes d'actions impliqués font de la métamorphose des amphibiens un modèle de choix pour l'étude des \glspl{ht}.


% =====================================================
% ======= BEGIN - Rôle des hormones thyroïdiennes

\section{Hormones thyroïdiennes et métamorphose}

	Chez les amphibiens, les \glspl{ht} jouent véritablement le rôle d'interrupteur à usage unique permettant l'initiation de la métamorphose \citep{Shi1998}.
	L'élucidation du rôle et des mécanismes d'action des \glspl{ht} a énormément bénéficié de ce modèle et réciproquement \citep{Wong1995,Wong1998a}.

	% -------------------------------
	% +++ BEGIN - Taux d'HT durant la métamorphose

	\subsection{Taux d'hormones thyroïdiennes circulantes au cours du développement du Xénope}

		Gudernatsch observait il y a plus d'un siècle que de l'extrait de glande thyroïde, et uniquement de cet organe, était capable d'induire la métamorphose de têtards.
		Il faudra toutefois attendre 1977 pour que les niveaux de \gls{t3} et de \gls{t4} chez des têtards en cours de métamorphose soient mesurés \citep{Leloup1977}.
		Bien que la glande thyroïde ne devienne fonctionnelle qu'à partir du \gls{snf} 45, les stocks d'\glspl{ht} d'origine maternelle sont détectés dès les premiers stades \citep{MorvanDubois2006}.
		Au cours de la métamorphose, les niveaux \glspl{ht} endogènes augmentent progressivement à partir du \gls{snf} 54 jusqu'au \gls{snf} 59.
		Durant les stades ultérieurs, au moment du climax, les taux d'hormones augmentent considérablement et transitoirement (quelques jours), pour revenir à des niveau basaux chez la grenouille juvénile.
		Il est à noter que le pic de \gls{t4} est plus étalé que celui de la \gls{t3}, et atteint son maximum un peu plus tardivement (\gls{snf} 63 vs 61) (voir Figure~1 \citealp{Grimaldi2012}; Annexe~\ref{sec:bba-review}).

	% +++ END - Taux d'HT durant la métamorphose
	% -------------------------------
	% +++ BEGIN - Compétence des tissus aux HT

	\subsection{Compétence des tissus aux hormones thyroïdiennes et diversité d'actions}\label{tissue-compet-ht}

		La métamorphose de l'amphibien est un processus complexe caractérisé par deux paradoxes:
		\begin{itemize}
		\item Comment les \glspl{ht}, signal initiateur et indispensable de la métamorphose, peuvent avoir des effets aussi divers selon les tissus ? 
		\item Alors que les \glspl{ht} circulantes augmentent considérablement durant la métamorphose, comment les différentes étapes sont aussi finement coordonnées temporellement ?
		\end{itemize}
		\par
		Une partie de la réponse à ces questions prend source dans la dynamique de la compétence des différents tissus à répondre au signal thyroïdien.

		\subsubsection{Régulation de la disponibilité en ligand}
			L'activité des \glspl{ht} est régulée en partie localement au niveau de la cellule notamment par les désiodases et les \glspl{cthbp}.
			Ces dernières régulent la portion libre d'\gls{ht} au sein de la cellule et jouent un rôle dans leur stockage, export et métabolisme.
			Dans des tissus sur le point de subir un effet dramatique des \gls{ht} (pousse des membres ou régression de la queue), le gène codant pour \gls{cthbp} est hautement transcrit \citep{Shi1994}.
			Les désiodases sont, quant à elles, essentielles dans la modulation des ratios d'\gls{ht} active par rapport à des précurseurs ou des métabolites inactifs.
			En particulier, l'expression de la \gls{dio2} (enzyme activant les \glspl{ht}) est élevée dans des tissus en cours de métamorphose, alors que l'expression de la désiodase inactivante (\gls{dio3}) est élevée dans les tissus avant et après métamorphose \citep{Leloup1981,Galton1989}.
			De façon intéressante, l'expression des désiodases est connue pour être modulée par les \glspl{ht}. Toutefois, \citet{Bonett2010} ont montré que \gls{dio2} répondait beaucoup moins rapidement à la \gls{t3} selon le type cellulaire.

		\subsubsection{Le modèle de la double fonction des récepteurs aux hormones thyroïdiennes}
			Un autre aspect important pour la coordination spatiale et temporelle de la capacité des différents tissus à répondre aux \glspl{ht} se trouve dans les profils d'expression des \glspl{tr} dont la caractérisation au cours de la métamorphose a été réalisée très tôt \citep{Yaoita1990}.
			Au cours du développement du Xénope, la combinatoire entre \gls{ht} et \gls{tr}, associée au modèle selon lequel la présence d'\glspl{ht} détermine l'action répressive ou activatrice de leurs récepteurs, permet de dégager un rôle double de la signalisation thyroïdienne (voir Figure~1 \citealp{Grimaldi2012}; Annexe~\ref{sec:bba-review}).
			Ainsi durant la pré- et la pro- métamorphose, l'absence relative d'\gls{ht} et l'augmentation croissante de l'expression de \gls{tra} contribue à réprimer les gènes cibles.
			Pendant le climax de la métamorphose, la concentration élevée en \gls{ht} concomitante à l'expression de \gls{tra} et \gls{trb} contribue à l'activation de la transcription gènes cibles cruciaux durant ce processus développemental \citep{Sachs2000}.
			L'utilisation d'animaux transgéniques exprimant des formes dominant-négative \citep{Schreiber2001,Buchholz2003} ou -positive \citep{Buchholz2004} confirme le rôle important de l'activation de \gls{tr} par les \glspl{ht} dans le recrutement différentiel de co-régulateurs essentiel pour la métamorphose (voir \citealp{Grimaldi2012}; Annexe~\ref{sec:bba-review}).

		\subsubsection{Spécificité d'action des isoformes des récepteurs aux hormones thyroïdiennes}
			L'étude de l'expression spatiale et temporelle des différentes isoformes de \gls{tr} et l'utilisation d'analogues des \glspl{ht} spécifiques ont pu mettre en évidence un rôle de \gls{tra} dans la prolifération cellulaire au niveau du cerveau et des membres postérieurs.
			Au contraire de \gls{tra}, l'expression de \gls{trb} est dépendante de l'augmentation de la concentration en \gls{ht} au cours de la métamorphose.
			L'utilisation d'analogues des \gls{ht} spécifiques de \gls{trb} supporte l'hypothèse de l'implication de cette isoforme dans des processus de différentiation cellulaire et d'apoptose.
			Pour revue, voir \citep{Furlow2006,Denver2009a}.

		\subsubsection{L'autoinduction de \gls{trb}}
			L'auto-induction de \gls{trb} était initialement supposée être la réponse la plus précoce au signal thyroïdien \citep{Yaoita1990}.
			Celle-ci est sous-tendue en partie par la présence d'au moins deux \glspl{tre} (chez \gls{xtrop} \citealp{Bilesimo2011}, quatre chez \gls{xlaevis}, \citealp{Ranjan1994,Machuca1995,Urnov2001}) en amont du gène, suggérant que \glspl{trb} est une cible directe des \glspl{ht}.
			Deux mécanismes peuvent expliquer le rôle de l'auto-induction de \gls{trb} dans la métamorphose.
			D'une part l'accumulation de \gls{trb} peut causer la sensibilisation des tissus aux \glspl{ht}, amplifiant ainsi les effets de ces dernières sur l'expression de gènes cibles.
			D'autre part, \gls{trb} pourrait réguler un sous-ensemble de gènes et de processus distincts de ceux régulés par \gls{tra}.
			\par
			Pour compléter ces informations sur la régulation de l'expression de \gls{trb} par les \glspl{ht}, il est important de noter que l'utilisation de \gls{chx} a montré que la transcription de facteurs additionnels était nécessaire à l'induction maximale de \gls{trb} par la \gls{t3}.
			Cette hypothèse fut corroborée par la découverte d'un gène répondant plus précocement aux \glspl{ht} que \gls{trb}, le gènes codant pour \gls{klf9}.
			Il s'agit d'un facteur de transcription faisant partie de la famille des facteurs krüppel-like \citep{Knoedler2014}, identifié comme important dans le développement et l'extension de neurites \citep{Scobie2009}.
			La découverte de \gls{klf9}, et le fait que \gls{trb} comprend plusieurs sites putatifs de liaison de \gls{klf9}, suggère que \gls{trb} nécessite partiellement \gls{klf9} dans le cadre de son auto-induction \citep{Bagamasbad2008}.
			Le modèle actuel décrit la nécessité précoce de \gls{tra} pour assurer la compétence à répondre des différents tissus, tandis que \gls{trb} est requis pour la métamorphose proprement dite.
			Ainsi, des variations tissu-spécifique des profils d'expression des deux \glspl{tr} pourraient expliquer des dynamiques de réponses caractéristiques de chaque organe.


% +++ END - Compétence des tissus aux HT
% -------------------------------

% =====================================================
% ======= BEGIN - Roles des GC

\section{Rôle des glucocorticoïdes}

	\subsection{Taux de glucocorticoïdes au cours de la métamorphose}

		De même que les \glspl{ht}, les \glspl{gc} sont essentiels pendant la métamorphose.
		Chez le têtard de \gls{xlaevis}, la concentration en \gls{cort} est élevée pendant la pré-métamorphose, et le climax (\gls{snf} 62, \autoref{fig:gc-gr-metamorphosis}), mais diminue considérablement durant la pro-métamorphose \citep{JolivetJaudet1984,Kloas1997}.
		Des profils similaires de concentration en \glspl{gc} sont observés chez certains sauropside à l'éclosion et durant les mues, chez certains téléostéens à l'éclosion et à la métamorphose/smoltification, et chez les mammifères à la naissance (pour revue \citealp{Wada2008}).

		% BOTTOM caption
% ------------------------
\begin{figure}[!htbp]
\centering
\vspace{1\baselineskip}
\includegraphics[width=\textwidth]
% ------------------------
%
% SIDE caption
% ------------------------
%\begin{SCfigure}[\sidecaptionrelwidth][!htbp]
%\centering\
%\vspace{1\baselineskip}
%\includegraphics[width=0.5\textwidth]
% ------------------------
%
% Main information
% ===========================================================
{Figures/gc-gr-metamorphosis/gc-gr-metamorphosis.pdf}
\caption[Niveaux de corticostérone circulant et de abondance des récepteurs aux glucocorticoïdes au cours de la métamorphose]
{
Niveaux de corticostérone circulant et abondance des récepteurs aux glucocorticoïdes au cours de la métamorphose.
La quantité relative de \gls{cort} augmente jusqu'à atteindre un pic au \gls{snf} 62.
Concomitamment, dans la queue en cours de résorption, la quantité de transcrits de \gls{gr} augmente également.
Quantification de la \gls{cort} tirée de \citet{JolivetJaudet1984}.
Quantification de \gls{gr} dans la queue tirée de \citet{Krain2004}
}
\label{fig:gc-gr-metamorphosis}
% ===========================================================
%
% BOTTOM caption
% ------------------------
\end{figure}
% ------------------------
%
% SIDE caption
% ------------------------
%\end{SCfigure}
% ------------------------
%
%
%\missingfigure{Make a figure}

		Toutefois, les actions des \glspl{gc} sur le développement des amphibiens Anoures sont difficiles à disséquer à cause de leur implication dans de nombreux processus biologiques.
		En particulier, les \glspl{gc} inhibent fortement la croissance, aussi bien chez des têtards pré- et pro-métamorphiques.
		De plus, au cours de développement post-embryonnaire des amphibiens, leurs effets varient selon le stade et le statut thyroïdien, ce qui rend l'isolation de leurs effets spécifiques indépendants des autres voies de signalisation endocrines malaisée.
		L'administration de \glspl{gc} exogènes à des têtards pré-métamorphiques inhibe en effet l'émergence des membres inférieurs \citep{Kobayashi1958}.
		L'utilisation de drogues inhibant l'action moléculaire des \glspl{gc} comme le RU486, ou empêchant la production de \glspl{gc} par les surrénales \citep{Kikuyama1982}, inhibent la métamorphose.
		De plus, les \glspl{gc} sont capables d'agir en synergie avec les \glspl{ht} chez des têtards pro-métamorphiques \citep{Hayes1993a,Hayes1994,Gray1990} (Voir section suivante).

	\subsection{Distribution des récepteurs aux glucocorticoïdes}
		Au cours de la métamorphose, les \gls{gr} ont des profils d'expression dynamiques et tissus-spécifiques :
		dans la queue (\autoref{fig:gc-gr-metamorphosis}), l'expression de \gls{gr} suit le profil de concentration des \glspl{gc}; l'intestin et le cerveau montrent une quantité de transcrits constante au cours de la pro-métamorphose, mais une légère diminution et augmentation respectivement durant le climax \citep{Krain2004}.

	\subsection{Rôles dans les interactions avec l'environnement}
		Les \glspl{gc}, via leurs effets stade-dépendant et leur implication dans la réponse au stress, peuvent moduler le développement chez la plupart des amphibiens Anoures en fonction des conditions environnementales \citep{Denver2009}.
		De façon intéressante, le Xénope, une espèce quasiment exclusivement aquatique, est marginalement affecté au niveau de son développement par un stress.
		Malgré des mécanismes d'action conservés, le type de réponse à un stress naturel peut donc être très variable, et imputable à des dynamiques de concentration en \glspl{gc} différentes, ou à des spécificités de régulation des axes \gls{hpa} et \gls{hpt}.
		Chez d'autres espèces d'Anoures au mode de vie particulièrement spécifique  et adapté à des conditions arides, les \glspl{gc} ont une action beaucoup plus spectaculaire sur l'initiation et la rapidité de déroulement de la métamorphose \citep{Kulkarni2011,Gomez-Mestre2013a}.
		L'exposition à des agents stresseurs de type prédateur peut également provoquer des modifications anatomiques, notamment au niveau de la queue \citep{Maher2013}.
		L'augmentation de la taille de la queue présente l'avantage d'apporter à l'animal une mobilité supérieure consécutive à la présence d'une musculature plus développée. Cette augmentation peut aussi changer l'intérêt du prédateur pour une proie qui ne sera peut-être plus accessible à cause de sa taille.
		Ces effets sont les conséquences de la signalisation \gls{gc}.
		Ils sont l'expression macroscopique des effets de réseaux de régulation conservés au cours de l'évolution.
		Chez \gls{xlaevis}, d'autres types de stress (agitation, prédateurs, stress prolongés, ponctuels ou chroniques) induisent des phénotypes adaptés.
		Le Xénope est donc aussi un modèle pertinent dans l'étude du stress sur la métamorphose car au niveau tissulaire et de façon indépendante des régulations centrales, les \glspl{gc} restent capables d'agir au niveau moléculaire en inhibant l'axe \gls{hpa}, et en régulant la transcription de gènes caractéristiques de cette transition développementale \citep{Hu2008,Bonett2010}.

	\subsection{Rôle du système immunitaire}
		Durant la métamorphose, la quasi-totalité des tissus larvaires sont remodelés et requièrent la résorption et la disparition d'un certain nombre de types cellulaires.
		Même le développement de tissus et d'organes adultes tels que les membres font appel à des processus d'apoptose afin d'assurer la mise en place des digitations.
		Quasiment tous les tissus subissent donc des évènements de mort cellulaire programmée, impliquent des réactions d'inflammation, et nécessitent l'activation du système immunitaire.
		En outre, durant la métamorphose, les têtards acquièrent un panel de nouvelles molécules spécifiques de la grenouille adulte (hémoglobine, kératine, vitellogénine et enzymes du cycle de l'urée), auxquelles il doivent devenir immunologiquement tolérants.
		Dans ce contexte, les \glspl{gc} sont importants durant cette transition afin de contrôler le système immunitaire, en particulier le renouvellement des populations de lymphocytes et de thymocytes larvaires qui sont en grande partie éliminés par apoptose et remplacés par des équivalents adultes \citep{Rollins-Smith1997,Schreiber2011}.

% ======= END - Roles des GC
% =====================================================

% :::::::::::::::::::::::::::::::::::::::::::::::::::::

% =====================================================
% ======= BEGIN - Interactions croisées

\section{Interactions croisées entre les hormones thyroïdiennes et les glucocorticoïdes durant la métamorphose}

	\subsection{Interactions croisées au niveau de l'axe \acrlong{hpt}}
		Bien que les \glspl{ht} soient le signal déclencheur de la métamorphose et soient requises tout au long de ce processus, elles n'agissent pas seules et leurs actions croisent celles d'autres hormones.
		Les \glspl{ht} sont en effet capables d'interagir avec la signalisation \gls{gc}, aussi bien au niveau central (\autoref{fig:switch-hpt-metamorphosis}) que périphérique.
		Alors que chez la grenouille adulte l'axe \gls{hpt} corresponde à la description donnée au \autoref{subsubsec:hpt}, chez le têtard, la synthèse de \gls{tsh} par l'anté-hypophyse est sous le contrôle de la \gls{crh} \citep{Denver1993}.
		Ce rôle double de la \gls{crh} permet au têtard de moduler la vitesse de la métamorphose en réponse à des stimuli environnementaux \citep{Denver1997a,Denver2009}.

		% BOTTOM caption
% ------------------------
\begin{figure}[!htb]
\centering
\vspace{1\baselineskip}
\includegraphics[width=\textwidth]
% ------------------------
%
% SIDE caption
% ------------------------
%\begin{SCfigure}[\sidecaptionrelwidth][!htbp]
%\centering
%\vspace{1\baselineskip}
%\includegraphics[width=0.5\textwidth]
% ------------------------
%
% Main information
% ===========================================================
{Figures/switch-hpt-metamorphosis/switch-hpt-metamorphosis.pdf}
\caption[Régulations de l'axe \gls{hpt} durant la métamorphose]
{
Régulations de l'axe \gls{hpt} durant la métamorphose.
Chez le têtard, c'est la \gls{crh} qui stimule l'hypophyse antérieure à produire la TSH.
Les \glspl{ht} ainsi synthétisées n'exercent quasiment pas de rétrocontrôle négatif, permettant ainsi à leur concentration d'augmenter au cour de la métamorphose.
Après la métamorphose, les \glspl{ht} exercent le rétrocontrole négatif classique sur la synthèse de \gls{trh} et \gls{tsh} (voir \autoref{par:hpt}).
}
\label{fig:switch-hpt-metamorphosis}
% ===========================================================
%
% BOTTOM caption
% ------------------------
\end{figure}
% ------------------------
%
% SIDE caption
% ------------------------
%\end{SCfigure}
% ------------------------
%
%
%\missingfigure{Switch CRH/TRH during metamorphosis}

	\subsection{Effets du stress sur la métamorphose \gls{ht}-dépendante}
		Durant la période larvaire, le système de la réponse au stress est étroitement imbriqué avec celui de la métamorphose \gls{ht}-dépendante.
		Il a été observé très tôt et chez différentes espèces d'amphibiens que la métamorphose était retardée ou accélérée selon le moment où les têtards étaient exposés à un stress ou à des \glspl{gc} (pré- et pro-métamorphose respectivement) \citep{Kobayashi1958,Kikuyama1983}.
		Si de façon générale, les \glspl{gc} inhibent la croissance, leur action sur la métamorphose semble en grande partie due à leurs interactions avec les \glspl{ht}.
		En effet, les \glspl{gc} sont capables (à eux seuls) de déclencher la métamorphose chez une seule espèce (\textit{Buffo boreas}, \citealp{Hayes1993a}).
		Cette action surprenante est probablement due à des niveaux d'\glspl{ht} endogènes avec lesquels les \glspl{gc} sont capables d'agir en synergie, mais insuffisants, à eux seuls, pour induire la métamorphose.

	\subsection{Effets sur les désiodases}
		Les \glspl{gc} sont connus pour favoriser l'expression des désiodases activantes et réprimer l'expression des désiodases inactivantes, contribuant à l'accumulation d'\glspl{ht} actives au sein des tissus cibles.
		Ce modèle a longtemps été à la base de l'explication de l'effet potentiateur des \glspl{gc} sur la métamorphose \gls{ht}-dépendante \citep{Galton1990}.

	\subsection{Régulation croisée de l'expression des récepteurs aux hormones thyroïdiennes et aux glucocorticoïdes}
		À l'exception de leurs effets non-génomique, ces deux hormones nécessitent leurs \glspl{rn} respectifs.
		Dans les tissus périphériques, les \glspl{gc} agissent en synergie avec les \glspl{ht} notamment via la régulation croisée au niveau de leurs \glspl{rn}.
		\par
		Les \glspl{gc} sont capables d'augmenter la capacité de liaison de la \gls{t3} dans le noyau \citep{Kikuyama1983} et la quantité de transcrits de \gls{trb} dans la queue, le cerveau et les intestins \citep{Bonett2010}.
		Réciproquement, l'expression de \gls{gr} semble être en partie contrôlée par les \glspl{ht} selon les tissus observés: alors que dans le cerveau et les intestins, un traitement à la \gls{t3} induit une diminution de l'expression de \glspl{gr}, l'opposé se passe dans la queue \citep{Krain2004}.
		Ces résultats suggèrent non seulement que les \glspl{gc} sont capables d'augmenter la sensibilité aux \glspl{ht} de certain tissus, accélérant ainsi les processus de métamorphose, mais également que la métamorphose, via l'augmentation des \glspl{ht} circulantes, induit une réponse stéréotypée au niveau de la sensibilité des différents tissus aux \glspl{gc}.

	\subsection{Intéractions croisées au niveau d'éléments \textit{cis}-régulateurs}
		Un autre niveau de régulations croisées entre hormones passe par les éléments régulateurs qui affectent leurs gènes cibles.
		Les régions régulatrices de certains gènes peuvent en effet présenter des \glspl{gre} et des \glspl{tre}, les rendant capables de répondre aux \glspl{ht} et aux \glspl{gc}, avec la possibilité de synergie locale entre ces deux hormones.
		Le gène le mieux caractérisé correspondant à ce cas de figure est \gls{klf9}.
		Chez l'amphibien, son expression est modulée par un stress ou l'administration de \glspl{gc} exogènes \citep{Bonett2009}, et présente une forte expression durant le climax de la métamorphose, de façon concomitante avec le pic d'\glspl{ht} \citep{Das2009}.
		Malgré sa forte expression dans le cerveau, ce facteur de transcription est exprimé dans la majorité des tissus; de par son profil d'expression temporel au cours du développement post-embryonnaire, le produit de ce gène semble avoir un rôle important d'intermédiaire des effets des \glspl{ht} sur les structures neuronales et leurs fonctions \citep{Denver1999,Cayrou2002}.
		Le contrôle de son expression par les \glspl{gc} s'effectue via un \gls{gre} présent en amont de son promoteur \citep{Bagamasbad2012}.
		De façon intéressante, ce \gls{gre} est conservé chez les vertébrés.
		Des études moléculaires ont pu précédemment mettre en évidence chez \gls{xtrop} la présence de trois \glspl{tre} en amont du promoteur (\citealp{Furlow2002}; données de l'équipe non publiées), dont le plus proximal est proche du \glspl{gre} préalablement décrit, suggérant que ce locus soit composite.
		Ceci confère la capacité à ce gène d'être induit par les \gls{ht} et les \glspl{gc} d'une manière au moins additive, si ce n'est synergique \citep{Bonett2010}.

% ======= END - Interactions croisées
% =====================================================

% :::::::::::::::::::::::::::::::::::::::::::::::::::::

% =====================================================
% ======= BEGIN - Utilisation en tant que modèle biologique

\section{Utilisation en tant que modèle biologique}

% -------------------------------
% +++ BEGIN - La métamorphose: un modèle de la période périnatale

	\subsection{Effet des hormones thyroïdiennes à la période post-embryonnaire}
		Chez la majorité des métazoaires, le développement embryonnaires et le stade adulte sont séparés par une période larvaire \citep{Laudet2011b,Holstein2014}.
		La métamorphose est le processus par lequel l'organisme acquiert des caractères adultes, délaissant son style de vie larvaire.
		Chez les amniotes, une telle transition écologique n'est pas morphologiquement aussi marquée.
		Pourtant, la parturition exige du fœtus qu'il acquière un certain nombre de fonctions absentes durant la gestation.
		En particulier, chez l'humain, les dernières semaines de grossesse sont caractérisées par la maturation des poumons, du système nerveux, du squelette et du système digestif.
		Ces changements s'accompagnent, aussi bien chez l'amphibien en cours de métamorphose que chez le nouveau né, d’évènements endocriniens restreints dans le temps, et requis dans le remodelage des tissus.
		Notamment, le pic d'\glspl{ht} nécessaire au déclenchement de la métamorphose et observé durant le climax entre les \glspl{snf} 62 et 64, a lieu également autour de la naissance.
		Il devient ainsi possible de tracer un parallèle étroit entre la biologie de la période périnatale chez les mammifères et la biologie de la métamorphose chez les amphibiens sur la base d'une période marquée par la signalisation thyroïdienne \citep{Laudet2011b}.
		Ceci justifie ainsi pleinement l'utilisation de la métamorphose comme modèle d'étude de l'action post-embryonnaire des hormones thyroïdiennes lors de la période périnatale.

	% +++ END - La métamorphose: un modèle de la période périnatale
	% -------------------------------
	% +++ BEGIN - Xenopus tropicalis, un modèle pertinent de génomique fonctionnelle

	\subsection{\textit{Xenopus tropicalis}, un modèle de génomique fonctionnelle}\label{subsec:xtrop-model}
		Le modèle amphibien a depuis longtemps été un modèle de choix dans l'étude du développement chez les vertébrés, en particulier par l'utilisation du Xénope \citep{Harland2011}.
		Essentiellement deux espèces du genre \textit{Xenopus} sont utilisées à des fins de recherche.

		\subsubsection{\textit{Xenopus laevis}}
			La première, \gls{xlaevis}, bénéficie d'une utilisation historique dans les cliniques et hôpitaux où cette espèce était utilisée comme test de grossesse.
			Dans son habitât naturel, géographiquement localisé aux régions sub-sahariennes de l'Afrique, \gls{xlaevis} est un animal charognard et se nourrissant de tout débris organique d'origine animale à sa portée.
			Ce comportement alimentaire confère au Xénope une adaptabilité remarquable concernant son régime, et en fait une espèce aisée à élever en animalerie.
			\gls{xlaevis} à été très utilisé durant le 20\textsuperscript{ème} siècle, notamment en biochimie, physiologie.
			De plus, les extraits d'œufs sont un matériel de choix dans le cadre d'expériences \textit{in vitro} principalement pour l'étude de la division cellulaire et du cytosquelette.
			L'oocyte est aussi un modèle \textit{in vivo} couramment utilisé pour l'étude des canaux ioniques.
			La connaissance actuelle sur l'embryogenèse et l'organogenèse des vertébrés a fortement bénéficié des travaux réalisés sur la première phase de développement de \gls{xlaevis}.
			Enfin, il est à noter que les premiers travaux de reprogrammation cellulaire ont été effectués sur ce modèle et ont été récompensés par le prix Nobel de Médecine en 2012 (Gurdon, 2012a; 2012b).
			Toutefois, à la fin du 20\textsuperscript{ème} siècle et au début du 21\textsuperscript{ème}, sa pertinence a été affectée par la difficulté de l'utiliser dans le domaine de la génétique.
			En effet, bien que son karyotype révèle la présence de 18 paires de chromosomes et suggère une diploïdie, des études ultérieures ont montré qu'un évènement de duplication du génome a eu lieu avant la divergence de \gls{xlaevis}.
			Ceci suggère que \gls{xlaevis} soit une espèce pseudo-tétraploïde, la redondance au moins partielle entre deux sets de chromosomes rendant ce modèle incommode pour l'étude de phénomènes génétiques.
			De plus, l'aire de la génomique n'a pas pu être envisagé rapidement pour cause de ce génome pseudo-tétraploïde, les technologies de séquençage et d'assemblage n'étant pas adaptées.
			Toutefois, avec l'essor des \gls{ngs} a permis de générer un premier assemblage (\url{http://www.xenbase.org}) qui n'est pas encore pertinent pour des études dans un contexte de génomique fonctionnelle (assemblage morcelé et annotation partielle).

		\subsubsection{\textit{Xenopus tropicalis}}
			La seconde, \gls{xtrop}, présente la plupart des avantages de \gls{xlaevis}, notamment sa facilité d'élevage et la taille importante de ses embryons.
			Toutefois, \gls{xtrop} est la seule espèce parmi les 20 espèces du genre \textit{Xenopus} à être diploïde, ce qui en fait un modèle particulièrement pertinent en génétique et en génomique fonctionnelle \citep{Amaya2005}.
			En outre, c'est la première espèce d'amphibien dont le génome a été séquencé.
			Un assemblage a été récemment publié \citep{Hellsten2010a}, mais souffre malheureusement d'un manque de ressources évident comparé à des organismes modèles plus populaires comme la souris ou la drosophile (pour revue, voir \citet{Grimaldi2013}; Annexe~\ref{sec:ctdb-review}).
			\gls{xtrop} reste un modèle particulièrement pertinent en génomique fonctionnelle et en endocrinologie, en particulier pour l'étude de la métamorphose et des mécanismes moléculaires sous-jacent.
			\par
			Compte tenu de la pertinence du modèle amphibien pour l'étude de la physiologie de la période post-embryonnaire fortement marquée par la signalisation thyroïdiennes et de la nécessité de ressources en génomique, \gls{xtrop} représente le modèle de choix pour lequel nous avons opté de façon à mener ces travaux.

	% +++ END - Xenopus tropicalis, un modèle pertinent de génomique fonctionnelle
	% -------------------------------
	% +++ BEGIN - Amélioration de l'assemblage et l'annotation du génome de X.tropicalis

	\subsection{Amélioration de l'assemblage et de l'annotation du génome de \textit{X. tropicalis}}
		Comme indiqué précédemment, le modèle \gls{xtrop} souffre tout de même et malheureusement d'un manque de ressources permettant d'améliorer non seulement l'assemblage de son génome et mais également son annotation \citep{Gilchrist2012}.
		Au cours de mon doctorat, j'ai participé aux efforts conduits par l'équipe d'accueil pour diminuer la fragmentation du génome et pour apporter une annotation pertinente pour la réalisation d'études de génomique fonctionnelle (\citealp{Grimaldi2013}; Annexe~\ref{sec:ctdb-review}).
		Ces efforts ont été particulièrement importants dans l'exploitation des données de \gls{chiapet} (Annexe~\ref{sec:buisine2014}) et dans une moindre mesure des résultats présentés dans ce manuscrit.
		Cependant, ces travaux m'ont permis de me familiariser avec ce génome et les outils bioinformatiques qu'il convient d'adapter pour son étude.
		\par
		En particulier, l'utilisation du \gls{gpet} permet de réduire la fragmentation de l'assemblage actuel de 60 \% en reconstruisant des chaînes de scaffolds.
		L'assemblage utilisé (version 4.1) présente de nombreux trous d'assemblages (Figure~10.1 de la revue \citealp{Grimaldi2013}; Annexe~\ref{sec:ctdb-review}).
		Curieusement, près de la moitié des trous d'assemblage font exactement 50 \gls{pb}, et correspondent à des trous de taille inconnue.
		Dans ce contexte, j'ai réutilisé les données de \gls{gpet} afin de réévaluer ces trous d'assemblage à une taille moyenne d'environs 500 \gls{pb}, correspondant à près de 15 \gls{mpb} supplémentaires (Annexe~\ref{sec:buisine2014}, section ``Large insert DNA-PET significantly reduced complexity of genome assembly'', premier paragraphe et Figures supplémentaires 3 à 7).
		Bien que le contenu en séquence connue des ces trous soit inchangé, l'estimation de leur taille réelle se révèle être une ressource précieuse, non-seulement pour l'analyse des données de \gls{ngs} produites par l'équipe, mais également dans le cadre d'expériences de biologie humide en proposant une image plus réelle de la lecture linéaire que nous pouvons avoir de la séquence du génome.
		\par
		Dans un second temps, les modèles de gènes ont été reconstruits grâce à l'utilisation combinée de la technologie de \gls{rnaseq} et de \gls{rnapet}.
		Une façon de valider les nouveaux modèles de gènes est de caractériser le positionnement de l'\gls{rnapol2} au niveau de leur partie 5'.
		En effet, de façon surprenante, l'\gls{rnapol2} tend à s'accumuler fortement à l'extrémité 5' des gènes transcrits (et certains non transcrits activement), avec une densité maximale 25 à 50 \glspl{pb} en aval du \gls{tss}.
		J'ai donc effectué un \gls{chipseq} dirigé contre \gls{rnapol2}.
		Les profils d'occupation de \gls{rnapol2} montrent bien un enrichissement 30 à 50 \gls{pb} en aval du \gls{tss}, ce qui n'est pas le cas pour une majorité des modèles de gènes publiés (Annexe~\ref{sec:buisine2014}, section ``Genome re-annotation with RNA-PET and RNA-Seq'', huitième paragraphe, Figures 4 et 5, Figure supplémentaire 27).
		Ces résultats confirment fortement la qualité des nouveaux modèles de gènes.

\end{document}