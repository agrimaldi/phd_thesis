 % -*- root: ../main.tex -*-
\documentclass[../main.tex]{subfiles}
\begin{document}

\chapter{Conséquences fonctionnelles dans l'interaction avec l'environnement}


\section{Conséquences en terme d'adaptation}

Jusqu'à présent, l'étude des régulations croisées entre \glspl{ht} et \glspl{gc} ont essentiellement été menées d'un point de vu biochimique \citep{Kikuyama1982,Galton1990}, moléculaire à l’échelle de quelques gènes \citep{Bagamasbad2008,Krain2004,Gil-Ibanez2014} et histologique \citep{Denver1993}.
Ce n’est que très récemment que ce sujet d'étude à commencé à bénéficier des techniques de mesure de l'expression de gènes à grande échelle.
La littérature reste cependant sommaire à ce sujet, et les liens entre les interactions moléculaires, les fonctions des gènes associés et leurs conséquences phénotypiques n'est pas pleinement abordé.
\par
Le travail présenté dans ce manuscrit peut être interprété dans le contexte des conséquences d'un stress sur le développement post-embryonnaire.
Les données présentées ici montrent qu'une partie des gènes régulés par les \glspl{ht} le sont également par les \glspl{gc} lors d'un co-traitement.
Mais il ressort aussi que certains gènes régulé par les \glspl{ht} subissent un effet antagoniste des \glspl{gc}.
Les conséquences fonctionnelles sont potentiellement importantes, puisque l'action des \glspl{gc} ne va pas simplement potentialiser la signalisation thyroïdienne (expliquant en partie l'accélération de la métamorphose), mais va également interférer avec celle-ci.
C'est une illustration typique de la robustesse des réseaux biologiques où un processus biologique suit son cour malgré des conditions adverses \citep{Kitano2004}.
\par
Les fonctions physiologiques intégratives et en particulier la voie \glspl{gc} de réponse au stress peuvent jouer un rôle essentiel dans la capacité des vertébrés à faire face aux changements globaux et aux environnements adverses tout en maintenant leur homéostasie \citep{Angelier2013}.
Dans ce cas, la survie immédiate peut avoir lieu au détriment d'autres fonctions biologiques comme par exemple la reproduction. Ces mécanismes adaptatifs sont malgré tout flexibles et leur rapport coût/bénéfice peut être ajusté par un rhéostat physiologique.
Par exemple ce phénomène a déjà été observé chez le Xénope où la combinaison de stress de nature différente (variations de température et accessibilité réduite à la nourriture) entrait des variations dans le minutage de l’ossification, produisant ainsi des juvéniles de petite taille et dont le squelette présente des défauts d'ossification \citep{Gomez-Mestre2010}.
Dans le cadre des traitements que nous avons réalisés sur l'animal entier, nous avons aussi pu observer une ossification prématurée qui reflète les altérations du transcriptome que nous avons caractérisées.
\par
De plus, certains gènes régulés par les \glspl{gc} sont antagonisés par les \glspl{ht}, suggérant que la réponse physiologique à un stress va potentiellement différer selon le statut thyroïdien de l'individu ou son stade de développement post-embryonnaire.
Il reste cependant difficile de spéculer sur cet aspect compte tenu qu'un seul type de traitement aux \glspl{ht} a été effectué.
\par
Il est tentant de faire le parallèle entre la balance coût/bénéfice des phénomènes adaptatifs au niveau écologique et les réponses moléculaires obtenues en laboratoire.
Ils se placent à des niveaux de résolution et des échelles très différentes qui rendent toutes comparaisons hasardeuses.
Tout au plus nous proposons des pistes qui permettraient de décrire des réponses moléculaires à certains stress (que nous ne connaissons pas).


\section{Stress et épigénome}

Il a été montré que les effets délétères d'un stress appliqué durant le développement post-embryonnaire pouvaient se manifester sous la forme de pathologies à l'age adulte (par la reprogrammation métabolique, \citealp{Drake2012,Begum2013} ou la reprogrammation du rhéostat de la réponse aux stress, \citealp{Weaver2004}).
Nous avons pu mettre en évidence que l'expression de plusieurs gènes impliqués dans le dépôt et la lecture de modifications de la chromatine (aussi bien la méthylation de l'\gls{dna} que la modification post-traductionnelle des queues d'histone) pouvait être régulée par les \glspl{ht}, les \glspl{gc}.
Il est bien caractérisé que la méthylation est indispensable au cours du développement précoce, et de la différentiation cellulaire \citep{Monk1987,Meissner2008}.
L'expression de \gls{mecp2} (impliquée dans la lecture de l'\gls{dna} méthylé, \citealp{Zou2012}) est réprimée par la \gls{t3} et la \gls{cort} et présente un légère potentiation de cet effet lorsque les deux hormones sont présentes.
La biologie de ce facteur est complexe, et il est difficile de tirer des conclusions sur sa fonction fine à l’échelle de l'organisme sur la base de son profil d'expression.
Nous savons cependant que sa mutation provoque le syndrome de Rett \citep{Amir1999}.
De plus, \gls{uhrf1} (impliqué dans la maintenance de la méthylation de l'\gls{dna}, \citealp{Bostick2007,Liu2013a}) voit son expression réprimée par la \gls{t3} et la \gls{cort}.
Son expression est fortement réprimée lorsque les deux hormones sont présentes.
Enfin, nous pourrons noter que APOBEC2 (impliqué dans les processus de déméthylation dépendants de l'hydroxyméthylation; \citealp{Guo2011}), est induit par la \gls{t3} dans le \gls{tf} mais pas dans les \glspl{hlb}.
Ceci suggère une différence tissu-spécifique de la dynamique de méthylation de l'\gls{dna}, avec une démethylation de l'\gls{dna} accrue dans la queue.
\par
La baisse de l'expression de UHRF1 pourrait ainsi provoquer une perte d'une partie de la méthylation lors de la mitose puisque \gls{dnmt1} ne sera plus addressé au niveau de sites hémi-méthylés au cours de la réplication.
Certains sites méthylés seront donc déméthylés et les gènes affectés par ce changement pourront être régulé différemment.
Nous avons aussi observé que l'expression de \gls{dnmt3a} est induite par les \gls{ht}, suggérant une méthylation \textit{de novo} au niveau de certains sites.
Il semblerait toutefois que \gls{dnmt3a} soit aussi impliquée dans la déméthylation \citep{Wu2010}.
Cependant, cette activité nécessite de faibles taux de \gls{sam}, le substrat de \gls{dnmt3a}, condition qui ne semble pas satisfaite dans notre modèle à la vue de l'expression des enzymes permettant sa synthèse.
Des mesures directes seront cependant nécessaires pour apporter une réponse définitive.
\par
L'ensemble de ces résultats, suggère deux caractéristiques fondamentales de la métamorphose.
Tout d'abord, dans le cas d'une métamorphose ayant lieu dans des conditions ``normales'', les \glspl{ht} vont être responsable de l'induction des enzymes nécessaires au dépôt de la méthylation (\gls{dnmt3a}).
Parallèlement, les enzymes nécessaires à la maintenance de la méthylation de l'\gls{dna} (\gls{uhrf1}) et à la lecture de l'information véhiculée par la méthylation de l'\gls{dna} sont réprimées (gls{mecp2} et \gls{mbd}2).
Ceci laisse penser qu'il y aurait donc une redistribution de la méthylation, afin d’autoriser la transcription de gènes ``adultes'' jusqu'alors éteints, et de réprimer l'expression de gènes ``larvaires''.
Par simple effet statistique (et si la diminution de l'expression de ces gènes se traduit bien par une diminution de la quantité de protéine correspondante dans la cellule), ceci pourrait introduire une composante stochastique accrue dans les profils de méthylation de l'\gls{dna} \citep{Xie2011,Landan2012}, et par extension, une perte de fidélité dans le maintient de la méthylation de l'\gls{dna}.
Malgré un écrasement de la variabilité de la réponse inter-cellulaire liée à son intégration au niveau des tissus, cette composante stochastique peut avoir un impact profond au niveau des tissus et de l'organisme dans son ensemble.
L'acquisition de lignées cellulaires issues de \gls{hlb} \citep{Sinzelle2012} nous permettra d'avoir plus de recul par rapport à ces aspects.
\par
Une altération ponctuelle des niveaux de méthylation pourraient aussi participer à la modulation des balances entre types cellulaires au sein d'un tissu, induisant perturbant potentiellement les fonctions biologiques de ce tissu.
Des études complémentaires dans d'autres organes sont nécessaires pour éclaircir ce point.
\par
Bien que les niveaux d'expression de ces trois gènes ne deviennent pas nuls suite à un traitement hormonal, il est possible de spéculer que la déficience du maintient et de la lecture de la méthylation de l'\gls{dna} pourrait contribuer à une sensibilité accrue aux signaux environnementaux et à la re-programmation du génome.
Nous savons qu'une partie de la diversité phénotypique au sein d'une population repose sur une variabilité génétique.
Ces résultats suggèrent que d'autres composantes pourraient entrer en jeu.
En particulier, une variabilité des profils d’expression de gènes pouvant être aggravée par des facteurs environnementaux pourrait être à l'origine de phénotype moléculaires altérés se traduisant à plus ou moins long terme par des phénotypes macroscopique ou physiologiques particuliers.



%%%%%%%%%%%%%%%%%%%%%%%%%%%%%%%%%%%%%%%
%%%%%%%%%%%%%%%%%%%%%%%%%%%%%%%%%%%%%%%
%%%%%%%%%%%%%%%%%%%%%%%%%%%%%%%%%%%%%%%

%Comme mentionné dans le \autoref{chap:metamorphosis}, la signalisation thyroïdienne et la réponse au stress sont deux voies étroitement imbriquée.
%Jusqu'a présent, l'étude des régulations croisées entre \glspl{ht} et \glspl{gc} ont éssentiellement été menées d'un point de vu biochimique \citep{Kikuyama1982,Galton1990}, moléculaire à l'échelle de quelques gènes \citep{Bagamasbad2008,Krain2004} histologique \citep{Denver1993}.
%Ce n'est que très récemment que ce sujet d'étude à commencé à bénéficier des techniques de mesure de l'expression de gènes à grande échelle.
%La littérature reste cependant sommaire à ce sujet, et le lien entre les intéractions croisées et leur conséquences phénotypique n'est pas pleinement abordé.
%Le travail présenté dans ce manuscrit vise donc en partie à lier entre elles différentes facettes des conséquences à court terme d'un stress sur le développement post-embryonaire.
%\par
%Comme mentionné précédemment, les données présentées ici confirment l'pbservation de longue date que les \glspl{gc} modulent les niveaux d'\glspl{ht} disponibles en favorisant la transcription du gène codant pour \gls{dio2} et en réprimant celle de \gls{dio3}.
%Toutefois, si l'effet des \glspl{gc} ne passait que par ce mécanisme, on s'attendrait à ce que l'immense majorité des gènes régulés par les \glspl{ht} le soient également par les \glspl{gc} lors d'un co-traitement.
%De plus, dans ce contexte, les effets d'"antagonisme" ne sont pas expliqués.
%\par
%Le simple fait que les réponses soient aussi différentes selon le tissu et que les profils de type "potentiation" ne soient (malgré leur sur-représentation) pas les seuls suggère fortement des mécanismes de régulation des gènes au cas par cas.
%\par
%Les conséquences conceptuelles sont fortes, puisque cela suggère que l'action des \glspl{gc} ne va pas simplement augmenter le signal thyroïdien (expliquant en partie l'accélération de la métamorphose), mais également interférer avec celui-ci.
%Les conséquences pour l'organisme peuvent être profondes, puisque ce sont des fonctions biologiques particulières qui vont être affectée selon l'effet des \glspl{gc} sur la régulation de la transcription \glspl{ht}-dépendante.
%\par
%\explainindetail{
%La métamorphose chez l'amphibien est un processus qui fait intervenir le remodelage quasi complet de l'organisme.
%De plus, certaines fonctions biologiques vont véritablement être mises en pause, ce qui va être necessaire à cette transition.
%Certains aspects du système immunitaire sont ainsi freinés pour éviter que le nouveau soi ne soit visé par le système immun.
%Les glucocorticoides, dont l'action anti-inflammatoire est bien connue et décrite peuvent par conséquent jouer un rôle particulièrement important durant la métamorphose.
%}
%
%\subsection{conséquences pour l'individu}
%Les conséquence des régulations croisées par les \glspl{ht} et les \glspl{gc} pourraient ainsi avoir des effets à plusieurs niveau sur l'individu.
%
%On peut voir la perturbation ponctuelle de la signalisation thyroïdienne par des agents stresseurs comme une modulation d'un phénomène biologique intrinsèque.
%
%\comment{Perturbation signalisation \glspl{ht} par environnement exterieur => effets délétères (squelette, métabolisme, structures)
%Cependant, ce type de réponse à été sélectionné.
%Donc soit il offre également un avantage évolutif durant la vie (programmation des réhostats métaboliques à la naissance, qui restent ou non pertinents durant la vie adulte. pathologie quand plus adapté).
%Soit réponse nécessaire pour survie immédiate du nouveau-né.
%Conséquence: advienne que pourra}
%
%\subsubsection{court terme}
%Dans le contexte de la métamorphose ou de la période périnatale, la régulation de la signalisation thyroïdienne par des agents extérieurs (stress par exemple) peut-être vu comme un mécanisme fin d'adaptation à un environnement adverse.
%En effet, ce que l'on perçoit comme des conséquences délétères semblent avoir pour origine des mécanismes communs ayant lieu à un moment précis du développement, et faisant l'objet de pressions de sélection.
%À court terme, la modulation du signal thyroïdien pourrait ainsi tout simplement être un mécanisme de survie immédiat, assurant la viabilité de l'organisme dans un environnement offrant des contraintes particulière.
%Chez le Xénope, il a ainsi été observé que la combinaison de stress de nature différente (réduction du niveau d'eau et accéssibilité réduite à la nourriture) entrainaient des variations dans le timing de l'ossification, produisant ainsi des juvéniles de petite taille et dont le squelette est prématurément ossifié.
%Dans cet exemple, l'action des \glspl{gc} est principalement d'accélérer les processus de la métamorphose \glspl{ht}-dépendante.
%Les \glspl{gc} agiraient alors comme un senseur du stress, assurant la survie à court terme des têtards pour qu'il passent le plus rapidement possible le cap de la métamorphose.
%À une échelle moléculaire, ceci est reflété par exemple par "l'extinction" accrue du système immunitaire inné et une accélération des processus de mise en place du squelette et de remodelage de la matrice extra-cellulaire.
%
%\subsubsection{long terme}
%Cependant, les conséquences de l'interaction entre ces deux voies de signalisation à une période précise du développement post-embryonnaire peuvent se manifester sous la forme de pathologies plus tard à l'age adulte.
%Le développement embryonnaire et le développement post-embryonnaire se démarquent l'un de l'autre notamment par le contact de l'organisme avec son environnement.
%Chez le premier, l'environnement intra-utérin permet de partiellement tamponner les actions de l'environnement exterieur. Seuls les stress majeurs sont ainsi susceptibles d'affecter le développement, et de façon modérée ou très spécifique, la viabilité de l'embryon étant rapidement mise en jeu.
%Chez le second, le nouveau né est en général en contact direct avec l'environnement.
%De plus, à la période périnatale, les organes formés sont en général déjà fonctionnels mais subissent une maturation visant à l'établissement de fonctions additionnelles qui n'était pas nécessaires pour la vie fœtale.
%Ainsi, cette transition de milieu et de mode de vie s'accompagne de la mise en place de rhéostats métaboliques à laquelle les \glspl{ht} participent et que les conditions environnementales vont pouvoir partiellement moduler via des perturbations labiles.
%\citet{Angelier2013} ont récemment élaboré un cadre théorique dans lequel pourraient s'inscrire les résultats obtenus ici.
%En particulier, il est séduisant d'imaginer que le phénotype métabolique et homéostatique des grenouilles juvéniles issues de têtards ayant subit un stress durant la métamorphose pourrait être adapté à cet environnement particulier.
%L'aspect délétère du phénotype ne se révèlerait alors que dans le contexte d'un nouvel environnement dont les contraintes sont différentes (fréquence ou nature des perturbations labiles), et correspondant aux limites de la flexibilité phénotypique.
%\explainindetail{on utilise \glspl{gc} directement, qui tempère la réponse au stress, mais le stress en lui même qui peut être de nature variée => certains aspects passent à la trappe dans notre modèle. Cependant pour détail moléculaire d'un CT particulier, pertinent car interaction avec nombreuse autres voies.}
%
%\subsection{Conséquences possibles à l'échelle d'une population}
%
%Il est maintenant de plus en plus admis que non seulement différentes espèce ne vont pas répondre de la même façon à un stress, mais que les individus même d'une population ne vont pas répondre de la même façon à un stress \insertref{ref variabilité réponses au sein d'une population}.
%Cet aspect est particulièrement bien illustré par la grande diversité du phénotype squelettique des têtards traités à la \gls{t3} et à la \gls{cort} (voir \autoref{sec:grimaldi2014}), et des grenouilles juvéniles issues de têtards traités ponctuellement avec de la \gls{cort} (données préliminaire non-illustrées dans ce manuscrit).
%\par
%Bien que cette diversité phénotypique ne soit pas directement sous-tendue par de la variabilité génétique, il n'en demeure pas moins que les mécanismes moléculaires des régulations croisées entre différentes voies de signalisation le sont.
%Ainsi, la capacité de réponse à une perturbation labile par un individu peut, elle, être soumise à des pression de sélection qui vont potentiellement enrichir la population en individu capable de répondre de façon appropriée à un stress.
%\par
%Il important de remarquer que dans le cas de la manifestation d'un phénotype délétère auquel il est difficile d'attribuer un avantage séléctif (comme c'est le cas pour le squelette), il se peut que les individus d'une telle population ne soient séléctionés que sur leur capacité de survie précoce face aux perturbation labiles.
%En outre, on peut spéculer qu'une telle population sera potentiellement plus à même de perdurer face à la répétition de la même perturbation.

\end{document}