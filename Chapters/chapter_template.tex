 % -*- root: ../main.tex -*-
\documentclass[../main.tex]{subfiles}
\begin{document}

\chapter{MM}


\section{section 1}


% =====================================================
% ======= BEGIN - Généralités

\subsection{Généralités}

La curiosité et la santé humaine sont les deux principaux catalyseurs sous-jacent au besoin humain d'élargir sa connaissance en science et en biologie.
L'historique de la biologie de la glande thyroïde et des \glspl{ht} illustre parfaitement cela.
Dès le 15ème et le 16ème siècle, des anatomistes décrivaient la glande thyroïde et réalisaient que le goitre correspondait à une hypertrophie de cet organe.

% ======= END - Généralités
% =====================================================

% :::::::::::::::::::::::::::::::::::::::::::::::::::::

% =====================================================
% ======= BEGIN - Métabolisme des HT

\subsection{subsection 2}

% -------------------------------
% +++ BEGIN - Synthèse des HT

\subsubsection{subsec 2.1}

blabla
blbla

\end{document}